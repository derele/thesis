% this file is called up by thesis.tex
% content in this file will be fed into the main document

% Glossary entries are defined with the command \nomenclature{1}{2}
% 1 = Entry name, e.g. abbreviation; 2 = Explanation
% You can place all explanations in this separate file or declare them in the middle of the text. Either way they will be collected in the glossary.

% required to print nomenclature name to page header
\markboth{\MakeUppercase{\nomname}}{\MakeUppercase{\nomname}}


% ----------------------- contents from here ------------------------

%  
\nomenclature{SNP}{Single Nucleotide Polymorphism; variation occurring
in a single nucleotide between two closely related homlogous
sequences. Leading to for example to allelic differences within a
population or even the homologous chromosomes in an individual}


\nomenclature{ORF}{Open Reading Frame; a region in a DNA-sequence
begining with a start-codon and not containing a stop-condon. For
example a region within a processed mRNA transcript being transcribed
into a protein}

\nomenclature{dpi}{Days post infection; In infection experiments, a
point in time given in days after an individual has been infected}

\nomenclature{DNA}{Desoxy Ribonucleic Acid; a chemical molecule
  bearing the heritable genetic information in all life on earth}


