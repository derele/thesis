% Dissertation Zusammenfassung -----------------------------------------------------


%\begin{abstractslong}    %uncommenting this line, gives a different abstract heading
\begin{zusammenfassung}        %this creates the heading for the
                               %abstract page it is the german version
                               %of "abstract"


  Die F\"ahigkeit sich in neuen Umgebungen und Nieschen auszubreiten,
  obwohl sie h\"ochst angepasst an ihren angestammten Lebensraum sind,
  stellt eine faszinierende Eigenschaft von Lebenwesen dar. Der
  Wechsel der Wirtsart durch \textit{Anguillicola crassus} kann als
  Modell f\"ur einen Extremfall dieses Vorganges gesehen werden, bei
  dem Parasiten neue Wirte besiedeln. Selektion in solch einer neuen
  Umgebung, die zu einer Anpassung f\"uhrt gilt als eien treibende
  Kraft f\"ur Divergenz und so zum Entstehen neuer Arten und
  biologischer Vielfalt.  Gen-regulatorische Netzwerke, als eine
  Br\"ucke zwischen Genotyp and Phenotyp, haben eine zentrale Rolle
  sowohl in der Antwort auf Stress (etwa durch eine ver\"anderte
  Umwelt) als auch in der Entwicklung von Barrieren f\"ur die
  Fortpflanzung.

  Im hier vorgestellen Projekt sollen die Unterschiede im Transkriptom
  zweier Populationen von \textit{A. crassus} untersucht werden. Der
  Parasit wurde vor 30 Jahren nach Europa eingeschleppt, wo er sich
  erfolgreich in einer neuen Wirtsart ausbreitet und etablierte.



\end{zusammenfassung}
%\end{abstractlongs}


% ---------------------------------------------------------------------- 



%%% Local Variables: ***
%%% mode: latex ***
%%% TeX-master: "../thesis.tex" ***
%%% tex-main-file: "../thesis.tex" ***
%%% End: ***
