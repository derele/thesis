% Dissertation Zusammenfassung -----------------------------------------------------


%\begin{abstractslong}    %uncommenting this line, gives a different abstract heading
\begin{zusammenfassung}        %this creates the heading for the
                               %abstract page it is the german version
                               %of "abstract"


  Die F\"ahigkeit sich in neuen Umgebungen und Nieschen auszubreiten,
  obwohl sie h\"ochst angepasst an ihren angestammten Lebensraum sind,
  stellt eine faszinierende Leistung von Lebenwesen dar. Vor 30 Jahren
  wurde der Schwimmblasen-Nematode \textit{Anguillicola crassus} aus
  Asien, wo er \textit{Anguilla japonica} parastiert, nach Europa
  eingeschlppt und breitete sich hier in der neuen Wirtsart
  \textit{Anguilla anguilla} aus. Ob und in wie weit ph\"anotypische
  Plastizit\"at oder die schnelle Anpassung an unterschiedliche
  Selektionsdr\"ucke zum Erfolg der Invasion beitragen stellt eine Frage
  von großer evolutionsbiologischer Bedeutung dar.

  Gen-regulatorische Netzwerke, als eine Verbindung zwischen Genotyp
  and Ph\"anotyp, haben eine zentrale Rolle sowohl in der Antwort auf
  Stress (etwa durch eine ver\"anderte Umwelt) als auch in der lokalen
  Anpassung.

  Im hier vorgestellen Projekt wurden Unterschiede in der
  Gen-Expression zwischen Populationen von \textit{A. crassus}
  untersucht und erbliche Komponenten dieser Unterschiede in einem
  Kreuzinfektions-Experiment mit Asiatischen und Europ\"aischen Wirten
  und Parasiten isoliert.

  Mehrere Proteasen zeigten Spuren positiver Selektion auf der
  Sequenz-Ebene und heben diese Gruppe von Enzymen als ein mögliches
  Ziel des Immunangriffs auf \textit{A. crassus} hervor. Auf der
  Expressions-Ebene \"uberwiegen erbliche Ver\"anderungen gegen\"uber
  Modifikationen in unterschiedlicher Wirts-Umgebung. Mitochondrial
  codierte Enzyme der Atmungskette zeigten unterschiedliche Expression
  in Europ\"aischen und Asiatischen Populationen des Parasiten,
  Collagen-Gene der Cuticula zeigten ``angepasste'' Expressionsmuster
  in Wirt-Parsit Paaren gemeinsamer Herkunft.


\end{zusammenfassung}
%\end{abstractlongs}


% ---------------------------------------------------------------------- 



%%% Local Variables: ***
%%% mode: latex ***
%%% TeX-master: "../thesis.tex" ***
%%% tex-main-file: "../thesis.tex" ***
%%% End: ***
