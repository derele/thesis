
% Thesis Abstract -----------------------------------------------------


%\begin{abstractslong}    %uncommenting this line, gives a different abstract heading
\begin{abstracts}        %this creates the heading for the abstract page

The ability to expand into new environments and niches, despite being
highly adapted for survival in their angestammten environment, is a
fascinating feat of organisms. The propensity of \textit{Anguillicola
  crassus} to capture new hosts can serve as a model for an extreme
case of this, in which parasites accquire new hosts. Selection in such
new environments leading to adaptation is considered a driving force
of divergence and thus for the origin of species and biotic diversity.

Gene regulatory networks, as a bridge between genotype and phenotype,
are thought to play a central role both in the response to stress
(e.g. from sofar unexperienced environmental stressors) and in the
divergence and eventually establisment of reproductive barriers
between populations.

In the present project the differences in gene-expression in
\textit{A. crassus} populations should be illuminated. The parasite
was introduced to Europe 30 years ago, spread successfully in a new
host and established stable populations. 

\end{abstracts}
%\end{abstractlongs}


% ---------------------------------------------------------------------- 

%%% Local Variables: ***
%%% mode:latex ***
%%% TeX-master: "../thesis.tex"  ***
%%% tex-main-file: "../thesis.tex" ***
%%% End: ***
     