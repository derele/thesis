
% Thesis Abstract -----------------------------------------------------


%\begin{abstractslong}    %uncommenting this line, gives a different abstract heading
\begin{abstracts}        %this creates the heading for the abstract page

  The ability to expand into new environments and niches, despite
  being highly adapted for survival in their habitual environment, is
  a fascinating feat of organisms. The propensity of
  \textit{Anguillicola crassus} to capture new hosts can serve as a
  model for an extreme case of this, as parasites are thought to live
  in competition with and closely adapted to their hosts. Differential
  selection in such new environments leading to local adaptation is
  considered a driving force of divergence and thus for the origin of
  species and biotic diversity.

  \textit{A. crassus} is an ecologically, economically and
  evolutionary interesting nematode. It has been introduced from Asia,
  where it parasitises the Japanese eel \textit{Angilla japonica}, to
  Europe ~30 years ago. Today it infects stocks of the endangered,
  commercially exploited European eel \textit{Anguilla anguilla},
  permitting and necessitating research in a newly established
  host-parasite system. Furthermore phylogenetics places
  \textit{A. crassus} at a key position for the emergence of
  parasitism, basal to one of the major clades of parasitic nematodes.

  Gene regulatory networks, as a bridge between genotype and
  phenotype, are thought to play a central role both in the response
  to stress (e.g. from sofar unexperienced environmental stressors)
  and in the divergence and eventually establisment of reproductive
  barriers between populations.

  In the present project the differences in gene-expression in
  \textit{A. crassus} populations should be illuminated and genetic
  components of differences isolated. With this aim we conducted
  cross-innoculation experiments with both Asian and European
  host-species and parasite populations. 

  In our sequence assembly we identified twelve proteases under
  positive seleciton highlighting this group of enzymes as possible
  targets of an immune-attack on \textit{A. crassus}.
  
  In 






\end{abstracts}
%\end{abstractlongs}


% ---------------------------------------------------------------------- 

%%% Local Variables: ***
%%% mode:latex ***
%%% TeX-master: "../thesis.tex"  ***
%%% tex-main-file: "../thesis.tex" ***
%%% End: ***
     