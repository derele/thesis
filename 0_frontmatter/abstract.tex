
% Thesis Abstract -----------------------------------------------------


%\begin{abstractslong}    %uncommenting this line, gives a different abstract heading
\begin{abstracts}        %this creates the heading for the abstract page

  The ability to expand into new environments and niches, despite
  being highly adapted to a habitual environment, is a fascinating
  feat of organisms. 30 years ago \textit{Anguillicola crassus} has
  been introduced from Asia, where it parasitises \textit{Angilla
    japonica}, to Europe and spread here in the new host species
  \textit{Anguilla anguilla}. Whether and how much phenotypic
  plasticity or the rapid adaptation to differential selection are
  contributing to the success invasion of new host-populations is a
  question of big evolutionary interest.

  Gene regulatory networks, as a an important link between genotype
  and phenotype, are thought to play a central role both in the
  response to stress (e.g. from sofar unexperienced environmental
  stressors) and in local adaptation.

  In the present project differential gene-expression in
  \textit{A. crassus} populations were illuminated and genetic
  components of differences were isolated in cross-inoculation
  experiments with both Asian and European host-species and parasite
  populations.

  Several proteases were shown to be under positive seleciton on the
  sequence level, highlighting this group of enzymes as possible
  targets of an immune-attack on \textit{A. crassus}. On the
  gene-expression level the extent of heritable change was large in
  comparison to the effect of modification in different
  host-environments. Mitochondrially encoded subunits of the
  respiratory chain showed diverged expression patterns in European
  vs. Asian parasite populations, cuticle collagen genes showed
  ``adapted'' patterns of expression in host-parasites pairs of shared
  origin.

\end{abstracts}
%\end{abstractlongs}


% ---------------------------------------------------------------------- 

%%% Local Variables: ***
%%% mode:latex ***
%%% TeX-master: "../thesis.tex"  ***
%%% tex-main-file: "../thesis.tex" ***
%%% End: ***
     