% this file is called up by thesis.tex
% content in this file will be fed into the main document

%: ----------------------- name of chapter  -------------------------
\chapter{Pilot sequencing (Sanger method)} % top level followed by section, subsection
\label{pilot-seq}

%: ----------------------- paths to graphics ------------------------

% change according to folder and file names
\ifpdf
    \graphicspath{{3_pilot/figures/PNG/}{3_pilot/figures/PDF/}{3_pilot/figures/}}
\else
    \graphicspath{{3_pilot/figures/EPS/}{3_pilot/figures/}}
\fi

%: ----------------------- contents from here ------------------------

\subsection{Overview}

This chapter reports a small pilot-project investigating the
RNA-extraction and cDNA preparation in preparation for high-throughput
transcriptome sequencing of the swimbladder nematode
\textit{A. crassus}. I generated expressed sequence tags (ESTs) using
traditional Sanger-technology and conducted a first assessment of the
sequence diversity expected in deeper sequencing. Especially the
expected coverage of unwanted rRNA and host-derived sequences was
investigated.

In total 945 reads from adult \textit{A. crassus} (5 libraries from 4
cDNA preparations, including 541 sequences generated by students in a
laboratory course) and 288 reads from liver-tissue of the host species
\textit{An. japonica} (3 libraries from 3 cDNA preparations) were
sequenced.

\subsection{Initial quality screening}
\label{qual-pil}

The initial quality screening revealed a high number of sequences that
had to be discarded due to failed sequencing reactions (sequences
being too short after quality trimming by \texttt{trace2seq}) in the
library prepared by students. For sequences of \textit{An. japonica}
and the other libraries from \textit{A. crassus}, failed sequencing
reactions were less common.

In the next screening-step for \textit{A. crassus} 125 (13.23\%) and
for \textit{An. japonica} 64 (22.22\%) of the sequences were
excluded because of homopolymer-runs considered to be artificial. This
resulted in 452 of the nematode and 195 of the host reads being
regarded of sufficient quality for further processing after
base-calling and quality screening.

\subsection{rRNA screening}
\label{rRNA-pil}

The further screening of sequences revealed a high abundance of rRNA
(see Figure \ref{rRNA-con}) ranging from 71.67\% to 91.67\% of the
obtained sequences.  High abundances of rRNA were also found in the
libraries from host liver tissue (see table \ref{tab:num}), ranging
from 71.67\% to 77.42\%. This contamination in libraries from both
species was mainly responsible for a low number of sequences being of
sufficient quality for submission to NCBI-dbEST. At this point for the
\textit{An. japonica}-dataset, 36 sequences were submitted to
NCBI-dbEST under the Library Name ``\textit{Anguilla japonica} liver''
and were assigned the accession LIBEST\_027503.

\figuremacro{rRNA-con}{Proportion of rRNA in different libraries for
  \textit{A. crassus} and \textit{An. japonica}}{rRNA abundance as
  proportion of the raw sequencing-reads (rRNA from total) and as
  proportion of the reads after quality screening (rRNA from
  good). Libraries starting with ``Ac\_'' are from \textit{A. crassus},
  libraries starting with ``Aj\_'' are from \textit{An. japonica}.}

% latex table generated in R 2.13.0 by xtable 1.5-6 package
% Mon Sep 19 15:44:26 2011
\small\begin{table}[ht]
\begin{center}
\begin{tabular}{rrrrrr}
  \hline
 & short & poly & rRNA & fishpep & good \\ 
  \hline
Ac\_197F(n=96) &   4 &  17 &  58 &   1 &  16 \\ 
  Ac\_106F(n=96) &  25 &   9 &  48 &   0 &  14 \\ 
  Ac\_M175(n=116) &  30 &  19 &  41 &   3 &  23 \\ 
  Ac\_FM(n=96) &  12 &  29 &  34 &   1 &  20 \\ 
  Ac\_EH1(n=541) & 297 &  51 & 143 &   8 &  42 \\ 
  Ac\_total(n=945) & 368 & 125 & 324 &  13 & 115 \\ 
  Aj\_Li1(n=96) &  10 &  23 &  50 &  &  13 \\ 
  Aj\_Li2(n=96) &  10 &  26 &  43 &  &  17 \\ 
  Aj\_Li3(n=96) &   9 &  15 &  66 &  &   6 \\ 
  Aj\_total(n=288) &  29 &  64 & 159 &  &  36 \\ 
   \hline
\end{tabular}
\caption[Screening statistics for pilot sequencing] {\textbf{Screening
    statistics for pilot sequencing} - Number of ESTs discarded at
  each screening-step for single libraries and totals for
  species. Short, sequence to short in \texttt{trace2seq}; poly,
  sequences with artificial homopolymer-runs from poly-A tails; rRNA,
  with hits to rRNA databases; fishpep, with better hits to
  host-protein-databases than to nematode protein databases; good,
  sequences regarded ``valid'' after all screening steps. Note that
  the 13 sequences in the \textit{A. crassus}-dataset, for which
  fish-origin was inferred, were still submitted to NCBI-dbEST.}

\label{tab:num}
\end{center}
\end{table}
\normalsize

\subsection{Screening for host-contamination}
\label{host-pil}

For the \textit{A. crassus}-dataset screening for host-sequences at
this stage was regarded necessary based on the notion that a large
proportion of the tissue prepared in RNA extraction consisted of
eel-blood inside the gut of the worms (see also Figure
\ref{worm_diff}). Additionally, a bimodal distribution of GC-content
in the \textit{A. crassus}-dataset was observed with one of the modes
consistent with the mean GC-content of the ESTs from the Japanese eel.

Comparison of \texttt{Blast}- results for these sequences versus
nempep4 and a fishprotein-database (derived from NCBI non-redundant),
showed that 13 sequences were more likely to originate from host
contamination than from \textit{A. crassus}. These 13 sequences in the
\textit{A. crassus} data-set were submitted to NCBI-dbEST with a
comment that host origin had been inferred. This reduced the dataset
essentially to 115 ESTs. However, these 13 ESTs are still accessible
through the same library name ``Adult \textit{Anguillicola crassus}''
and library-identifier LIBEST\_027505 and are taxonomically attributed
to \textit{A. crassus} on NCBI-dbEST.

\figuremacro{GC-pil}{GC-content of sequences from \textit{An. japonica}
  and \textit{A. crassus}}{The Japanese eel has a slightly higher
  GC-content than the parasite: This sequence characteristic is useful
  for separation of sequences from the host-parasite interface, note
  the higher GC-content of the sequences from \textit{A. crassus}, for
  which host origin was inferred from similarity searches (red line
  labeled \textit{A. crassus/An .japonica}).}

After screening of host-sequences the GC-content of
\textit{A. crassus} ESTs had a unimodal distribution (see Figure
\ref{GC-pil}). \textit{A. crassus} had a lower mean GC-content $(37.32
\pm 8.36$ mean $\pm$ sd) than \textit{An. japonica} $(45.79 \pm
8.36$ mean $\pm$ sd; two-sided t-test $p<0.001$). The distribution of
the GC-contents for sequences, for which host-origin was inferred was
in agreement with the GC-distribution for host sequences.

\texttt{Blast}-annotations obtained (by similarity searches against
NCBI-nr, bit-score threshold of 55) for the sequences of putative host
origin were also largely in agreement with the expectations for
eel-blood: one sequence could be identified being highly similar to
``hemoglobin anodic subunit'' from the European eel. Others were
annotated with best hits to highly expressed housekeeping genes from
fish or vertebrates (see table \ref{tab:hostan}). Two sequences in the
set had lower similarities only to proteins predicted from
genome-sequences of chordates, and one sequence of the 13 lacked any
similarity to NCBI-nr above the threshold of 55 bits.

115 of the submitted sequences for ``Adult \textit{Anguillicola
  crassus}'' (LIBEST\_027505) were regarded as ``valid'', i.e. not
clearly of host origin.

However, two ESTs (Ac\_EH1f\_01D10 and Ac\_EH1r\_01D10; forward and
reverse read of the same clone) were annotated with
``ref$|$ZP\_05032178.1$|$; exopolysaccharide synthesis, ExoD
superfamily'' from \textit{Brevundimonas} sp. BAL3. The family
Caulobacteraceae, comprises bacteria living in freshwater and
sequences are probably derived from a commensal, symbiont or pathogen
of eels or swimbladder-nematodes. These off-target data were left in
the submission file.

For 66 (58.4\%) of the remaining 113 ESTs annotations were obtained
from orthologous sequences. All of these orthologous sequences were
from other species in the phylum nematoda.

% Sun Oct  2 07:46:25 2011
\begin{sidewaystable}[ht]
\begin{center}
\begin{tabular}{lrp{3.5cm}lrr}
  \hline
  sequence & hit identifier & hit description &species & bit-score & e-value \\
  \hline
  Ac\_EH1f\_005B07 & gb$|$AAQ97992.1$|$ & cyclin G1 & \textit{Danio rerio} & 67.0 & 9e-10 \\ 
  Ac\_EH1f\_01A02 & gb$|$ACO10003.1$|$ &  Nicotinamide riboside kinase 2 & \textit{Osmerus mordax} &  333 & 1e-89 \\ 
  Ac\_EH1f\_01C10 & gb$|$ADF80517.1$|$ & ferritin M subunit & \textit{Sciaenops ocellatus}  &  328 & 5e-88 \\ 
  Ac\_EH1r\_004A04 & ref$|$XP\_003340320.1$|$ & cytoplasmic 1-like actin & \textit{Monodelphis domestica}  &  102 &3e-20 \\ 
  Ac\_EH1r\_005B07 & gb$|$ABN80454.1$|$ & cyclin G1  &\textit{Poecilia reticulata}  & 90.5 & 8e-17 \\ 
  Ac\_EH1r\_009C03 & ref$|$NP\_001122208.1$|$ & THAP domain containing protein 4 & \textit{Danio rerio}  &  176 & 1e-42 \\ 
  Ac\_EH1r\_01A07 & sp$|$P80946.1$|$ & Hemoglobin anodic subunit beta &\textit{Anguilla anguilla} & 283 &1e-74 \\ 
  Ac\_FMf\_08F03 & ref$|$XP\_003226802.1$|$ & cohesin subunit SA-2-like isoform 2 & \textit{Anolis carolinensis}  &  219 &8e-56 \\ 
  Ac\_M175\_01H02 & emb$|$CAQ87569.1$|$ & NKEF-B protein & \textit{Plecoglossus altivelis}  &  365 &3e-99 \\ 
  Ac\_197Ff\_01E04 & ref$|$XP\_002121150.1$|$ & CUB and sushi domain-containing protein 3 & \textit{Ciona intestinalis} & 80.5 & 2e-13 \\ 
  Ac\_EH1f\_01D07  & ref$|$XP\_002606965.1$|$ & hypothetical protein & \textit{Branchiostoma floridae} & 82.8 &3e-14 \\ 
  Ac\_M175\_01B06 & ref$|$XP\_422710.2$|$ & hypothetical protein & \textit{Gallus gallus} &  123 &1e-26 \\
  \hline
\end{tabular}
\end{center}
\caption[Annotation of putative host-derived sequences in the
\textit{A. crassus}-dataset]{\textbf{Annotation of putative
    host-derived sequences in the \textit{A. crassus}-dataset} -
  Sequences excluded because of inferred host-origin comparing
  similarity to nematode- and fish-proteins. The annotation obtained
  against NCBI-nr are in agreement with this inference of host origin,
  as only best hits to vertebrate proteins are found.}
\label{tab:hostan}
\end{sidewaystable}

%%% Local Variables: ***
%%% mode:latex ***
%%% TeX-master: "../thesis.tex"  ***
%%% tex-main-file: "../thesis.tex" ***
%%% End: ***
     
