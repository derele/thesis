
% this file is called up by thesis.tex
% content in this file will be fed into the main document

\chapter{Materials \& methods} % top level followed by section, subsection


% ----------------------- paths to graphics ------------------------

% change according to folder and file names
\ifpdf
    \graphicspath{{8/figures/PNG/}{8/figures/PDF/}{8/figures/}}
\else
    \graphicspath{{8/figures/EPS/}{8/figures/}}
\fi

% ----------------------- contents from here ------------------------


\section{Sampling of worms from wild eels (Sanger- and
  pyro-sequencing) }

Cultured eels were acquired from an aquaculture directly adjacent to
Kaoping river (22.6418N; 120.4440E) 15km stream upwards from it's
estuary. Wild eel were bought from a fisherman, fishing in the estuary
of Kao-Ping river (22.5074N; 120.4220E). All eels were transported to
the Institute of Fisheries Science at the National Taiwan University
in Taipei in aerated plastic bags, where they were sheltered until
dissection.

Eels were decapitated, length (to the nearest 1.0mm) and weight (to
the nearest 0.1g) were measured, and sex was determined by visual
inspection of the gonads. The swimbladder was opened, adult worms were
removed from the lumen with a forceps, their sex was determined, and
they were counted. All adult \textit{A. crassus} were preserved in
RNAlater(Quiagen, Hilden, Germany) in individual plastic tubes.

Worms from the European eel were sampled in Sniardwy Lake, Poland
(53.751959N ,21.730957E) by Urszula Weclawski and from the
Linkenheimer Altrhein, Germany (49.0262N; 8.310556E), following a
procedure similar to the one described above for worms from Taiwan.

\section{RNA-extraction and cDNA synthesis for Sanger- and
  454-sequencing}

Total RNA was extracted from single, whole worms using the RNeasy kit
(Quiagen, Hilden, Germany), following the manufacturers
protocol. Alternatively parts of the liver of the host species
\textit{Anguilla japonica}, which also had been preserved in RNAlater
were used for RNA extraction, following the same protocol.

The Evrogen MINT cDNA synthesis kit (Evrogen, Moscow, Russia ) was
then used to amplify mRNA transcripts according to the manufacturers
protocol. It uses an adapter sequence at 3' the end of a a poly
dT-primer for first strand synthesis and adds a second adapter
complementary to the bases at the 5' end of the transcripts by
terminal transferase activity and template switching. Using these
adapters it is possible to specifically amplify mRNA enriched for
full-length transcripts.

\section{Cloning for Sanger-sequencing}

The obtained cDNA preparations were undirectionally cloned into
TOPO2PCR-vectors (Invitrogen, Carlsbad, USA) and TOP10 chemically
competent cells (Invitrogen, Carlsbad, USA) were transformed with this
construct. The cells were plated on LB-medium-agarose containing
Kanamycin (5mg/ml), xGal
(5-bromo-4-chloro-3-indolyl-$\beta$-D-galactopyranoside) and IPTG
(Isopropyl-$\beta$-D-1-thiogalactopyranosid). After 24h of incubation
at $36\,^{\circ}\mathrm{C} $ cells were picked into 96-well
micro-liter-plates containing liquid LB-medium and Kanamycin (5mg/ml)
and incubated for another 24h. Subsequently 2ml of the cells were used
as template for amplification of the insert by PCR using the primers
\begin{description}
\item[Forward] M13F(GTAAAACGACGGCCAGT) and
\item[Reverse] M13R(GGCAGGAAACAGCTATGACC)
\end{description}
in a concentration of 10$\mu$M. The protocol for PCR cycling is shown
% in table \ref{tab:PCR}.

\begin{table}[h]
  \centering
  \begin{tabular}{lllll} 
    \textbf{Inital denaturation} &  $ 94\, ^{\circ}\mathrm{C} $ & 5min &  &\\ 
    \hline
    \textbf{Denaturation} &  $ 94\, ^{\circ}\mathrm{C} $ &30s& & \\ 
    \textbf{Annealing} &   $ 54\, ^{\circ}\mathrm{C} $ & 45s & 35 cycles &\\ 
    \textbf{Elongation} &   $ 72\, ^{\circ}\mathrm{C} $ & 2min &  &\\ 
    \hline
    \textbf{Filnal Elongation} &   $ 72\, ^{\circ}\mathrm{C} $ & 10min &\\ 
  \end{tabular}   
  \caption{PCR protocol for insert amplification}
  \label{tab:PCR}
\end{table}

Amplification products were controlled on gel and cleaned using SAP
(Shrimp Alkaline Phosphatase) and ExoI (Exonuclease I). Sequencing
reactions were performed using the BigDye-Terminator kit and
PCR-primers (forward or reverse) in a concentration of 3.5$\mu$M and
sequenced on an ABI 3730 DNA Analyzer (Applied Biosystems, Foster
City, California, USA).  For \textit{A. crassus} the following
libraries were prepared:
 
\begin{description}
\item{Ac\_197F:} Female from Taiwanese aquaculture
\item{Ac\_106F:} Female from Taiwanese aquaculture
\item{Ac\_M175:} Male from Taiwanese aquaculture
\item{Ac\_FM:} Female from Taiwanese aquaculture
\item{Ac\_EH1:} Same cDNA preparation as Ac\_FM, but sequenced by
  students in a practical
\end{description}

For \textit{Anguilla japonica} the following three libraries:
\begin{description}
\item{Aj\_Li1:} liver of an eel from aquaculture
\item{Aj\_Li2:} liver of an eel from aquaculture
\item{Aj\_Li3:} liver of an eel from aquaculture
\end{description}

\section{Pilot Sanger-sequencing}

The original sequencing-chromatographs ("trace-files") were renamed
according to the NERC environmental genomics scheme. "Ac" was used as
project-identifier for \textit{Anguillicoloides crassus}, "Aj" for
\textit{Anguilla japonica}. In \textit{Anguillicoloides} sequences
information on the sequencing primer (forward or reverse PCR primer
\textit{Anguilla japonica} sequences were all sequenced using the
forward PCR primer) was stored in the middle
"library"-field, resulting in names of the following form:

\begin{itemize}
\item \texttt{Ac\_[\textbackslash{}d|\textbackslash{}w]\{2,4\}(f|r)\_\textbackslash{}d\textbackslash{}d\textbackslash{}w\textbackslash{}d\textbackslash{}d}
\item  \texttt{Aj\_[\textbackslash{}d|\textbackslash{}w]\{2,4\}\_\textbackslash{}d\textbackslash{}d\textbackslash{}w\textbackslash{}d\textbackslash{}d}
\end{itemize}

The last field indicates the plate number (two digits), the row (one
letter) and the column (two digits) of the corresponding clone. For
first quality trimming trace2seq, a tool derived from trace2dbEST
(both part of PartiGene \cite{parkinson_partigene--constructing_2004})
was used, briefly it performs quality trimming using
phred\cite{ewing_base-calling_1998} and trimming of vector sequences
using cross-match\cite{PHRAP}. The adapters used by the MINT kit were
trimmed by supplying them in the vector-file used for trimming along
with the TOPO2PCR-vector. After processing with trace2seq additional
quality trimming was performed on the produced sequence-files using a
custom script. This trimming was intended to remove artificial
sequences produced when the sequencing reaction starts at the 3' end
of the transcript at the poly-A tail. These sequences typically
consist of numerous homo-polymer-runs throughout their length caused
by "slippage" of the reaction.
The basic perl regular expression used for this was:\\

\texttt{/(.*A\{5,\}|T\{5,\}|G\{5,\}|C\{5,\}.*)\{\$lengthfac,\}/g}\\

Where \texttt{\$lengthfac} was set to the length of the sequence
devided by 70 and rounded to the next integer. So only one
homo-polymer-run of more then 5 bases was allowed per 75 bases.

Sequences were screened for host contamination by a comparison of
\texttt{BLAST} searches against the version of nempep4 and a fish
protein database. Sequences producing better bit scores againt fish
proteins than nematode proteins were labeled as host-contamination.

Only the trace-files corresponding to the sequences still regarded as
good after this step were processed with trace2dbEST. Additionally to
the processing of traces already included in trace2seq sequences were
preliminary annotated using \texttt{BLAST} versus the NCBI-NR
non-redundant protein database and EST-submission-files were produced.

\section{454-pyro-sequencing}

\subsection*{cDNA preparation and sequencing}

RNA was extracted from individual adult male and female nematodes and
from a population of L2 larvae (Table 1). RNA was reverse transcribed
and amplified into cDNA using the MINT-cDNA synthesis kit (Evrogen,
Moscow, Russia).  For host contamination screening a liver-sample from
an uninfected \textit{An. japonica} was also processed. Emulsion PCR
was performed for each cDNA library according to the manufacturer’s
protocols (Roche/454 Life Sciences), and sequenced on a Roche 454
Genome Sequencer FLX. All samples were sequenced using the FLX
Titanium chemistry, except for the taiwanese female sample T2, which
was sequenced using FLX standard chemistry, to generate between 99,000
and 209,000 raw reads. For the L2 larval library, which had a larger
number of non-\textit{A. crassus}, non-\textit{Anguilla} reads, we
confirmed that these data were not laboratory contaminants by
screening Roche 454 data produced on the same run in independent
sequencing lanes.


\subsection*{Trimming, quality control and assembly}

Raw sequences were extracted in \texttt{fasta}-format (with the
corresponding qualities files) using \texttt{sffinfo} (Roche/454) and
screened for adapter sequences of the MINT-amplification-kit using
\texttt{cross-match} \cite{PHRAP} (with parameters \texttt{-minscore
  20 -minmatch 10}). \texttt{Seqclean} \cite{tgicl_pertea} was used to
identify and remove poly-A-tails, low quality, repetitive and short
(<100 base) sequences. All reads were compared to a set of screening
databases using \texttt{BLAST} (expect value cutoff E<1e-5, low
complexity filtering turned off: -F F). The databases used were (a) a
host sequence database comprising an assembly of the
\textit{An. japonica} Roche 454 data, a unpublished assembly of
\textit{An. anguilla} Sanger dideoxy sequenced expressed sequence tags
(made available to us by Gordon Cramb, University of St Andrews) and
transcripts from EeelBase \cite{pmid21080939} a publicly available
transcriptome database for the European eel; (b) a database of
ribosomal RNA (rRNA) sequences from eel species derived from our Roche
454 data and EMBL-Bank; and (c) a database of rRNA sequences
identified in our \textit{A. crassus} data by comparing the reads to
known nematode rRNAs from EMBL-Bank. This last database notably also
contained xenobiont rRNA sequences. Reads with matches to one of these
databases over more than 80\% of their length and with greater than
95\% identity were removed from the dataset. Screening and trimming
information was written back into sff-format using \texttt{sfffile}
(Roche 454). The filtered and trimmed data were assembled using the
combined assembly approach \cite{pmid20950480}: Two assemblies were
generated, one using \texttt{Newbler v2.6} \cite{pmid16056220} (with
parameters \texttt{-cdna -urt}), the other using \texttt{Mira v3.2.1}
\cite{miraEST} (with parameters
\texttt{--job=denovo,est,accurate,454}). The resulting two assemblies
were combined into one using \texttt{Cap3} \cite{Cap3_Huang} at
default settings and contigs were labeled by whether they derived from
both assemblies or one assembly only (for a detailed analysis of the
assembly categories see the supporting Methods file).

\subsection*{Post-assembly classification and taxonomic assignment of
  contigs}

After assembly contigs were assessed a second time for host and other
contamination by comparing them (using \texttt{BLAST}) to the three
databases defined above, and also to nembase4, a nematode
transcriptome database derived from whole genome sequencing and EST
assemblies \cite{parkinson_nembase:resource_2004, pmid21550347}. For
each contig, the highest-scoring match was recorded as long as it
spanned more than 50\% of the contig. We also compared the contigs to
the NCBI non-redundant nucleotide (NCBI-nt) and protein (NCBI-nr)
databases, recording the taxonomy of all best matches with expect
values better than 1e-05. TUGs with a best hit to non-Metazoans and to
Chordata within Metazoa were additionally excluded from further
analysis.

\subsection*{Protein prediction and annotation}

Protein translations were predicted from the contigs using
\texttt{prot4EST} (version 3.0b)
\cite{wasmuth_prot4est:_2004}. Proteins were predicted either by
joining single high scoring segment pairs (HSPs) from a \texttt{BLAST}
search of uniref100 \cite{pmid18836194}, or by \texttt{ESTscan}
\cite{estscan}, using as training data the \textit{Brugia malayi}
complete proteome back-translated using a codon usage table derived
from the \texttt{BLAST} HSPs, or, if the first two methods failed,
simply the longest ORF in the contig. For contigs where the protein
prediction required insertion or deletion of bases in the original
sequence, we also imputed an edited sequence for each affected
contig. Annotations with Gene Ontology (GO), Enzyme Commission (EC)
and Kyoto Encyclopedia of Genes and Genomes (KEGG) terms were inferred
for these proteins using \texttt{Annot8r} (version 1.1.1)
\cite{schmid_annot8r:_2008}, using the annotated sequences available
in uniref100 \cite{pmid18836194}. Up to 10 annotations based on a
\texttt{BLAST} similarity bitscore cut-off of 55 were obtained for
each annotation set. The complete \textit{B. malayi} proteome (as
present in uniref100) and the complete \textit{C. elegans} proteome
(as present in wormbase v.220) were also annotated in the same
way. \texttt{SignalP V4.0} \cite{pmid21959131} was used to predict
signal peptide cleavage sites and signal anchor signatures for the
\textit{A. crassus}-transcriptome and similarly again for the
proteomes of the tow model-worms.  Additionally \texttt{InterProScan}
\cite{pmid11590104} (command line utility \texttt{iprscan} (version
4.6) with options \texttt{-cli -format raw -iprlookup -seqtype p
  -goterms}) was used to obtain domain based annotations for the high
credibility assembly (highCA) derived contigs.

We recorded the presence of a lethal rnai-phenotype in the
\textit{C. elegans} ortholog of each TUG using the biomart-interface
\cite{pmid22083790} to wormbase v. 220 through the R-package
\texttt{biomaRt} \cite{pmid19617889}.

\subsection*{Single nucleotide polymorphism analysis}

We mapped the raw reads against the the complete set of contigs,
replacing imputed sequences for originals where relevant, using
\texttt{ssaha2} \cite{pmid11591649} (with parameters \texttt{-kmer 13
  -skip 3 -seeds 6 -score 100 -cmatch 10 -ckmer 6 -output sam -best
  1}). From the \texttt{ssaha2} output, pileup-files were produced
using \texttt{samtools} \cite{journals/bioinformatics/LiHWFRHMAD09},
discarding reads mapping to multiple regions. \texttt{VarScan}
\cite{pmid19542151} (\texttt{pileup2snp}) was used with default
parameters on pileup-files to output lists of single nucleotide
polymorphisms (SNPs) and their locations. For enrichment analysis of
GO-terms we used the R-package \texttt{GOstats} \cite{pmid17098774}.

Using \texttt{Samtools} \cite{journals/bioinformatics/LiHWFRHMAD09}
(\texttt{mpileup -u}) and \texttt{Vcftools} \cite{pmid21653522}
(\texttt{view -gcv}) we genotyped individual libraries for the list of
previously found overall SNPs. Genotype-calls were accepted at a
phred-scaled genotype quality threshold of 10. In addition to the
relative heterozygosity (number of homozygous sites/number of
heterozygous sites) we used the R package \texttt{Rhh}
\cite{pmid21565077} to calculate internal relatedness
\cite{pmid11571049}, homozygosity by loci \cite{pmid17107491} and
standardized heterozygosity \cite{coltman81j} from these data.

Using 1000 bootstrap replicates we confirmed the significance of
heterozygote-heterozygote correlation by analyzing the mean and 95\%
confidence intervals from 1000 bootstrap replicates estimated for all
measurements.

\subsection*{Gene-expression analysis}

Read-counts were obtained from the \texttt{bam}-files generated also
for genotyping using the R-package \texttt{Rsamtoools}
\cite{rsamtools}. TUGs with less than 48 reads over all libraries were
excluded from analysis, as diagnostic plot (not shown) indicated a lack
of statistical power for lower overall expression. We used the
R-package \texttt{DESeq} \cite{pmid20979621} to assess statistical
significance of differences in counts according to groups of
libraries.

Additionally we collapsed TUGs by their orthologous assignment in
\textit{C. elegans} and \textit{B. malayi}. We used the sums of counts
for these orthologous-groups to asses the influence of mapping to our
potentially fragmented reference. For both model-nematodes fold-change
and p-values were obtained the same way than for the contigs and
merged with these.


\section{Transcriptomic divergence in a common garden experiment}

\subsection{Experimental infeciton of eels} 

\textit{An. anguilla} were obtained from the Albe-Fishfarm in
Haren-Rütenbrock, Germany. \textit{An. japonica} were caught at the
glass-eel stage in the estuary of Kao-ping River, Taiwan by
professional fishermen and kept at a water temperature of
26$^{\circ}$C until they reached a size of $>$ 35 cm.

The absence of infecions with \textit{A. crassus} in both eel-species
was confirmed by dissection of 10 individuals of each species.

After an acclimatisation periode of 4 weeks (\textit{An. anguilla}) or
when they reached a size of $>$ 35cm (\textit{An. japonica} eels were
infected using a stomach tube as described in
\cite{boon1990effect}. During the infection periode water temperature
was held constant at 20$^{\circ}$C. Eels were kept in 160-liter tanks
in groups of 5-10 individuals and continuously provided with fresh,
oxygenated water and commercial fish pellets (Dan-Ex 2848, Dana Feed
A/S Ltd, Horsens, Denmark).

L2 larvae used for the infection were collected from the swimbladders
of wild yellow and silver eels from the River Rhine near Karlsruhe and
from Lake M\"uggelsee near Berlin in Germany. Taiwanese larvae were
obtained from aquacultre in Kao Ping and from 150km further north in
Tainan County in Taiwan. They were stored at 4$^{\circ}$C for no
longer than 2 weeks before copepods were infected. Mixed species
samples of uninfected copepods were collected from a small pond near
Karlsruhe, known to be free of eels. They were infected individually
in wells of micro-titer plates at an intensity of 10 L2-larvae per
copepod. One week after infection they were placed in bigger
tanks. Twice a week yeast was provided as food and at 21 dpi the L3
were harvested with a tissue potter using a modified procedure
developed by \cite{haenen_improved_1994}. 50 L3 were suspended in 100
$\mu$l RPMI-1640 medium (Quiagen, Hilden, Germany) and eels were
infected.

55-57 days post infection (dpi) eels were euthanized and dissected.
%% The length and width of the adults were measured to the nearest
%% 0.01 mm.
After determination of the sex of adult worms under a binocular
microscope (Semi 2000, Zeiss, Germany), they were immediately immersed
in RNAlater (Quiagen, Hilden, Germany).

\section{RNA exrtaction and preparation of sequencing libraries}

RNA was extracted from 12 individual female worms and for 12 pools of
male worms using the RNeasy-kit (Quiagen, Hilden, Germany) (see table
XXX).

The Paired-End TruSeqTM RNA sample preparation kit (illumina) was
followed to build cDNA libraries with insert sizes of roughly 270 bp
for paired-end sequencing: Poly‐T oligo‐attached magnetic beads were
used for purification of mRNA and to simultaniously fragment the
RNA. The RNA was then primed with random hexamer primers for cDNA
synthesis and reverse transcribed into first strand cDNA using reverse
transcriptase. The cDNA was cleaned from the 2nd strand reaction,
overhangs were repaired to form blunt ends, a single ``A'' nucleotide
was added at the 3' end and paired end sequencing adapters ware
ligated with a complementary ``T''-overhang. At this step multiple
different indexed paired-end adapters were used to enable multiplexing
of the 24 different sequening libraries, in 3 pools of 8 samples each,
on one lane of the instrument.  Molecules having adapter sequences
were enriched in the mix using PCR and the libraries were controled
for quality and quantity on the BioAnalyzer (Agilent). Clusters were
generated by bridge amplification. The resulting clusters were
sequenced on the Genome Analyzer IIX in combination with the
paired-end module. The first read was sequenced using using the first
primer Rd1 SP. The original template strand was then used to
regenerate the complementary strand, the original strand was removed
and complementary strand acted as a template for the second read
sequenced primed by the second sequencing primer Rd1 SP.


\section{Mapping and read-counts}


\section{Statistical analysis}


\section{Collapsing of orthologous for two model-species}


%%% Local Variables: ***
%%% mode:latex ***
%%% TeX-master: "../thesis.tex"  ***
%%% tex-main-file: "../thesis.tex" ***
%%% End: ***
 






