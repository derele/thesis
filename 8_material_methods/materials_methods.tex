
% this file is called up by thesis.tex
% content in this file will be fed into the main document

\chapter{Materials \& methods} % top level followed by section, subsection


% ----------------------- paths to graphics ------------------------

% change according to folder and file names
\ifpdf
    \graphicspath{{8/figures/PNG/}{8/figures/PDF/}{8/figures/}}
\else
    \graphicspath{{8/figures/EPS/}{8/figures/}}
\fi

% ----------------------- contents from here ------------------------


\section{Sampling of worms from  wild eels}

\subsection{Sampling in Taiwan}

Cultured eels were acquired from an aquaculture directly adjacent to
Kaoping river (22.6418N; 120.4440E) 15km stream upwards from it's
estuary, on the 29th of April 2008. On the same day wild eels were
picked up at Tunkang Biotechnology Research Centre Fisheries Research
institute in Tunkang, Pintung, Taiwan, where they had been sheltered
since the time of purchase during the 2nd two weeks of April 2008 from
a fisherman, fishing in the estuary of Kao-Ping river (22.5074N;
120.4220E). All eels were transported to the Institute of Fisheries
Science at the National Taiwan University in Taipei in aerated plastic
bags, where they were sheltered until dissection.

Dissection of eels was carried out during May 2008. Eels were
decapitated, length (to the nearest 1.0mm) and weight (to the nearest
0.1g) were measured, and sex was determined by visual inspection of
the gonads. The swimbladder was opened, adult worms were removed from
the lumen with a forceps, their sex was determined, and they were
counted. All adult \textit{A. crassus} were preserved in
RNAlater(Quiagen, Hilden, Germany) in individual plstic tubes.

\subsection{Sampling of European worms}

Worms from the European eel were sampled in Sniardwy Lake, Poland
(53.751959N ,21.730957E) by Urszula Weclawski and from the
Linkenheimer Altrhein, Germany (49.0262N; 8.310556E), following a
procedure similar to the one described above for worms from Taiwan.

\section{RNA-extraction and cDNA synthesis for Sanger- and
  454-sequencing}

Total RNA was extracted from single, whole worms using the RNeasy kit
(Quiagen, Hilden, Germany), following the manufacturers
protocol. Alternatively parts of the liver of the host species
\textit{Anguilla japonica}, which also had been preserved in RNAlater
were used for RNA extraction, following the same protocol.

The Evrogen MINT cDNA synthesis kit (Evrogen, Moscow, Russia ) was
then used to amplify mRNA transcripts according to the manufacturers
protocol. It uses an adapter sequence at 3' the end of a a poly
dT-primer for first strand synthesis and adds a second adapter
complementary to the bases at the 5' end of the transcripts by
terminal transferase activity and template switching. Using these
adapters it is possible to specifically amplify mRNA enriched for
full-length transcripts. 

\section{Cloning and Sanger-sequencing}

The obtained cDNA preparations were undirectionally cloned into
TOPO2PCR-vectors (Invitrogen, Carlsbad, USA) and TOP10 chemically
competent cells (Invitrogen, Carlsbad, USA) were transformed with this
construct. The cells were plated on LB-medium-agarose containing
Kanamycin (5mg/ml), xGal
(5-bromo-4-chloro-3-indolyl-$\beta$-D-galactopyranoside) and IPTG
(Isopropyl-$\beta$-D-1-thiogalactopyranosid). After 24h of incubation
at $36\,^{\circ}\mathrm{C} $ cells were picked into 96-well
micro-liter-plates containing liquid LB-medium and Kanamycin (5mg/ml)
and incubated for another 24h. Subsequently 2ml of the cells were used
as template for amplification of the insert by PCR using the primers
\begin{description}
\item[Forward] M13F(GTAAAACGACGGCCAGT) and
\item[Reverse] M13R(GGCAGGAAACAGCTATGACC)
\end{description}
in a concentration of 10$\mu$M. The protocol for PCR cycling is shown
% in table \ref{tab:PCR}.

\begin{table}[h]
  \centering
  \begin{tabular}{lllll} 
    \textbf{Inital denaturation} &  $ 94\, ^{\circ}\mathrm{C} $ & 5min &  &\\ 
    \hline
    \textbf{Denaturation} &  $ 94\, ^{\circ}\mathrm{C} $ &30s& & \\ 
    \textbf{Annealing} &   $ 54\, ^{\circ}\mathrm{C} $ & 45s & 35 cycles &\\ 
    \textbf{Elongation} &   $ 72\, ^{\circ}\mathrm{C} $ & 2min &  &\\ 
    \hline
    \textbf{Filnal Elongation} &   $ 72\, ^{\circ}\mathrm{C} $ & 10min &\\ 
  \end{tabular}   
  \caption{PCR protocol for insert amplification}
  \label{tab:PCR}
\end{table}

Amplification products were controlled on gel and cleaned using SAP
(Shrimp Alkaline Phosphatase) and ExoI (Exonuclease I). Sequencing
reactions were performed using the BigDye-Terminator kit and
PCR-primers (forward or reverse) in a concentration of 3.5$\mu$M and
sequenced on an ABI 3730 DNA Analyzer (Applied Biosystems, Foster
City, California, USA).  For \textit{A. crassus} the following
libraries were prepared:
 
\begin{description}
\item{Ac\_197F:} Female from Taiwanese aquaculture
\item{Ac\_106F:} Female from Taiwanese aquaculture
\item{Ac\_M175:} Male from Taiwanese aquaculture
\item{Ac\_FM:} Female from Taiwanese aquaculture
\item{Ac\_EH1:} Same cDNA preparation as Ac\_FM, but sequenced by
  students in a practical
\end{description}

For \textit{Anguilla japonica} the following three libraries:
\begin{description}
\item{Aj\_Li1:} liver of an eel from aquaculture
\item{Aj\_Li2:} liver of an eel from aquaculture
\item{Aj\_Li3:} liver of an eel from aquaculture
\end{description}

\section{Bioinformatic analysis of pilot Sanger-sequencing}

The original sequencing-chromatographs ("trace-files") were renamed
according to the NERC environmental genomics scheme. "Ac" was used as
project-identifier for \textit{Anguillicoloides crassus}, "Aj" for
\textit{Anguilla japonica}. In \textit{Anguillicoloides} sequences
information on the sequencing primer (forward or reverse PCR primer
\textit{Anguilla japonica} sequences were all sequenced using the
forward PCR primer) was stored in the middle
"library"-field, resulting in names of the following form:\\

\texttt{Ac\_[\textbackslash{}d|\textbackslash{}w]\{2,4\}(f|r)\_\textbackslash{}d\textbackslash{}d\textbackslash{}w\textbackslash{}d\textbackslash{}d}\\
\texttt{Aj\_[\textbackslash{}d|\textbackslash{}w]\{2,4\}\_\textbackslash{}d\textbackslash{}d\textbackslash{}w\textbackslash{}d\textbackslash{}d}\\

The last field indicates the plate number (two digits), the row (one
letter) and the column (two digits) of the corresponding clone. For
first quality trimming trace2seq, a tool derived from trace2dbEST
(both part of PartiGene \cite{parkinson_partigene--constructing_2004})
was used, briefly it performs quality trimming using
phred\cite{ewing_base-calling_1998} and trimming of vector sequences
using cross-match\cite{PHRAP}. The adapters used by the MINT kit were
trimmed by supplying them in the vector-file used for trimming along
with the TOPO2PCR-vector. After processing with trace2seq additional
quality trimming was performed on the produced sequence-files using a
custom script. This trimming was intended to remove artificial
sequences produced when the sequencing reaction starts at the 3' end
of the transcript at the poly-A tail. These sequences typically
consist of numerous homo-polymer-runs throughout their length caused
by "slippage" of the reaction.
The basic perl regular expression used for this was:\\

\texttt{/(.*A\{5,\}|T\{5,\}|G\{5,\}|C\{5,\}.*)\{\$lengthfac,\}/g}\\

Where \texttt{\$lengthfac} was set to the length of the sequence
devided by 70 and rounded to the next integer. So only one
homo-polymer-run of more then 5 bases was allowed per 105 bases.

Sequences were screened for host contamination by a comparison of
\texttt{BLAST} searches against the version of nempep4 and a fish
protein database. Sequences producing better bit scores againt fish
proteins than nematode proteins were labeled as host-contamination.

Only the trace-files corresponding to the sequences still regarded as
good after this step were processed with trace2dbEST. Additionally to
the processing of traces already included in trace2seq sequences were
preliminary annotated using \texttt{BLAST} versus the NCBI-NR
non-redundant protein database and EST-submission-files were produced.

\section{Bioinformatic analysis of 454-pyro-sequencing}

\subsection*{Trimming, quality control and assembly}

Raw sequences were extracted in fasta format (with the corresponding
qualities files) using sffinfo (Roche/454) and screened for adapter
sequences of the MINT-amplification-kit using cross-match \cite{PHRAP}
(with parameters -minscore 20 and -minmatch 10). Seqclean
\cite{tgicl_pertea} was used to screen poly-A-tails, low quality,
repetitive and short (<100 bases) sequences. In addition all reads
were \texttt{blast}ed (1e-5 -F F) against the following databases:

\begin{itemize}
\item a combined eel-mRNA database consisting of an assembly of
  sequences from the liver of the Japanese eel sequenced for this
  purpose (as described above), a sequence assembly of unpublished
  (sanger-) ESTs (made available to us by Gordon Cramb; University of
  St Andrews) and from EeelBase \cite{pmid21080939} a publically
  availble transcriptome database for the European eel.
\item a eel-rRNA database from a rRNA screening of the above and
  assembly together with publically available rRNA-sequences.
\item an \textit{A.crassus} rRNA-database from screening of our
  dataset against nematode-rRNA, and assembly of these rRNA
  reads. This database notably also contained xenobiotic rRNA
  sequences.
\end{itemize}

Reads mapping to one of the databases with more than 80\% of their
length and 95\% identity were removed from the dataset. Screenig and
trimming information was written back into sff-format using
\texttt{sfffile} (Roch/454).

We used an approach proposed by Kumar and Blaxter\cite{pmid20950480},
combining assemblies from the \texttt{mira} \cite{miraEST} and
\texttt{newbler} \cite{pmid16056220}. Briefly the two assemblies are
combined into one using Cap3\cite{Cap3_Huang} and only contigs
supported by both assemblers are regarded good quaility. For further
details see the supplementary methods.

\subsection*{Post assembly classification and taxonomic assesment}

After assembly contigs were assesed a second time for
host-contamination and other xenobiotics:

The contigs were \texttt{blast}ed (with a cut-off 1e-5) against the
same databases used prior to assembly (Eel-mRNA, Eel-rRNA,
\textit{A.crassus}-rRNA and additionally against the nucleotide
version of nempep4 \cite{parkinson_nembase:resource_2004,
  pmid21550347}, determining the best hit across databases. These best
hits across databases were screened and only such hits involving more
then 50\% of the 

Additionally \texttt{blast} (\texttt{blastn} e-value cut-off 1e-5)
against NCBI-nt and (\texttt{blastx} e-value cut-off 1e-5) against
NCBI-nt was used to determine taxon-membership of the top hit at the
family, phylum and kingdom rank.

\subsection*{Protein prediction and annotation}

Proteins were predicted using the \texttt{Prot4EST} (version 3.0b)
\cite{wasmuth_prot4est:_2004}: First \texttt{blast} searches against a
rRNA-database, a mitochondrial database and against
uniref100\cite{pmid18836194} were preformed. Then results were used to
predict proteins directly (joining single high scoring pairs, and
thereby intorducing gaps and ambiguous bases if needed). Secondly
using the codon-usage from \texttt{blast}-predictions a simulated
transcriptome was generated, reverse translating the
\textit{B. malayi} proteom, as training-data-set for
\texttt{ESTscan}’s\cite{estscan} hidden Markov models.  If both
\texttt{blast}-based prediction and \texttt{ESTscan} failed, simply
the longest ORF is inferred.

\texttt{Blast}-based annotations were inferred using Annot8r (version
1.1.1) \cite{schmid_annot8r:_2008}: Searches were performed against
all sequences in uniref100 being annotated with GO-terms, EC-numbers
and KEGG-parthways. Up to 10 (possibly contradictory) annotations
based on a bitscore cut-off of 55 were obtained for each annotated
database.

% Additionally domain-based annotation was
% obtained using InterProScan (iprscan command-line tool, version 4.6)
% \cite{pmid11590104}.

\texttt{SignalP V3.0} \cite{pmid17446895} was used to predict signal
peptide cleavage sites and signal anchor signatures.

\subsection*{SNP analysis}

As protein-prediction inferres gaps (e.g from sequencing errors) to
predict the most likely protein, not only start- and end-coordinates
of open reading frames (ORFs) had to be extracted from the output of
\texttt{Prot4EST}. We did this in a custom \texttt{perl}-script using
a \texttt{blast}-search with the nucleotide equivalent of the protein
as query and the raw sequence as subject. We obtain the
hit-coordinates as ORF-coordinates and imputed the
\texttt{blast}-query as corrected ORF-sequencs.

We mapped the raw reads against the the complete unigene set, with the
imputed sequences for those contigs with proteins predicted, using
ssaha2 (with parameters -kmer 13 -skip 3 -seeds 6 -score 100 -cmatch
10 -ckmer 6 -output sam -best 1).

\texttt{pileup}-files were produced using \texttt{samtools}
\cite{journals/bioinformatics/LiHWFRHMAD09}, discarding sequences
mapping to multiple regions with the best hit. VarScan
\cite{pmid19542151} (pileup2snp) was used with default parameters on
\texttt{pileup}-files. This output was further screened as described
in the results part of the manuscript.

% \subsection*{Respiration analysis}

% We extracted contigs annotated with ``GO:0009060 aerobic respiration''
% and ``GO:0009061 anaerobic respiration''

\subsection*{Gene-expression analysis}

For NlaIII-tag-sequencing total RNA was prepared as described above
from a worm from the Polish sampling site. A sequence-tag libray was
created following the protocol supplied by Illmina for this method.
Briefly after synthesis of cDNA on oligo(dt)-beads, this cDNA is
digested with the enzyme NlaIII (restriction site ``CATG''). After
ligation of an adaptor containing its restriction site the enzyme MmeI
cuts 17 bases downstream of its binding site generating a sequence tag
of in total 21 bases.

For 454 reads, read counts were obtained from the mapping to imputed
sequence descrebed above. Tag-sequences were mapped using
\texttt{BWA}\cite{pmid19451168}. And read counts extracted using
\texttt{Samtools}.

The R-package DESeq\cite{pmid20979621} was used to normalize for
library-size and analyse statistical significance of differential
expression.





%%% Local Variables: ***
%%% mode:latex ***
%%% TeX-master: "../thesis.tex"  ***
%%% tex-main-file: "../thesis.tex" ***
%%% End: ***



 






