% this file is called up by thesis.tex
% content in this file will be fed into the main document

\chapter{Discussion} % top level followed by section, subsection


% ----------------------- paths to graphics ------------------------

% change according to folder and file names
\ifpdf
    \graphicspath{{7/figures/PNG/}{7/figures/PDF/}{7/figures/}}
\else
    \graphicspath{{7/figures/EPS/}{7/figures/}}
\fi


% ----------------------- contents from here ------------------------
\section{Overview}


\section{Sanger-method pilot-sequencing}
\label{sec:sanger-pil}

In was not achieved to reproducibly alleviate the rRNA-levels in
libraries prepared for sequencing. This has probably been due to the
fact that extraction of total-RNA from worms filled with host blood
resulted in low amounts of starting material, and reaction conditions
did not allow specific amplification of mRNA from a rRNA
background. As the same problems existed in preparation of liver
tissue of the host species it seems likely that the blood of eels
contains substances limiting the success of specific amplification
protocols. In fact it is known that compounds like hemoglobin can
inhibit PCR reactions \cite{pmid9327537} and reverse transcription
\cite{pmid16109794}.

Nevertheless the stringent quality trimming and processing of raw
reads, as summarized in \ref{pilot-seq}, made the remaining ESTs a
valuable resource for comparison with future 454-sequencing-data.

In fact all sequenced ESTs, for which host-orgin was inferrred were
later found also in pyrosequencing: The observation hemoglobin and
ferritin subunits form \textit{An. anguilla} are expected, as fish
erythrocytes contain a nucleus and still transcribe genes actively
\cite{pmid20614118} and these are typical proteins for the functioning
of red blood cells. The overservation of fish cyclin G1 (in Sanger and
pyrosequencing) and cohesin (in Sanger sequencing) is remarkable as
fish erythrocytes are thought to exhibit low rates of mitosis
\cite{pmid7506383}. Other obersvation of host-sequences like
e.g. Leukocyte cell-derived chemotaxin 2 or natural killer
cell-enhancing factor (NKEF)-B protein in pyrosequencing make an
analysis of this fish-derived off-target data (form all sequencing
technologies) a very promising, it is howerver beyond the scope of the
present thesis.

\section{454-pyrosequencing}
\label{sec:454-pyr}

We have generated a de novo transcriptome for \textit{A. crassus} an
important invasive parasite that threatens wild stocks of the European
eel \textit{An. anguilla}. These data enable a broad spectrum of
molecular research on this ecologically and economically important
parasite. As \textit{A. crassus} lives in close association with its
host, we have used exhaustive filtering to attempt to remove all
host-derived, and host-associated organism-derived contamination from
the data. To do this we have also generated a transcriptome dataset
from the definitive host \textit{An. japonica}. The non-nematode,
non-eel data identified, particularly in the L2 sample, showed highest
identity to flagellate protists, which may have been parasitising the
eel (or the nematode). Encapsulated objects observed in eel swim
bladder walls \cite{heitlinger_massive_2009} could be due solely to
immune attrition of \textit{A. crassus} larvae or to other
coinfections.

A second examination of sequence origin was performed after assembly,
employing higher stringency cutoffs.  Similar taxonomic screening was
used in a garter snake transcriptome project \cite{pmid21138572}, and
an analysis of lake sturgeon tested and rejected hypotheses of
horizontal gene-transfer when xenobiont sequences were identified
\cite{pmid20386959}. A custom pipeline for transcriptome assembly from
pyrosequencing reads \cite{pmid20034392} proposed the use of EST3
\cite{pmid17218127} to infer sequence origin based simply on
nucleotide frequency. We were not able to use this approach
successfully, probably due to the fact that xenobiont sequences in our
data set derive from multiple sources with different GC content and
codon usage.

Compared to other NGS transcriptome sequencing projects
\cite{pmid20478048}, the combined assembly approach (see
\ref{sec:over-eval}) generated a smaller number of contigs that had
lower redundancy and higher completeness. Projects using the mira
assembler often report substantially greater numbers of contigs for
datasets of similar size (see e.g. \cite{pmid21364769}), comparable to
the mira sub-assembly in our approach. The use of oligo(dT) to capture
mRNAs probably explains the bias towards 3' end completeness and a
relative lack of true initiation codons in our protein
prediction. This bias is near-ubiquitous in deep transcriptome
sequencing projects (e.g. \cite{pmid20331785}).

We were able to obtain high-quality annotations for a large set of
TUGs: For 40\% of the complete assembly and 60\% of our highCA
assembly \texttt{BLAST}-based annotations could be obtained. 45\% of
the contigs in the highCA assembly were additionally decorated with
domain-based annotations through \texttt{InterProScan}
\cite{pmid11590104}.

Comparison with complete protein sequence from the genomes of
\textit{B. malayi} and \textit{C. elegans} showed a remarkable degree
of agreement regarding the occurrence of terms in the two parasitic
worms. This agreement was higher than with the free living nematode
\textit{C. elegans} and even the two genome-sequencing-derived
proteomes showed less agreement with each other than the filarial
parasite with our dataset. This implies that our transcriptome is
truly a representative partial genome
\cite{parkinson_partigene--constructing_2004} of a parasitic nematode.

Analysis of conservation identified more sequence novel in Nematode
than in the eukaryote kingdom or in clade III this is in agreement
with prevalence of genic novelty in the Nematoda
\cite{wasmuth_extent_2008}. Furthermore the basal position of
\textit{A. crassus} in clade III could be leading to most novelty in
the clade not being shared with \textit{A. crassus}.

TUGs predicted to be novel in the phylum Nematoda and novel to
\textit{A. crassus} contained the highest proportion of
signal-positives. This confirms observations made in a study on
\textit{Nippostrongylus brasiliensis} \cite{harcus_signal_2004}, where
signal positives were reported as less conserved. Interestingly
enrichment of signal sequence bearing TUGs in our dataset was
constrained to sequences novel in nematodes and \textit{A. crassus}
(i.e. not to the level of clade III). This may be explained, whit two
different hypotheses involving the basal position of
\textit{A. crassus}: First the signal positives shared with all
nematodes could be conserved molecules not excreted by parasites. A
different class of secreted/excreted molecules with prominent role in
host parasite interactions would not have arisen early in the
evolution of parasitism in clade III - or be too fast-evolving - and
thus be detected as specific to deeper sub-clades (i.e. to
\textit{A. crassus} in our dataset). A second explanation would be,
that orthologs of excreted parasite-specific genes could be among
those shared with other nematodes and the fewer shared with clade III
implying a predisposition to parasitism outside of the Spirurina or
even the convergent evolution of secreted molecules in other parasitic
nematodes. However analysis of dn/ds (see below) across conservation
categories favours the first hypothesis, as it identifies a higher
amount of positive selection in TUGs novel to clade III and
\textit{A. crassus} than to nematodes.

We generated transcriptome data from multiple \textit{A. crassus} of
Taiwanese and European origin, and identified SNPs both within and
between populations. Screening of SNPs in or adjacent to homopolymer
regions improved overall measurements of SNP quality. The ratio of
transitions to transversions (ti/tv) increased. Such an increase is
explained by the removal of ``noise'' associated with common homopolymer
errors \cite{pmid21685085}. The value of 1.93 (1.25 outside, 2.41
inside ORFs) is in good agreement with the overall ti/tv of humans
(2.16 \cite{pmid21169219}) or \textit{Drosophila} (2.07
\cite{pmid21143862}). The ratio of non-synonymous SNPs per
non-synonymous site to synonymous SNPs per synonymous site (dn/ds)
decreased with removal of SNPs adjacent to homopolymer regions from
0.42 to 0.231 after full screening. The most plausible explanation is
the removal of error, as unbiased error would lead to a dn/ds of
1. While dn/ds is not unproblematic to interpret within populations
\cite{pmid19081788}, the assumption of negative (purifying) selection
on most protein-coding genes makes lower mean values seem more
plausible. We used a threshold value for the minority allele of 7\%
for exclusion of SNPs, based on an estimate that approximately 10
haploid equivalents were sampled (5 individual worms plus an
negligible contribution from L2 larvae in the L2 library and within
the female adult worms). The benefit of this screening was mainly a
reduction of non-synonymous SNPs in high coverage contigs, and a
removal of the dependence of dn/ds on coverage. Working with an
estimate of dn/ds independent of coverage, efforts to control for
sampling biased by depth (i.e. coverage; see \cite{pmid18590545} and
\cite{pmid20478048}) could be avoided.

Also in comparison with published intra-species values of dn/ds our
final estimate of seems plausible: in transcripts from the female
reproductive tract of \textit{Drosophila} dn/ds was 0.15
\cite{pmid15579698} and 0.21 in the male reproductive tract
\cite{pmid11404480} (although for ESTs specific to the male accessory
gland were shown to have a higher dn/ds of 0.47). A pyrosequencing
study in the parasitic nematode \textit{Ancylostoma canium}
\cite{pmid20470405} reported dn/ds of 0.3.

When the whole of coding sequences are studied, of which only a small
subset of sites can be under diversifying selection, dn/ds of ~0.5 has
been suggested as threshold for assuming diversifying selection
\cite{pmid15579698} instead of the classical threshold of 1
\cite{pmid6449605}. The use of this threshold for positive selection
led to the identification of over-represented of GO-term highlighting
very interesting transcripts:

13 peptidases under positive selection (from 43 with a dn/ds obtained)
meant an enrichment in the category. All 13 have different orthologs
in \textit{B. malayi} and \textit{C. elgans} and are conserved across
kingdoms. Despite their conservation peptidases are thought to have
acquired new and prominent roles in host-parasite interaction compared
to free living organisms: In \textit{A. crassus} a trypsin-like
proteinase has been identified thought to be utilised by the
tissue-dwelling L3 stage to penetrate host tissue and an aspartyl
proteinase thought to be a digestive enzyme in adults
\cite{polzer_identification_1993}. The 13 proteinases under positive
selection could be the targets of the adaptive immunity developed
against \textit{A. crassus} \cite{knopf_migratory_2008,
  knopf_vaccination_2008}, which is often only elicited against
subtypes of larvae \cite{molnar_caps}.

The under-representation of ribosomal proteins (term ``structural
constituent of ribosome'') in positive selected contigs is in good
agreement with the notion that ribosomal proteins are extremely
conserved across kingdoms \cite{pmid9664699} and should be under under
strong negative selection.

Genotyping of individual worms identified a set of 199
SNPs with highest credibility and a high information content for
population-genetic studies. Levels of genome-wide heterozygosity found
for the 5 adult worms examined in our study are in agreement with
microsattelite data \cite{wielgoss_population_2008} showing reduced
heterozygosity in European populations of \textit{A. crassus}.

We were able to use the DESeq \cite{pmid20979621} to report
transcripts significantly differing in expression between male and
female worms. This was possible for male worms, despite the fact that
no replicated samples were obtained. However only over-expression in
the non-repeated samples could be detected, as obviously lack of
expression in one sample can't statistically validate
under-expression. Genes over-expressed in male \textit{A. crassus}
comprise major sperm proteins well known for their high expression in
nematode sperm \cite{pmid15275275}.

We developed a method to assess the possible influence of
fragmentation of our reference-transcriptome on mapping. Using the
annotation with orthologous sequences of \textit{C. elegans} and
\textit{B. malayi} it is possible to highlight problems: For example a
contig annotated as ``Phosphoenolpyruvate carboxykinase'' showed
significant over-expression in the male. The 8 other contigs with this
annotation however, were nearly identical for the middle region of
their predicted proteins, had a high amount of homopolymeric regions in
their nucleotide sequence and attracted the missing reads mapping from
the other libraries. Our orthologous-collapsing method for read-counts
following manual inspection, could thus exclude this contig as a
false-positive.

In the comparison of European with Asian libraries most
differentially expressed contigs could be debunked as false positives
with this method. From the 6 remaining contigs one was annotated with
only one ortholog (`Contig200''), 3 were lacking any annotation
(``Contig4'', ``Contig5311'' and ``Contig40''). The only contig with
differential expression fully validated by our method (``Contig6355'')
was annotated as (``Trypsin family protein''). The identification of
another proteinase (it is not among those under positive selection) is
highlighting the interesting nature of these molecules in
\textit{A. crassus}.

Analysis of the expression data made clear what data is required to
analyse the differences in European vs. Asian nematodes: As variance
is high in outbred individuals from natural populations both
replication and depth of analysis have to be high. Furthermore it is
important to disentangle the influence of the host and the nematode
population e.g. in a co-inoculation experiment. The last approach
would also allow the analysis of transcriptomes roughly synchronised
for the time of development.

The \textit{A. crassus} transcriptome provides a basis of molecular
research on this important species. It further provides insight in
the evolution of parasitism complementing the catalogue of available
transcriptomic data with a member of the Spirurina phylogenetically
distant to so far sequenced parasites in this clade.


\section{Experimental infections}
\label{sec:exp-inf}

Whith some reservations discussed below our observation of higher
recovery of adult worms from locally matching
\textit{A. crassus}-\textit{Anguilla spp.} host-parastite combinations
allow the inferrence of local adaptation of different worm populations
to host-species.

It has to be emphasized that the observations made in common-garden
experiments first and foremost have to be interpreted as
phenotypes. An ideally suited phenotype to infer local adaptation
would be one with obviouse direct fitness-consequences, a so called
fitness-component. Such a fitness-component would ideally be a
measurement on a single individual and individual life-time
reporductive success would be such an ideal measurement. However the
techniques to measure this individual life-time reporductive success
have not been established in \textit{A. crassus} and it would be very
difficult to do so.

The recovery of certain developmental stages of worms is not an
optimal fitness-component. It is a composite measurement the speed of
development from previouse lifecycle stages (or speed of migration
towards the swimbladder) and of survival. While suvival is surely an
important component of the fitness, it is not so clear whether fast
development and/or migration to the swimbladder are. It is immaginable
that under certain requirements slower development could lead to
higher fitness, if it would e.g. allow to develop without attracting
the attention of the immune system.

Our results regarding recovery are also not in complete agreement with
previouse findings by Weclawski et al. (unpublished; see
\ref{div-ac}). Similar to our sutdy they found a higher recovery of
the European population of worms in the European eel but did not find
the complementary result of lower recovery in the Japanese eel for
this diverged population. They recorded recovery at slightly different
timepoints after infection (25, 50 and 100 dpi).

A slight problem with recovery in the experiments is that this value
is a mean measurement over many individuals.

While mRNA abundance (or gene-exprssion in its synonymous meaning) is
often closely linked to a genetics basis, the observed value of
expression in the expreient constituts a molecular phenotype. Partly
this phenotype has a genetic foundation partly it is ifluenced by
environmental factors (such as host genotype).

Nevertheless

One of the dangers of genomic data and 

Additionally, whereas functional effects are often caused by selec-
tion, functional differences alone do not demonstrate the past or
present action of selection.

CHANGE
Almost 30 years ago, Gould and Lewontin76 launched a crusade against
the adaptive paradigm that functional differences must be adaptive,
that is, caused by natural selection. Although their arguments were
controversial at the time, they have been highly influential on the
community of evolution- ary biologists. As a new generation of
biologists with a background in genomics, molecular biology or bioin-
formatics has taken leadership in the field of genomic evolutionary
biology, the old lessons from Gould and Lewontin seem to have been
forgotten. It is a desir- able addition to a story of selection to
identify possible functional reasons why selection might be acting,
but it will never be a method for identifying or verifying selection.
CHANGE

adaptationist \cite{gould_spandrels_1979} Nielsen genomics
\cite{pmid19744124}


The orthologous screening method develped in the analysis of
differential expression for the possibly fragmented transcriptome
assembly uses stringent conditions for the detection
significance. Demanding 

---

No surprise was the abundance of differential expression between male
and female worms. A lage number of genes are known to be even
sex-specific, regulating ovulation and spermatogenesis throughout the
metazoa \cite{pmid16825664} and especially in nematodes
\cite{pmid15987632}. On top of these sex-specific genes there is large
number of genes differently expressed due to differences in
metablolism between males and females.

Sex specific genes, especially male specific genes show elevated rates
of sequence evolution \cite{pmid11404480,pmid15371532}.

A study on divergence between the transcriptome of \textit{Drosophila}
species show that interspecific expression divergence is sex dependent
\cite{pmid19720861} and inferred the action of sex-dependent natural
selection during species divergence.

in many species male-specific reproductive traits evolve faster than
other traits (Eberhard 1985; Civetta and Singh 1998a).

Genes differentially expressed between males and females can have
generall highly modifieable expression level. 

---

A surprise was the low number of genes detected as differentially
between the two host-species. The genetic differences in those host
species leading to a different immune response have a big influence on
other phenotypes of worms.

Genes identified to differ between populatins can be rather
categorized as important in general metabolic processes instead of
specific host-parasite intereaction.



Interpro lists 164 entries for \textit{C. elegans} ``Nematode cuticle
collagen, N-terminal''
(\href{http://www.ebi.ac.uk/interpro/ISpy?ipr=IPR002486&tax=6239}{IPR002486})
and 51 for \textit{B. malayi}
(\href{http://www.ebi.ac.uk/interpro/ISpy?ipr=IPR002486&tax=6279}{IPR002486}).


Aerobic respiration is potential source for oxidative stress providing
a steady source of reactive oxigen species (ROS) as electrons are
leaking from the respiratory chain as superoxide anions.

It is well established that such ROS production is especially harmful
to blood-feeding parasites, as free inorganic iron as well as heme are
generating additional ROS.

An increase in glycogenic enzymes was observed in \textit{Aedes}


%%% Local Variables: ***
%%% mode:latex ***
%%% TeX-master: "../thesis.tex"  ***
%%% tex-main-file: "../thesis.tex" ***
%%% End: ***
