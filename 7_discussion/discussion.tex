% this file is called up by thesis.tex
% content in this file will be fed into the main document

\chapter{Discussion} % top level followed by section, subsection


% ----------------------- paths to graphics ------------------------

% change according to folder and file names
\ifpdf
    \graphicspath{{7_discussion/figures/PNG/}{7_discussion/figures/PDF/}{7_discussion/figures/}}
\else
    \graphicspath{{7_discussion/figures/EPS/}{7_discussion/figures/}}
\fi


% ----------------------- contents from here ------------------------

\section{Pilot-sequencing}
\label{sec:sanger-pil}

In was not achieved to alleviate the rRNA-levels in libraries prepared
for sequencing. This has probably been due to the fact that extraction
of total-RNA from worms filled with host blood resulted in low amounts
of starting material, and reaction conditions did not allow specific
amplification of mRNA from a rRNA background. As the same problems
existed in preparation of liver tissue of the host species, it seems
likely that the blood of eels contains substances limiting the success
of specific amplification protocols. In fact it is known that
compounds like haemoglobin can inhibit PCR reactions
\cite{pmid9327537} and reverse transcription \cite{pmid16109794}.

Nevertheless the stringent quality trimming and processing of raw
reads, as summarised in \ref{pilot-seq}, made the remaining ESTs a
valuable resource for comparison with future 454-sequencing-data.

In fact all sequenced ESTs, for which host-origin was inferred were
later found also in pyrosequencing: The observation haemoglobin and
ferritin subunits from \textit{An. anguilla} are expected, as fish
erythrocytes contain a nucleus and still transcribe genes actively
\cite{pmid20614118}. These are typical proteins for the functioning of
red blood cells. The observation of fish cyclin G1 (in Sanger- and
pyrosequencing) and cohesin (in Sanger-sequencing) is remarkable as
fish erythrocytes are thought to exhibit low rates of mitosis
\cite{pmid7506383}. Other observations of host-sequences like
e.g. Leukocyte cell-derived chemotaxin 2 or natural killer
cell-enhancing factor (NKEF)-B protein in pyrosequencing make an
analysis of this fish-derived off-target data (from all sequencing
technologies) very promising, it is however beyond the scope of the
present thesis.

\section{Pyrosequencing}
\label{sec:454-pyr}

I have generated a \textit{de novo} transcriptome for \textit{A. crassus} an
important invasive parasite that threatens wild stocks of the European
eel \textit{An. anguilla}. These data enable a broad spectrum of
molecular research on this ecologically and economically important
parasite. As \textit{A. crassus} lives in close association with its
host, I have used exhaustive filtering to attempt to remove all
host-derived, and host-associated organism-derived contamination from
the data. To do this I have also generated a transcriptome dataset
from the definitive host \textit{An. japonica}. The non-nematode,
non-eel data identified, particularly in the L2 sample, showed highest
identity to flagellate protists, which may have been parasitising the
eel (or the nematode). Encapsulated objects observed in eel swim
bladder walls \cite{heitlinger_massive_2009} could be due solely to
immune attrition of \textit{A. crassus} larvae or to other
coinfections.

A second examination of sequence origin was performed after assembly,
employing higher stringency cutoffs. Similar taxonomic screening was
used in a garter snake transcriptome project \cite{pmid21138572}, and
an analysis of lake sturgeon tested and rejected hypotheses of
horizontal gene-transfer when xenobiont sequences were identified
\cite{pmid20386959}. A custom pipeline for transcriptome assembly from
pyrosequencing reads \cite{pmid20034392} proposed the use of EST3
\cite{pmid17218127} to infer sequence origin based simply on
nucleotide frequency. I were not able to use this approach
successfully, probably due to the fact that xenobiont sequences in our
data set derive from multiple sources with different GC content and
codon usage.

Compared to other NGS transcriptome sequencing projects
\cite{pmid20478048}, the combined assembly approach (see
\ref{sec:over-eval}) generated a smaller number of contigs that had
lower redundancy and higher completeness. Projects using the
\texttt{Mira} assembler often report substantially greater numbers of
contigs for datasets of similar size (see e.g. \cite{pmid21364769}),
comparable to the mira sub-assembly in our approach. The use of
oligo(dT) to capture mRNAs probably explains the bias towards 3' end
completeness and a relative lack of true initiation codons in our
protein prediction. This bias is near-ubiquitous in deep transcriptome
sequencing projects (e.g. \cite{pmid20331785}).

I were able to obtain high-quality annotations for a large set of
TUGs: For 40\% of the complete assembly and 60\% of our highCA
assembly \texttt{BLAST}-based annotations could be obtained. 45\% of
the contigs in the highCA assembly were additionally decorated with
domain-based annotations through \texttt{InterProScan}
\cite{pmid11590104}.

Comparison with complete protein sequence from the genomes of
\textit{B. malayi} and \textit{C. elegans} showed a remarkable degree
of agreement regarding the occurrence of terms in the two parasitic
worms. This agreement was higher than with the free living nematode
\textit{C. elegans} and even the two genome-sequencing-derived
proteomes showed less agreement with each other than the filarial
parasite with our dataset. This implies that our transcriptome is
truly a representative partial genome
\cite{parkinson_partigene--constructing_2004} of a parasitic nematode.

Analysis of conservation identified more sequence novel in Nematode
than in the eukaryote kingdom or in clade III this is in agreement
with prevalence of genic novelty in the Nematoda
\cite{wasmuth_extent_2008}. Furthermore the basal position of
\textit{A. crassus} in clade III could be leading to most novelty in
the clade not being shared with \textit{A. crassus}.

TUGs predicted to be novel in the phylum Nematoda and novel to
\textit{A. crassus} contained the highest proportion of
signal-positives. This confirms observations made in a study on
\textit{Nippostrongylus brasiliensis} \cite{harcus_signal_2004}, where
signal positives were reported as less conserved. Interestingly
enrichment of signal sequence bearing TUGs in our dataset was
constrained to sequences novel in nematodes and \textit{A. crassus}
(i.e. not to the level of clade III). This may be explained, with two
different hypotheses involving the basal position of
\textit{A. crassus}: First the signal positives shared with all
nematodes could be conserved molecules not excreted by parasites. A
different class of secreted/excreted molecules with prominent role in
host parasite interactions would not have arisen early in the
evolution of parasitism in clade III - or be too fast-evolving - and
thus be detected as specific to deeper sub-clades (i.e. to
\textit{A. crassus} in our dataset). A second explanation would be,
that orthologs of excreted parasite-specific genes could be among
those shared with other nematodes and the fewer shared with clade III
implying a predisposition to parasitism outside of the Spirurina or
even the convergent evolution of secreted molecules in other parasitic
nematodes. However analysis of dn/ds (see below) across conservation
categories favours the first hypothesis, as it identifies a higher
amount of positive selection in TUGs novel to clade III and
\textit{A. crassus} than to nematodes.

I generated transcriptome data from multiple \textit{A. crassus} of
Taiwanese and European origin, and identified SNPs both within and
between populations. Screening of SNPs in or adjacent to homopolymer
regions improved overall measurements of SNP quality. The ratio of
transitions to transversions (ti/tv) increased. Such an increase is
explained by the removal of ``noise'' associated with common homopolymer
errors \cite{pmid21685085}. The value of 1.93 (1.25 outside, 2.41
inside ORFs) is in good agreement with the overall ti/tv of humans
(2.16 \cite{pmid21169219}) or \textit{Drosophila} (2.07
\cite{pmid21143862}). The ratio of non-synonymous SNPs per
non-synonymous site to synonymous SNPs per synonymous site (dn/ds)
decreased with removal of SNPs adjacent to homopolymer regions from
0.42 to 0.231 after full screening. The most plausible explanation is
the removal of error, as unbiased error would lead to a dn/ds of
1. While dn/ds is not unproblematic to interpret within populations
\cite{pmid19081788}, the assumption of negative (purifying) selection
on most protein-coding genes makes lower mean values seem more
plausible. I used a threshold value for the minority allele of 7\%
for exclusion of SNPs, based on an estimate that approximately 10
haploid equivalents were sampled (5 individual worms plus an
negligible contribution from L2 larvae in the L2 library and within
the female adult worms). The benefit of this screening was mainly a
reduction of non-synonymous SNPs in high coverage contigs, and a
removal of the dependence of dn/ds on coverage. Working with an
estimate of dn/ds independent of coverage, efforts to control for
sampling biased by depth (i.e. coverage; see \cite{pmid18590545} and
\cite{pmid20478048}) could be avoided.

Also in comparison with published intra-species values of dn/ds our
final estimate seems plausible: in transcripts from the female
reproductive tract of \textit{Drosophila} dn/ds was 0.15
\cite{pmid15579698} and 0.21 in the male reproductive tract
\cite{pmid11404480} (although for ESTs specific to the male accessory
gland were shown to have a higher dn/ds of 0.47). A pyrosequencing
study in the parasitic nematode \textit{Ancylostoma canium}
\cite{pmid20470405} reported dn/ds of 0.3.

When the whole of coding sequences are studied, of which only a small
subset of sites can be under diversifying selection, dn/ds of ~0.5 has
been suggested as threshold for assuming diversifying selection
\cite{pmid15579698} instead of the classical threshold of 1
\cite{pmid6449605}. The use of this threshold for positive selection
led to the identification of over-represented of GO-term highlighting
very interesting transcripts:

Twelve peptidases under positive selection (from 43 with a dn/ds
obtained) meant an enrichment in the category. All twelve have different
orthologs in \textit{B. malayi} and \textit{C. elgans} and are
conserved across kingdoms. Despite their conservation peptidases are
thought to have acquired new and prominent roles in host-parasite
interaction compared to free living organisms: In \textit{A. crassus}
a trypsin-like proteinase has been identified thought to be utilised
by the tissue-dwelling L3 stage to penetrate host tissue and an
aspartyl proteinase thought to be a digestive enzyme in adults
\cite{polzer_identification_1993}. The twelve proteinases under positive
selection could be the targets of the adaptive immunity developed
against \textit{A. crassus} \cite{knopf_migratory_2008,
  knopf_vaccination_2008}, which is often only elicited against
subtypes of larvae \cite{molnar_caps}.

The under-representation of ribosomal proteins (term ``structural
constituent of ribosome'') in positive selected contigs is in good
agreement with the notion that ribosomal proteins are extremely
conserved across kingdoms \cite{pmid9664699} and should be under
strong negative selection.

Genotyping of individual worms identified a set of 199 SNPs with
highest credibility and a high information content for
population-genetic studies. Levels of genome-wide heterozygosity found
for the 5 adult worms examined in our study are in agreement with
microsatellite data \cite{wielgoss_population_2008} showing reduced
heterozygosity in European populations of \textit{A. crassus}.

I employed methods to developed for the comparison of cDNA-libraries
to make inference about possible differential gene-expression
according to experimental groups (origin of sequencing-libraries)
\cite{pmid9331369}. Such approaches are widely used with
pyrosequencing-data (e.g. \cite{pmid20470405}). For the statistically
valid comparison of conditions however, the unit of replication would
be the individual library and approaches respecting this fact would be
desirable. However, I was not able to use the R-packages
\texttt{DESeq} \cite{pmid20979621} or \texttt{edgeR}
\cite{pmid19910308} developed for count data from deep sequencing (but
more targeted towards RNA-seq on the solexa-platform) as both
repetition and throughput of our present experiment were too low. As a
result the differentially expressed genes are by no means significant
for the investigated conditions, but just for the specific
cDNA-libraries. With these reservations we identified genes
differentially expressed between libraries prepared from worms of
different sex and worms from different origin.

Genes over-expressed in male \textit{A. crassus} comprise major sperm
proteins well known for their high expression in nematode sperm
\cite{pmid15275275}. A surprise was the overexpression of ribosomal
proteins in the male library.

That collagen processing enzymes are overexpressed in female worms,
filled with developing embryos and larvae, is in line with a
complicated regulation and modulation of collagen in nematode larval
development \cite{pmid10637627}.

The overexpression acetyl-CoA acetyltransferase in European worms are
interesting especially because of the role of these enzymes in
fatty-acid $\beta$-oxidation in peroxisomes and mitochondria
\cite{pmid4721607}. Together with a change in steroid metabolism and
the enrichment of mitochondrially localised enzymes these are
suggestive of changes in energy metabolism of \textit{A. crassus} from
different origins. Possible explanations would include a change to
more or less aerobic processes in worms in Europe due to their bigger
size and/or increased availability of nutrients.

Contigs overexpressed in the female libraries showed elevated levels
of dn/ds but genes overexpressed in males decreased levels of
dn/ds. The first finding is unexpected, as overexpressed in female
libraries will also contain contigs related to larval development
(such as the collagen modifying enzymes discussed above), these larval
transcripts in turn are expected to be under purifying selection
because of pleiotropic effects of genes in early development
\cite{pmid15371532}. Also the second finding is in slight contrast to
published results for male specific traits and transcripts are often
showing hallmarks of positive selection
\cite{pmid15795858,pmid11404480}. In \textit{Ancylostoma caninum}
however, female-specific transcripts showed an enrichment of
``parasitism genes'' \cite{pmid20470405} and a possible expansion
would be a similar enrichment of positively selected parasitism
related in our dataset. For males the decreased dn/ds can be explained
by the by the high number of ribosomal proteins, which are all show
very low levels of dn/ds (that these proteins are found differentially
expressed remains puzzling though), while single transcripts
e.g. major sperm protein (expressed in the male library only) showed
elevated dn/ds but did not level the overall effect. But this also has
a positive aspect: it is unlikely that correlation of differential
expression with positive selection results from mapping artefacts, as
all the ribosomal proteins identified overexpressed in males have very
low dn/ds.

Genes differential expressed according to worm-origin (in either
direction) showed significantly elevated levels of dn/ds. This is
interpretable as a correlation between sequence evolution and
phenotypic modification in different host-environments or even
correlation between sequence evolution and evolution of
gene-expression. Thus, whether expression of these genes is modified
in different hosts or evolved rapidly in a contemporary divergence
between European and Asian populations of \textit{A. crassus}, is in
the centre of a future research program building on the reference
transcriptome presented here. For such an analysis it is important to
disentangle the influence of the host and the nematode population in a
co-inoculation experiment. Such a project will also use the individual
worm as the level of replication for ``conditions'' (that is,
worm-population and host-species) to allow rigid hypothesis
testing. Based on the pilot evaluation presented here differences in
these factors are expected overlap with differences in male vs. female
worms and the careful cross-examination of the above factors with
worm-sex is advised.

The \textit{A. crassus} transcriptome provides a basis of molecular
research on this important species. It further provides insight in the
evolution of parasitism complementing the catalogue of available
transcriptomic data with a member of the Spirurina phylogenetically
distant to so far sequenced parasites in this clade. Differences in
energy metabolism between European and Asian \textit{A. crassus}
constitute a candidate phenotype relevant for phenotypic modification
or contemporary divergent evolution as well as for the long term
evolution of parasitism.


\section{Transcriptomic divergence in a common garden experiment}
\label{sec:exp-inf}

With some reservations discussed below our observation of higher
recovery of adult worms from sympatric
\textit{A. crassus}-\textit{Anguilla} spp. host-parasite combinations
imply local adaptation of different worm populations to host-species.

The precentage of recovered European worms is in agreement with data
from \cite{knopf_differences_2004} (33.2\% for the host-parasite
combination sympatric in Erope and 13.8\% for European worms applied
back to \textit{An. japonica}). This pattern of recovery was precisely
mirrored for the Taiwanese population of \textit{A. crassus}, for
which thus recovery was rouhgly 30\% in the sympatric
\textit{An. japonica} and only 10\% in \textit{An. anguilla}. These
Data are not in complete agreement with findings by Weclawski et
al. (unpublished; see \ref{div-ac}), who recorded recovery at only
slightly different timepoints after infection (25, 50, 100 and 150
dpi). Similar to my study they found a higher recovery of the European
population of worms in the European eel but did not find the
complementary result of lower recovery of this diverged population in
the Japanese eel. A possible explanation for these different results
are interactions of host-parasite genotypes conditional on the
environment (GxGxE interactions, see also \ref{div-ac}). It is
imaginable that the environment provided in the common-garden setting
slightly differed between the two experiments (despite the fact, that
these experiments were performed in the same experimental setup).

It has to be emphasised that the observations made in common-garden
experiments first and foremost have to be interpreted as
phenotypes. An ideally suited phenotype to infer local adaptation
would be one with obvious direct fitness-consequences, a so called
fitness-component. Such a fitness-component would ideally be a
measurement on a single individual, and individual life-time
reproductive success would be such an ideal measurement. However, the
techniques used to measure individual life-time reproductive success
have not been established in \textit{A. crassus} and it would be very
difficult to do so.

The recovery of certain developmental stages of worms is only a proxy
interpretable as a fitness-component. It is a composite measurement of
the speed of development from previous lifecycle stages (or speed of
migration towards the swimbladder) and of survival. While survival is
surely an important component of fitness, it is not completely clear
whether fast development and/or migration to the swimbladder are. It
is possible that under certain conditions slower development could
lead to higher fitness, if it would, for example allow development
without attracting the attention of the immune system.

Another slight problem with recovery in these experiments is that it
is a mean measurement over many individuals. If one would want to find
genotype associations with the most suboptimal phenotype it would not
be possible to isolate individuals bearing this trait, because these
would be dead or still on their way migrating to the
swimbladder. Apart from the problem of clear definition and
measurement, lifecyle traits are also notoriously complex in the
underlying genetic architecture \cite{pmid18454194}.

I decided on a study design using pools of individuals for one sex
(males) and single individuals for the other. A study on
\textit{Fundulus heteroclitus} revealed that approximately 18\% of the
transcripts are differentially expressed between individual fish from
the same population, grown under controlled environmental conditions
\cite{pmid12219088}. And it thus not surprising, that between
individual variation in female samples was leading to higher variance
of these female samples compared to pooled male samples in my study.

This interindividual variation in gene-expression under a particular
environmental condition is generally agreed to be closely linked to a
genetic basis \cite{pmid15498452}. For example in a cross between two
parental strains of yeast the genetic component of variation was
estimated from haploid segregants to be 84\% \cite{pmid11923494}. The
genetic component was found to be the main factor determining
expression level variability between two strains, sexes and ages of
\textit{Drosophila melanogaster} for 267 (7\%) from 3,931 genes and at
least 25\% of the transcriptome were estimated to be affected mainly
by genotypic factors in any of the groups
\cite{pmid11726925}. Variation in the regulation of gene-expression is
thought to constitute a major source of evolutionary novelty
\cite{pmid11341673}.

A second study from the line of research on \textit{Fundulus
  heteroclitus} \cite{pmid16567645} used genetic relatedness as
inferred from phylogenetics to separate variation in gene-expression
in a common experimental environment into a neutral component and a
selected component, this way removing variation most likely accounted
for by the shared neutral evolutionary history. My case of
\textit{A. crassus} is potentially simpler: the investigated European
populations are a direct descendants and thus a subset of a Taiwanese
source populations. In fact I studied two European and two Taiwanese
populations as a few hundred kilometers between the geographical
origins of the two different locations in Germany and Taiwan probably
constitute a barrier to gene-flow in a parasite with an aquatic
intermediate host. However, I treated worms from both European and
Taiwanese populations as replicates (and use the terminology of one
European and one Asian population throughout the text) with the
rationale of increasing variance for random genetic differences and
raising the bar for potentially adaptive differences to be detected.

Given the sampling of only twelve Taiwanese worms the question could
be raised, whether these consititute a representative sample of the
true source population, of which a sub-population was funding European
populations. A microsattelite study indicated gene-flow even between
populations of \textit{A. crassus} seperated by thousands of
kilometers in Asia (Japan and Taiwan)
\cite{wielgoss_population_2008}. Given the high interconnectivity of
Taiwanese water systems used for aquaculture both by man-build
structural links and anthropogneic exchange of fish, a sampling from
two Taiwanese populations similarly neutrally diverged from the true
European funding population seems very unlikely. The worms sampled
from Taiwan can thus be regarded a sample of the (meta-)population
appropriate for finding differences in ralation to the source of the
introduction.

When I later venture into adaptive interpretations of the observed
gene-expression differences it has to be remembered, that these
constitute nothing more than a molecular phenotype. This phenotype is
not necessarily a fitness-component. It is one of the dangers of
genomic data to forget the fundamental lesion from the debate
initiated by Gould and Lewontin in 1979
\cite{gould_spandrels_1979}. Briefly, while functional changes are
often caused by selection, differences in function do not necessarily
demonstrate the past or present action of selection. There is no way
to infer the action of selection based on functional considerations,
and even if selection can be inferred otherwise, it is not necessarily
a pariticulare observed variable trait that selection acted on
\cite{pmid19744124}.

Of no surprise was the abundance of differential expression between
male and female worms in roughly one third of the genes. A large
number of genes are known to be sex-specific, regulating ovulation and
spermatogenesis throughout the metazoa and especially in nematodes
\cite{pmid15371532}. On top of these sex-specific genes there are
large numbers of genes differently expressed due to differences in
metabolism between males and females. Estimates for
\textit{Drosophila} based on similar sample sizes to those used in my
study range between one- and two-thirds of the transcriptome showing
sex-biased expression \cite{pmid11726925}. In the liver transcriptome
of \textit{Mus musculus}, even 70\% of transcripts have been shown to
differ between sexes \cite{pmid16825664} (note however, that this
study used 169 female and 165 male mice to guarantee the finding of
even the most subtle differences). Given the scale of these
differences in other species my estimate of roughly a third of the
transcripts in \textit{A. crassus} showing differential expression
according to the sex of the worms implies conservative thresholds used
in the statistical analysis and moderate power for detection of
differeces.

Nearly the same proportion (roughly 30\%) of contigs was confirmed
through summation and analysis of contigs for orthologs in
\textit{B. malayi} and \textit{C. elegans}. Development of this
orthologous confirmation method was necessitated by the possibly
fragmented and chimeric transcriptome assembly. This introduces
stringent conditions for the detection of significance, as p-value
correction for multiple testing is employed during each analysis (once
for raw counts and twice for orthologous counts). Although the
underlying tests are not independent, the false discovery rate of 5\%
for raw contigs can be expected to be immensely lowered by applying an
FDR of 10\% twice.

In addition biological implications could produce false negatives in
such an evaluation: All genes duplicated in \textit{A. crassus} (a)
and following antithetic expression patterns will be evaluated
negatively, as will duplicated genes in any of the model-species (b)
following such a pattern. However there is no other choice then
applying these stringent conditions to screen for artefacts producing
the same patterns based on mapping to fragmented (a) or chimeric (b)
reference contigs. I think that an evaluation based on this
scrutinised confidence in an assembly previously computed from
454-data is even more appropriate then an analysis solely based on
counts collapsed for orthologs excluding only possible fragmentation
artefacts (as used e.g. in \cite{pmid22084086}).

In general, my statistical analysis aimed to minimise false positives
(type I error) at expenses of possible false negatives (type II error)
and is thus not fully suited to address the proportions of
differentially regulated genes.

Nevertheless it is surprising that less than 1\% of transcripts were
detected differentially expressed between worms in different
host-species and less then 0.3\% were confirmed with the orthologous
method. This was an unexpected finding, as the differeces in the
immune response of the host species have a big influence on other
phenotypes of worms \cite{knopf_swimbladder_2006}. In addition to the
low number of genes, multifactorial analysis revealed that below 10\%
of the variance could be explained by host-species effect, even in
significantly differential regulated genes for this factor.

Although these differences between worms in different host species
were the most marginal of any of the factors, it is possible to
connect some (at least two) of the genes to a prominent physiological
difference: the digestion of haemoglobin. Two different aspartic
proteases (both confirmed through orthologs, one of them differing for
all three main effects, the other for an interaction of worm-sex and
host species) known to be involved in the first steps of digestion of
haemoglobin from other nematodes \cite{pmid12782060} were
overexpressed in worms in \textit{An. anguilla}. This expression
phenotype could potentially be linked to the often observed phenotype
of bigger size of \textit{A. crassus} in this host
\cite{knopf_swimbladder_2006}, as the main contribution to this
increase in size is the larger volume of host-blood taken up by the
parasite. Accordingly the parasite probably digests haemoglobin at a
higher rate.

Close to 1\% of contigs were significantly different in expression
between European and Asian \textit{A. crassus}, making this difference
significant for a higher number of contigs than the
host-differences. For this contrast the proportion of orthologous
confirmation was lower than for sex differences but higher than for
host-species differences. Additionally multivariate analysis of all
differently expressed transcripts for worm-population revealed that
the variance contributed by the population-factor was higher than 10\%
for all significant contigs or even 20\% for orthologous confirmed
contigs.

Another important finding was the large overlap in contigs expressed
differentially depending on worm-sex and worm-population. Such an
overlap is expected if genes expressed differentially according to sex
are evolving faster towards a differential expression according to
other factors. Faster evolution of reproductive (and especially male
specific traits) have been shown in many species at a phenotypic and
at a sequence level \cite{pmid15795858}. In \textit{Drosophila}, male
reproductive proteins have been shown to evolve at elevated levels and
under positive selection \cite{pmid11404480}. Moreover, gene
expression should evolve at a higher rate in sex-specific genes,
indeed the transcriptomes of \textit{Drosophila} species show that
interspecific expression divergence is sex dependent and the action of
sex-dependent natural selection during species divergence has been
inferred from this \cite{pmid15034135,pmid19720861}.

Taken together, my findings strongly support a larger influence of the
genetic differences between European and Asian populations
\textit{A. crassus} than of the modification in the different
host-species on gene-expression. When additive and interaction effects
are considered, the influence of host-species even vanishes almost
completely in favour of a combination of effects combining parasite
population and sex of the worms.

From a functional perspective, genes identified to differ between
populations can be categorised as important in general metabolic
processes instead of specific host-parasite interactions. This
constitutes a negative evaluation of one of my \textit{a priori}
hypotheses based on finding parasite-specific genes, identified as
vaccine candidates in a number of nematodes, within the genes modified
or diverged in my study (\ref{sec:dna-sequ-nemat}). However, more
direct host-parasite interactions are expected in tissue-dwelling
larval stages (L3 and L4) and in fact most immunomodulators are
expressed predominantly in these stages
\cite{maizels_helminth_2004}. Adults of \textit{A. crassus} could thus
be the wrong lifecycle stages to detect such expression differences,
if they existed.

\figuremacro{tGO_pop_MF_classic_10_all}{GO molecular function graph
  for enriched terms in DE according to worm-population}{Subgraph of
  the GO-ontology molecular function category induced by the top 10
  terms identified as enriched in DE genes between different parasite
  populations. Boxes indicate the 10 most significant terms. Box colour
  represents the relative significance, ranging from dark red (most
  significant) to light yellow (least significant). In each node the
  category-identifier, a (eventually truncated) description of the term,
  the significance for enrichment and the number of DE / total number
  of annotated gene is given. Black arrows indicate a is ``is-a''
  relationship.}

Instead enzymes and enzyme subunits important for aerobic respiration
are especially expressed at lower levels in European
\textit{A. crassus}. In fact, most transcripts significantly differing
between populations were annotated as ``oxidoreductase'' in
gene-ontology. 
% While at face values of single genes have to be interpreted with
% care such trends in the data are very likely to point to important
% differences.
Downregulation of cytochrome C oxidase subunit 2 (COXII) in the
European population of \textit{A. crassus} was the most persistent
finding. This downregulation was confirmed by the high expression of
the same contig in all tree libraries from Taiwanese worms and very
low expression in the Euopean librarises evaluated by pyrosequencing.
Cytochrome C oxidase subunits 1-3 are are essential components of
complex IV, the cytochrome c oxidase. They are encoded in the
mitochondrial genome and especially subunit 1 and 2 coordinate
catalytic haeme and copper cofactors \cite{pmid18023115}.

In fact, not only enrichment analysis highlighted oxidoreductases, but
expression values of COXII clustered with other enzymes from energy
metabolism: two lecitin:cholesterol acyltransferase transcripts are
putative recently duplicated genes. They showed slightly divergent
protein sequences but hit the same orthologs in \textit{C. elegans}
and \textit{B. malayi}. They also shared very similar expression
profiles. Expression of different cholesterol acyltransferases has
been shown to vary in response to the presence of haeme and
anaerobiosis in yeast \cite{pmid11786267}. 3-hydroxyacyl-CoA
dehydrogenase (involved fatty-acid $\beta$-oxidation
\cite{pmid8454629}), malate/L-lactate dehydrogenase (from the
anaerobic glycolytic pathway or the krebs-cycle
\cite{sturm1969vergleichende}) and aspartyl proteases (involved in the
digestion of host haemoglobin in helminths \cite{pmid12782060})
completed this particular cluster.

These patterns can be interpreted as a biological confirmation of the
at face values for single genes, especially for COXII. In addition the
differential reaction of metabolic genes to different factors (genetic
vs. modification) invites speculation on a causal structure behind
these correlations. The expressions of metabolic enzymes are
interpretable as a change to use more anaerobic metabolism in the
European population of \textit{A. crassus}. In one possible scenario
in European worms, one of the subunits of core enzymes of the
respiratory chain (probably COXII) would have evolved a genetically
fixed lower level of expression. This model follows the logic, that
the most differential expressed gene could be the driver of observed
change. Other enzymes related to aerobic energy metabolism (e.g. lipid
metabolism) indirectly and only partially controlled by feedback
mechanisms from the respiratory chain and the citric acid cycle would
show similar patterns of altered expression in European
worms. However, the expression of some these indirectly and also by
additional environmental factors controlled genes would be perturbed
when worms are applied back to their Asian hosts. Also in the two
sexes differences in size and metabolism would be perturbing the
pleiotropic effects of the persistent core-change.

\figuremacro{pop_ortho_heat}{Clustering of expression values for OC
  contigs DE between populations}{A heatmap of variance/mean
  stabilised expression values. Deprograms are based on euclidean
  distance and hierarchically clustering. Green indicates expression
  below the mean, red above the mean. Experimental conditions are
  indicated by black bars for groups of samples (columns) below the
  plot. Expression levels for libraries are clustering mainly
  according to the sex of worms. However, in both male and female
  worms subordinate clusters are following a worm-population and to a
  lesser extend (and mainly in males) a host-species pattern. Below
  contig-names uniprot names are given for ortholog genes in
  \textit{B. malayi}. Genes are clustering according to
  annotation-profiles: the top cluster represents genes important in
  energy metabolism. They cluster with COXII, which shows clear
  overexpression in - without any exception - all libraries from
  Taiwanese worms.}

Such a scenario also provokes some speculation about the adaptive
value of such a change in a core metabolic process:\\
Aerobic respiration is potential source for oxidative stress providing
a steady source of reactive oxygen species (ROS) as electrons are
leaking from the respiratory chain as superoxide anions. It is well
established that such ROS production is especially harmful to
blood-feeding parasites, as free inorganic iron, as well as haeme,
have the potential to generate additional ROS
\cite{pmid21087517}. Anaerobic metabolism is thus thought to occur in
many haematophagous parasites as a counter-measure against oxidative
stress from haemoglobin catabolism \cite{pmid12163151}. It could thus
be hypothesised that the bigger size and the larger amount of
eel-blood ingested leading to a higher rate of haemoglobin digestion
provided the selective pressure to reduce aerobic
respiration. Additionally helminths can simply get too large to
maintain oxygen diffusion to mitochondriae in the absence of a
cardiovascular system. As yet proton-pumping electron transport
constitutes the most profitable energy-providing process, the
mitochondriae of facultatively anaerobic helminths produce an electron
gradient for the use of ATPase with the help of acceptors other than
O$_2$ \cite{pmid12417132}. Such an alternate electron sink is fumarate
used in many helminths in a process called malat dismutation
\cite{pmid15275412}.

An interesting implication is that such metabolic differences could
potentially be visible ultrastructurally. Indeed in my own diploma
thesis \cite{heitlinger_vergleichende_2008} I identified two different
kinds of mitochondriae, one with standard christae-like morphology,
the other with unusual sacculus-like morphology in
\textit{A. crassus}. Additionally I observed less electron-dense
inclusions (probably lipid reserves) in bigger worms and more glycogen
granulae. The fact that such lipids are less usable under anaerobic
conditions led me to the hypothesis, that bigger worms are using less
aerobic processes. Reanalysing this data and probably obtaining new
data with additional histochemical staining methods could be a way to
put gene-expression into a physiological perspective. Furthermore, a
biochemical examination of isolated mitochondriae could highlight
changes in the mitochondrial respiratory chain under \textit{in vitro}
conditions \cite{pmid18314717}. Such direct measurements of COX enzyme
acitivity (using well established assays \cite{pmid8592440}) would be
desirable to establish even the validity of the first logical step in
these adaptive speculations that underexpression of COXII is leading
to decreased enzyme acivity. It would be counterintuitive to expect
higher enzyme activity when COXII mRNA levels are low, but, for
example, in \textit{Schistosoma mansoni} COXI over-expression in
praziquantel-resistant strains is leading rather to decreased enzyme
activity \cite{pmid9695101}.

The sensitivity to preturbation of mitochondrial genes for respiratory
chain complexes in nematode parasites is underlined by their
up-regulation after depletion of Wolbachia from filarial nematodes
\cite{pmid20362581, pmid19806204}. Wolbachia are obligate endosymbiont
bacteria of some clade III nematodes they are supplying haeme to
non-heamotophagous parasites in the abscence of intrinsic pathway for
haeme synthesis \cite{ghedin_draft_2007} (which is absent also in free
living \textit{C. elegans} \cite{pmid15767563}). While our sequence
analysis suggests the abscenece of Wolbachial symbionts in
\textit{A. crassus}, such studies support a cetral role of host or
endosymbiont derived haeme for respritory processes and suggest a
propensity for evolutionary change in related processes (in filaria
even acquisition of an endosymbiont).

Assuming a genetically fixed lower expression of COXII in Euopean
\textit{A. crassus} as a driver for other metabollic differences does
not imply a simple regulation of the expression itself, or a
genetically simple change undlying the changed expression
phenotype. Regulation of the mitochondrially encoded genes has been
extensively integrated into the regulatory netowork of eukaryotic
cells and is contolled by and interacting with nuclear transcription
factors \cite{pmid8289797}.

Intriguingly overexpression of respiratory chain enzymes was limited
to cytochrome c oxidase transcripts (COXII and to lesser extent also
COXI and COXIII). Mitochondrial transcription produces multiple
polycistronic unmatured transcripts, which are cleaved and modified in
their expression post-transcriptionally. Cleavage occures at t-RNA
sequences intersparesed between protein coding genes and can be
imprefect to laeve some transcripts poly-cistronic in a matured
state. Nevertheless, individual transcripts can be expressed
uncoordinated, even when expressed on the same unmatured
poly-cistronic transcript \cite{pmid19843606}, for example, the
addition of poly-A tails is vital for stability of matrue transcritps
in metazoans. The mitochondrial genome contains only very little
untranscribed sequence and is haploid and transmitted completely
linked, without recombination \cite{pmid18023115}. Cis-regulatory
change in a control region would thus be very likely detectable in our
transcriptome data. Even if the sequence variation leading to the
observed expression phenotypes would locate to the untranscribed
hypervariable mitochondrial control region (in D-Loop associated
promoters), selection on such a variant would render the whole
mitochondrial genome inadequate for phylogenetic analysis, as a
variant sweeping to fixation would have removed polymorphism from the
complete mitochondrial genome due to the prefect linkage
\cite{pmid19821901}. Such a pattern has not been found in populations
of \textit{A. crassus} in Europe when COXI was used as a marker
\cite{dl_py} (see also figure \ref{mCOXI-phylo}) and is also not
visible from preliminary analysis of polymorphism in mitochondiral
genes in my RNAseq data.

Constraints are also expected regarding the mechanism by which the
expression of COXII could evolve. Most infective L3 larvae of prasitic
nematodes rely on aerobic respiration. Dixenous parasites like
\textit{A. crassus} migrate through tissue of definiteve hosts, where
oxygen is readily available, after leaving the haeomocoel of the
intermediate host \cite{kennedy2001parasitic}. Enzyme subuntis
building a functioning aerobic resriatory chain are thus likely to be
expressed at earlier lifecycle stges of \textit{A. crassus} and
anaerobiosis is expected to be restricted to the adult stages.

These considerations make sole or predominat cis-reglatory change in
mitochondrial DNA very unlikely to explain the divergent expression
phenotypes. Still identification of the genetic architecture, for
example sequence variation in a transcription factor, a co-factor or a
protein modifying mitochondrial transcripts, may be possible (to a
limited extent even in the present RNAseq data). 

RNAi screens in \textit{C. elegans} for increased lifespan focus on
genes leading to lower oxygen consumption and altered mitochondrial
morphology and function \cite{pmid12447374}. Such candidate genes will
provide an additional link back from screening of genomic regions with
signature of selection to functional considerations.

% Numts should normally be avoided by reverse transcription, but
% occasionally Numts are transcribed

% (Blanchard, J.L. and Schmidt, G.W. (1996) Mitochondrial DNA migration
% events in yeast and humans: Integration by a common end-joining
% mechanism and alternative perspectives on nucleotide substitution
% pattern. J. Mol. Evol. 13, 537–548).

% Pseudo-mRNAs shadowy entities that resist classification and analysis
% resemble protein-coding mRNA, but cannot encode full-length proteins
% owing to disruptions of the reading frame.  \cite{pmid16683022}

A second group of genes differentially expressed in populations of
\textit{A. crassus} emerged from both cluster and enrichment
analyses. Two transcripts in this cluster were significant for
interaction effects between host-species and parasite-population, they
were annotated as collagens. For both genes this meant an ``adjusted''
(to avoid the suggestive ``adapted'') expression difference leading to
a lower expression in sympatric host-species/parasite-population
pairs. Cuticle collagens are a large multigene family (Interpro lists
164 entries for ``Nematode cuticle collagen, N-terminal''
(\href{http://www.ebi.ac.uk/interpro/ISpy?ipr=IPR002486&tax=6239}{\textit{C. elegans}
}) and 51 for
\href{http://www.ebi.ac.uk/interpro/ISpy?ipr=IPR002486&tax=6279}{\textit{B. malayi}}),
containing extensive repeat regions (roughly 50\% Gly-X-Y residues,
often Gly-Pro-Hpy). In the genome of \textit{B. malayi} 82 genes
encoding collagen repeats have been found \cite{ghedin_draft_2007}. It
was thus very important to have orthologous confirmation for these two
contigs, as misassembly could have easily lead false positives here.

The two collagens were clustered with a third contig sharing a
collagen-annotation (failing to be significant for the interaction
term probably because of low overall expression) and a contig
annotated as ``Matrixin'' (metallo-proteinase assumed to be involved
in remodelling the extracellular matrix \cite{mealloprot}) and a
ABC-transporter family protein. This can be interpreted as a
confirmation of differential expression via clustering with
functionally related proteins.

Functional speculations are more difficult for collagen than for the
respiratory chain enzymes. The cuticle constitutes an exoskeleton and
a barrier between the worm and its host-environment. Synthesis of most
collagens is believed to occur at negligible levels in adult male
worms and is rather constrained to discrete temporal periods in larval
development, the moults \cite{pmid10637627}. The differential
expression could thus be due to changes in larval development or due
to alternations in the low-level, steady renewal of the adult cuticle
and remodelling of the extracellular matrix. Some considerations would
favour of the second explanation: in \textit{C. elegans} genes
expressed after reproductive maturity evolve faster than genes
expressed earlier in development \cite{pmid15371532}. This suggests a
model of elevated pleiotropic effects in genes expressed at earlier
stages of development and hence more conserved expression patterns in
larval stages. Independent of these considerations, both the primary
assembly and the constant remodeling of the cuticule involve complex
post-translational processes hardly accessible at the transcriptomic
level: a zipper-like nucleation/growth mechanism leads to the folding
of a triple helix of and heterotrimers and homotrimers
\cite{kennedy2001parasitic}. If and how differential exression of two
particular collagens interferes with this process requests further
research. As for the metabolic differences, differential expression
patterns could be reflected in morphology. One approach would be to
measure thickness and density of the cuticle of worms from
cross-inoculation experiments.

\section{Outlook}

The presented project on divergence of gene expession obviousely
constitutes work in progress. The observed differences in subunits of
respiratory chain enzymes, expecially in COXII, necessitate and permit
confirmation by reverse transcription quantitative PCR (RTqPCR) for
these tanscripts. Such evaluations of a single gene (or few genes)
will be possible on many individual specimen of \textit{A. crassus}
from both Europe and Taiwan to fruther test the signifcance of the
observed differences. Therefore, in additon to the validation of
expression values for sequenced samples, many of the worms from the
presented coinoculation experiment yielding lower amounts of RNA
inadequate for sequencing will be used to further establish the
divergence in gene-expression. Additionally sampling of worms from
their present day sympatirc hosts is possible for genes differing only
for populations unconditional on eel-host species. Moreover, if
selection in Europe would have acted on standing variation, one would
expect to find worms expressing for example COXII at low levels also
in the Taiwanese source populations, at least in low frequency. Thus
hundreds of individual worms from Taiwanese populations will be tested
as new funding becomes available. Appropriate \textit{A. crssus}
samples stored in RNA-later are readily available from broad sampling
for the present transcriptome projects from populations of worms in
both wild and cultered \textit{An. japonica}.

An assembly of the mitochondrial genome of \textit{A. crassus} from
preliminary genome-sequencing data (discussed below) and the
idntification of the poly-cistronic unmatured and, if present, matured
transcripts (similar to \cite{pmid19843606}), will further inform and
validate the analyisis on the expression of mitochondrial genes.

Multiple starting points also exist for further functional examination
of metabolic change, as mentioned throughout the text. However, the
search for ultimate causes for evolutionary change \textit{sensu}
\cite{mayr1961cause} will potentiall be even more rewarding.

In parallel to the RTqPCR evaluations I will conduct a study on the
association between gene expression and sequence variants. This kind
of quantitive expression trait locus (eQTL) mapping is possible as
both sequence and expression infromation are available from the
present mRNA-seq data. Both simple cis-acting variation in promoter or
enhancer regions, as well as trans-acting variation can theoretically
detected \cite{pmid21838806}. To detect trans-acting variants,
however, might be impossible with the (for population studies)
relative low number of sequenced individuals, as it relies on
statistical associations requiring broad sampling. Yet, cis-acting
variation, more readly detectable as allele-specific variation, is
unlikely to explain variations in mitochondrial gene expressions for
the reasons discussed above.

Therefore large scale meta-population wide sampling must not be
limited to a evalutation of the divergent gene-exression phentypes,
but has to further elucidate the population genetic relationships
between Taiwanese and European worms. A future research program will
thus need to employ population-scale sampling of genotype data, densly
spread across the genome. Genotyping of many European
\textit{A. crassus} from different populations and comparison with
many individual genomes from different Asian populations will enable
tests for selection: based on the fact that around selected variants
nucleotide diversity is reduced by hitchhiking of neutral variation in
so called selective sweeps \cite{pmid16251466}, a punctual increase of
population differentiation measured by the fixation index F$_{st}$
\cite{wright1949genetical} in regions linked to selected variants can
be measured. Other well establshed population genetic measurements
include Tajima's D, a measure based on the allele frequency spectrum
\cite{pmid2513255}. When these methods are applied on a genome wide
scale the neutral null-expectation to seperate a loss in variability
based on selection from neutral loss due to demography is given by the
diversity across all regions of the genome. A microsatellite study
\cite{wielgoss_population_2008} as well as my own evaluations (based
on pyrosequencing see \ref{sing-w}) and RNAseq (data not shown)
indicate only a moderate genetic bottleneck caused by the introduction
of \textit{A. crassus} to Europe and thus the necessary neutral
diversity as a background for these tests will be present.

Furthermore statistical models need to be parameterised by divergence
time to disentange the influence of demography and selection (i.e. to
estimate the effective population size). Relieable estimates for
divergence time are readily available for the introduction of
\textit{A. crassus} to Europe: 60 to 90 generations. As for such a
short periode linkage to putatively selected variants will not be
borken down in large blocks, marker density is of minor concern, but
empathis should be on the breadth (many individuals from many
populations) of sampling.

One methods enabling such population wide genotyping emerging from NGS
technology is the sequencing of restriction-site associated DNA (RAD)
markers. Preparation of RAD libraries involves digestion of genomic
DNA with a restriction enzyme. Individually tagged adaptors can then
ligated to the fragments and individual samples for can be pooled. The
choice of restriction enzyme is important to optimise the number of
restricion sites (depth of sampling the genome) relative to the number
of individual samples being investigated \cite{pmid18852878}. In the
case of \textit{A. crassus} this optimisation also concerncs the
minimisation of restiction sites in host-genome, as present in
unavoidable contamination. The \textit{de novo} assembly of a
reference genome for \textit{A. carssus} will enable the search for
such an optimal restriction enzyme. Preliminary data has been
generated on one lane of the Illumina HiSeq machine giving 110 million
100 bases long paired-end reads, in total over 10 gigabases of
sequence data.

A preliminary assembly yielded a mean coverage of below 15-fold, for
the \textit{A. crassus} derived contigs. This coverage is surprisingly
low given the large amount of input-data and I will need to construct
improved assemblies informed by the analysis of this preliminary
assembly. A seemingly trivial but nevertheless important prerequisite
for any high-througput genomic sequencing project on a prarasite was
the confirmation that genomic DNA could be obtained sufficiently clean
from other xenobiont DNA.

\figuremacroW{genome_cov_gc}{GC-content and coverage for a prliminary
  genome assembly}{A preliminary assembly of routhly 10 Gb sequence
  data in over 110 million reads. The analysis of GC-content and
  coverge identifies host-cotamination at higher GC, but lower
  coverage. For this sequencing library only 10-20\% of read-data is
  lost to this kind of off-target data. The preliminary assembly was
  provided by Sujai Kumar from Mark Blaxter's lab.}{0.5}

It has been possible to isolate roughly 1$\mu$g of genomic DNA from
big specimen of the worm. Only ca. 20\% of the DNA were derived from
the genome of the eel-host (see figure \ref{genome_cov_gc}). As only
300 ng of DNA material (with low amounts of contamination with
host-blood) are needed for RAD-sequencing, this can be achived in most
big specimen of \textit{A. crassus}.

For both reference genome assembly and annotation and for the future
genome-scans I will continue to collaborate with Mark Blaxter's lab at
the University of Edinburgh. The group there is actively developing
methods expecially for RAD-sequencing and applying them to questions
in evolutionary model-species \cite{pmid21681211}.

Another useful strategy enabled by RAD-sequencing construction of a
physical genetic map in families of \textit{A. crassus} (backcross is
impossible). In addition to the population scale approaches outlined
above mapping of gene-expression quantitative trait loci (eQTL) in
mapping crosses between the two divergent expression-phenotypes
constitues a promising route for the investigation of genomic variants
underlying the divergent phenotypes. Once transcripts can be anchored
on genomic contigs and linkage groups can be constucted to build a
physical map of the genome, a readout for hybrid F2 individuals could
even be transcriptomic data (RNAseq) providing both genotype and
expression-phenotype.

A prime example for a research program on the evolution of
ecologically important traits is provided by the Stickleback
\textit{Gasterosteus aculeatus}: QTL-mapping has been performed to
fine-map the loss of lateral plates in freshwater populations
\cite{pmid18852878} and parallel adaptation has been investigated
using population genomics \cite{pmid20195501}. Both approaches used
RAD-sequencing. The sophistication and depht of insight available in
such a evolutionary model species is underlined by research on
adaptive reduction of pelvic structures, an evolutionary trajectory
shown to be feavored by the localisation of the underlying change in a
instable reagion of the genome \cite{pmid20007865}.

The hope to develop a similar research program based on the present
humble thesis seems presumptuous. Nevertheless, making full use of the
advances in sequencing technology it might be possible to rapidly gain
insight into the genomic organisation underlying contemporary
evolutionary change. The prestent RNAseq data will be crucial in
achiving this goal, as is will be used to link expression phenotypes
with genomic sequence. An evoltuionary leap in a core metabolic
process seems possible.

The ability to evolve via such a leap could even be a evolutionary old
trait retained in \textit{A. crassus} allowing it to colonise new
hosts. Therefore, comperative genomics relating population genetic
processes in \textit{A. crassus} to putatively adaptive change during
the accquision of new host by other Anguillicoloid species in
evoltuionary time constitues another route of research. If such a link
beetween microevolutionary processes in \textit{A. crassus} and the
evolution of \textit{Anguillicola}-species would exist, it would
provide general insight in the evolution of parasitic phenotypes.

% While this change is due to a cis-acting enhancer
% element upstream of the \textit{Pitx1} gene, in my case of the
% metabolic change in \textit{A. crassus} the underlying change is
% expected to be rather trans-acting.

% Not only finding phenotypes in Taiwan representing the low-frequency
% standing variants selection might have acted on is desireable, but
% also European indviduals or populations not displaying the divergent
% (possibly selected) phenotyps would be advantageous.

% Mark Blaxter and Christoph Dieterich (see refs RAD
% \cite{pmid21681211} Pp genome \cite{pmid18806794}).

% If indeed all \textit{A. crassus} found in Asia would express
% respiratory chain enzymes at a much higher level, the populations
% gentetic scans would implicitly equal genome-wide association studies
% (GWAS).


%%% Local Variables: ***
%%% mode:latex ***
%%% TeX-master: "../thesis.tex"  ***
%%% tex-main-file: "../thesis.tex" ***
%%% End: ***
