% this file is called up by thesis.tex
% content in this file will be fed into the main document

\chapter{Discussion} % top level followed by section, subsection


% ----------------------- paths to graphics ------------------------

% change according to folder and file names
\ifpdf
    \graphicspath{{7_discussion/figures/PNG/}{7_discussion/figures/PDF/}{7_discussion/figures/}}
\else
    \graphicspath{{7_discussion/figures/EPS/}{7_discussion/figures/}}
\fi


% ----------------------- contents from here ------------------------


\section{Pilot-sequencing}
\label{sec:sanger-pil}

In was not achieved to reproducibly alleviate the rRNA-levels in
libraries prepared for sequencing. This has probably been due to the
fact that extraction of total-RNA from worms filled with host blood
resulted in low amounts of starting material, and reaction conditions
did not allow specific amplification of mRNA from a rRNA
background. As the same problems existed in preparation of liver
tissue of the host species it seems likely that the blood of eels
contains substances limiting the success of specific amplification
protocols. In fact it is known that compounds like hemoglobin can
inhibit PCR reactions \cite{pmid9327537} and reverse transcription
\cite{pmid16109794}.

Nevertheless the stringent quality trimming and processing of raw
reads, as summarized in \ref{pilot-seq}, made the remaining ESTs a
valuable resource for comparison with future 454-sequencing-data.

In fact all sequenced ESTs, for which host-orgin was inferrred were
later found also in pyrosequencing: The observation hemoglobin and
ferritin subunits form \textit{An. anguilla} are expected, as fish
erythrocytes contain a nucleus and still transcribe genes actively
\cite{pmid20614118}. These are typical proteins for the functioning of
red blood cells. The overservation of fish cyclin G1 (in Sanger- and
pyrosequencing) and cohesin (in Sanger-sequencing) is remarkable as
fish erythrocytes are thought to exhibit low rates of mitosis
\cite{pmid7506383}. Other obersvation of host-sequences like
e.g. Leukocyte cell-derived chemotaxin 2 or natural killer
cell-enhancing factor (NKEF)-B protein in pyrosequencing make an
analysis of this fish-derived off-target data (form all sequencing
technologies) very promising, it is howerver beyond the scope of the
present thesis.

\section{Pyrosequencing}
\label{sec:454-pyr}

We have generated a de novo transcriptome for \textit{A. crassus} an
important invasive parasite that threatens wild stocks of the European
eel \textit{An. anguilla}. These data enable a broad spectrum of
molecular research on this ecologically and economically important
parasite. As \textit{A. crassus} lives in close association with its
host, we have used exhaustive filtering to attempt to remove all
host-derived, and host-associated organism-derived contamination from
the data. To do this we have also generated a transcriptome dataset
from the definitive host \textit{An. japonica}. The non-nematode,
non-eel data identified, particularly in the L2 sample, showed highest
identity to flagellate protists, which may have been parasitising the
eel (or the nematode). Encapsulated objects observed in eel swim
bladder walls \cite{heitlinger_massive_2009} could be due solely to
immune attrition of \textit{A. crassus} larvae or to other
coinfections.

A second examination of sequence origin was performed after assembly,
employing higher stringency cutoffs. Similar taxonomic screening was
used in a garter snake transcriptome project \cite{pmid21138572}, and
an analysis of lake sturgeon tested and rejected hypotheses of
horizontal gene-transfer when xenobiont sequences were identified
\cite{pmid20386959}. A custom pipeline for transcriptome assembly from
pyrosequencing reads \cite{pmid20034392} proposed the use of EST3
\cite{pmid17218127} to infer sequence origin based simply on
nucleotide frequency. We were not able to use this approach
successfully, probably due to the fact that xenobiont sequences in our
data set derive from multiple sources with different GC content and
codon usage.

Compared to other NGS transcriptome sequencing projects
\cite{pmid20478048}, the combined assembly approach (see
\ref{sec:over-eval}) generated a smaller number of contigs that had
lower redundancy and higher completeness. Projects using the mira
assembler often report substantially greater numbers of contigs for
datasets of similar size (see e.g. \cite{pmid21364769}), comparable to
the mira sub-assembly in our approach. The use of oligo(dT) to capture
mRNAs probably explains the bias towards 3' end completeness and a
relative lack of true initiation codons in our protein
prediction. This bias is near-ubiquitous in deep transcriptome
sequencing projects (e.g. \cite{pmid20331785}).

We were able to obtain high-quality annotations for a large set of
TUGs: For 40\% of the complete assembly and 60\% of our highCA
assembly \texttt{BLAST}-based annotations could be obtained. 45\% of
the contigs in the highCA assembly were additionally decorated with
domain-based annotations through \texttt{InterProScan}
\cite{pmid11590104}.

Comparison with complete protein sequence from the genomes of
\textit{B. malayi} and \textit{C. elegans} showed a remarkable degree
of agreement regarding the occurrence of terms in the two parasitic
worms. This agreement was higher than with the free living nematode
\textit{C. elegans} and even the two genome-sequencing-derived
proteomes showed less agreement with each other than the filarial
parasite with our dataset. This implies that our transcriptome is
truly a representative partial genome
\cite{parkinson_partigene--constructing_2004} of a parasitic nematode.

Analysis of conservation identified more sequence novel in Nematode
than in the eukaryote kingdom or in clade III this is in agreement
with prevalence of genic novelty in the Nematoda
\cite{wasmuth_extent_2008}. Furthermore the basal position of
\textit{A. crassus} in clade III could be leading to most novelty in
the clade not being shared with \textit{A. crassus}.

TUGs predicted to be novel in the phylum Nematoda and novel to
\textit{A. crassus} contained the highest proportion of
signal-positives. This confirms observations made in a study on
\textit{Nippostrongylus brasiliensis} \cite{harcus_signal_2004}, where
signal positives were reported as less conserved. Interestingly
enrichment of signal sequence bearing TUGs in our dataset was
constrained to sequences novel in nematodes and \textit{A. crassus}
(i.e. not to the level of clade III). This may be explained, whit two
different hypotheses involving the basal position of
\textit{A. crassus}: First the signal positives shared with all
nematodes could be conserved molecules not excreted by parasites. A
different class of secreted/excreted molecules with prominent role in
host parasite interactions would not have arisen early in the
evolution of parasitism in clade III - or be too fast-evolving - and
thus be detected as specific to deeper sub-clades (i.e. to
\textit{A. crassus} in our dataset). A second explanation would be,
that orthologs of excreted parasite-specific genes could be among
those shared with other nematodes and the fewer shared with clade III
implying a predisposition to parasitism outside of the Spirurina or
even the convergent evolution of secreted molecules in other parasitic
nematodes. However analysis of dn/ds (see below) across conservation
categories favours the first hypothesis, as it identifies a higher
amount of positive selection in TUGs novel to clade III and
\textit{A. crassus} than to nematodes.

We generated transcriptome data from multiple \textit{A. crassus} of
Taiwanese and European origin, and identified SNPs both within and
between populations. Screening of SNPs in or adjacent to homopolymer
regions improved overall measurements of SNP quality. The ratio of
transitions to transversions (ti/tv) increased. Such an increase is
explained by the removal of ``noise'' associated with common homopolymer
errors \cite{pmid21685085}. The value of 1.93 (1.25 outside, 2.41
inside ORFs) is in good agreement with the overall ti/tv of humans
(2.16 \cite{pmid21169219}) or \textit{Drosophila} (2.07
\cite{pmid21143862}). The ratio of non-synonymous SNPs per
non-synonymous site to synonymous SNPs per synonymous site (dn/ds)
decreased with removal of SNPs adjacent to homopolymer regions from
0.42 to 0.231 after full screening. The most plausible explanation is
the removal of error, as unbiased error would lead to a dn/ds of
1. While dn/ds is not unproblematic to interpret within populations
\cite{pmid19081788}, the assumption of negative (purifying) selection
on most protein-coding genes makes lower mean values seem more
plausible. We used a threshold value for the minority allele of 7\%
for exclusion of SNPs, based on an estimate that approximately 10
haploid equivalents were sampled (5 individual worms plus an
negligible contribution from L2 larvae in the L2 library and within
the female adult worms). The benefit of this screening was mainly a
reduction of non-synonymous SNPs in high coverage contigs, and a
removal of the dependence of dn/ds on coverage. Working with an
estimate of dn/ds independent of coverage, efforts to control for
sampling biased by depth (i.e. coverage; see \cite{pmid18590545} and
\cite{pmid20478048}) could be avoided.

Also in comparison with published intra-species values of dn/ds our
final estimate of seems plausible: in transcripts from the female
reproductive tract of \textit{Drosophila} dn/ds was 0.15
\cite{pmid15579698} and 0.21 in the male reproductive tract
\cite{pmid11404480} (although for ESTs specific to the male accessory
gland were shown to have a higher dn/ds of 0.47). A pyrosequencing
study in the parasitic nematode \textit{Ancylostoma canium}
\cite{pmid20470405} reported dn/ds of 0.3.

When the whole of coding sequences are studied, of which only a small
subset of sites can be under diversifying selection, dn/ds of ~0.5 has
been suggested as threshold for assuming diversifying selection
\cite{pmid15579698} instead of the classical threshold of 1
\cite{pmid6449605}. The use of this threshold for positive selection
led to the identification of over-represented of GO-term highlighting
very interesting transcripts:

13 peptidases under positive selection (from 43 with a dn/ds obtained)
meant an enrichment in the category. All 13 have different orthologs
in \textit{B. malayi} and \textit{C. elgans} and are conserved across
kingdoms. Despite their conservation peptidases are thought to have
acquired new and prominent roles in host-parasite interaction compared
to free living organisms: In \textit{A. crassus} a trypsin-like
proteinase has been identified thought to be utilised by the
tissue-dwelling L3 stage to penetrate host tissue and an aspartyl
proteinase thought to be a digestive enzyme in adults
\cite{polzer_identification_1993}. The 13 proteinases under positive
selection could be the targets of the adaptive immunity developed
against \textit{A. crassus} \cite{knopf_migratory_2008,
  knopf_vaccination_2008}, which is often only elicited against
subtypes of larvae \cite{molnar_caps}.

The under-representation of ribosomal proteins (term ``structural
constituent of ribosome'') in positive selected contigs is in good
agreement with the notion that ribosomal proteins are extremely
conserved across kingdoms \cite{pmid9664699} and should be under under
strong negative selection.

Genotyping of individual worms identified a set of 199
SNPs with highest credibility and a high information content for
population-genetic studies. Levels of genome-wide heterozygosity found
for the 5 adult worms examined in our study are in agreement with
microsattelite data \cite{wielgoss_population_2008} showing reduced
heterozygosity in European populations of \textit{A. crassus}.

We were able to use the DESeq \cite{pmid20979621} to report
transcripts significantly differing in expression between male and
female worms. This was possible for male worms, despite the fact that
no replicated samples were obtained. However only over-expression in
the non-repeated samples could be detected, as obviously lack of
expression in one sample can't statistically validate
under-expression. Genes over-expressed in male \textit{A. crassus}
comprise major sperm proteins well known for their high expression in
nematode sperm \cite{pmid15275275}.

We developed a method to assess the possible influence of
fragmentation of our reference-transcriptome on mapping. Using the
annotation with orthologous sequences of \textit{C. elegans} and
\textit{B. malayi} it is possible to highlight problems: For example a
contig annotated as ``Phosphoenolpyruvate carboxykinase'' showed
significant over-expression in the male. The 8 other contigs with this
annotation however, were nearly identical for the middle region of
their predicted proteins, had a high amount of homopolymeric regions in
their nucleotide sequence and attracted the missing reads mapping from
the other libraries. Our orthologous-collapsing method for read-counts
following manual inspection, could thus exclude this contig as a
false-positive.

In the comparison of European with Asian libraries most
differentially expressed contigs could be debunked as false positives
with this method. From the 6 remaining contigs one was annotated with
only one ortholog (`Contig200''), 3 were lacking any annotation
(``Contig4'', ``Contig5311'' and ``Contig40''). The only contig with
differential expression fully validated by our method (``Contig6355'')
was annotated as (``Trypsin family protein''). The identification of
another proteinase (it is not among those under positive selection) is
highlighting the interesting nature of these molecules in
\textit{A. crassus}.

Analysis of the expression data made clear what data is required to
analyse the differences in European vs. Asian nematodes: As variance
is high in outbred individuals from natural populations both
replication and depth of analysis have to be high. Furthermore it is
important to disentangle the influence of the host and the nematode
population e.g. in a co-inoculation experiment. The last approach
would also allow the analysis of transcriptomes roughly synchronised
for the time of development.

The \textit{A. crassus} transcriptome provides a basis of molecular
research on this important species. It further provides insight in
the evolution of parasitism complementing the catalogue of available
transcriptomic data with a member of the Spirurina phylogenetically
distant to so far sequenced parasites in this clade.

\section{Transcriptomic divergence in a common garden experiment}
\label{sec:exp-inf}

Whith some reservations discussed below our observation of higher
recovery of adult worms from locally matching
\textit{A. crassus}-\textit{Anguilla} spp. host-parastite combinations
imply local adaptation of different worm populations to host-species.

Our results regarding recovery are not in complete agreement with
previouse findings by Weclawski et al. (unpublished; see
\ref{div-ac}). They recorded recovery at only slightly different
timepoints after infection (25, 50 and 100 dpi) and our decission to
record recovery at 55-57 dpi was based on their evaluation.  Similar
to our sutdy they found a higher recovery of the European population
of worms in the European eel but did not find the complementary result
of lower recovery of this diverged population in the Japanese eel. A
possible explanation for these different results are interactions of
host-parsite genotypes conditional on the environment (GxGxE
interactions, sse also \ref{div-ac}). It is imaginable that this
environment provided in our common-garden setting slighly differed
between the two experiments.

It has to be emphasized that the observations made in common-garden
experiments first and foremost have to be interpreted as
phenotypes. An ideally suited phenotype to infer local adaptation
would be one with obviouse direct fitness-consequences, a so called
fitness-component. Such a fitness-component would ideally be a
measurement on a single individual and individual life-time
reporductive success would be such an ideal measurement. However the
techniques to measure individual life-time reporductive success have
not been established in \textit{A. crassus} and it would be very
difficult to do so.

The recovery of certain developmental stages of worms is another proxy
interpretable as a fitness-component. It is a composite measurement
the speed of development from previouse lifecycle stages (or speed of
migration towards the swimbladder) and of survival. While suvival is
surely an important component of fitness, it is not completely clear
whether fast development and/or migration to the swimbladder are. It
is immaginable that under certain conditions slower development could
lead to higher fitness, if it would e.g. allow to develop without
attracting the attention of the immune system.

Another slight problem with recovery in these experiments is that it
is a mean measurement over many individuals. If one would want to find
genotype associations with the most suboptiomal phenotype it would not
be possible to isolate individuals bearing this trait, because these
would be dead or still on their way migrating to the swimbladder.

We decided on a study design usign pools of individuals for one sex
(males) and single individuals for the other. A studies on
\textit{Fundulus heteroclitus} revealed that approximately 18\% of the
genome is differentially expressed in individual fish from the same
population, grown under controlled environmental conditions
\cite{pmid12219088}. And it thus not surprising, that idividual
variation in female samples was leading to higher variance of these
female samples compared to male samples in our study.

Interindividual variation in the abundance of a mRNA molecules (or
gene-exprssion in it's synonymous meaning) under a particular
environmental condition is generally agreed to be closely linked to a
genetics basis \cite{pmid15498452}. For exapmle in a cross between two
parental strains of yeast the genetic component of variation was
estimated from haploid segregants to be 84\% \cite{pmid11923494}. The
genetic component was found to be the main factor determing expression
level variability between two strains, sexes and ages of
\textit{Drosophila melanogaster} for 267 (7\%) from 3931 genes and at
least 25\% of the transcriptome were estimateded to be affected mainly
by genotypic factors in any of the groups
\cite{pmid11726925}. Variation in the regulation of gene-expression is
thought to constitute as a major source of evolutionary novelty
\cite{pmid11341673}.

A second study from the line of research on \textit{Fundulus
  heteroclitus} \cite{pmid16567645} used genetic relatedness as
inferred form phylogenetics to seperate variation of gene-expression
in a common experimental environment into a neutral component and a
selected component i.e. they removed variation most likely accounted
for by the shared neutral evolutionary history. Our case of
\textit{A. crassus} is simpler: The investigated European population
is a direct descendant and thus a subset of the Taiwanese source
population. In fact we studied two European populations, as a few
hundred kilometers between the geographical origins of the two
different locations in Germany probably constitute a barrier to
gene-flow in a parastite with an aquatic intermediate host. We however
treated the two European populations as replicates (and use the
terminology of one European and one Asian population throughout the
text) with the reationale of incrasing variance for random genetic
differences and raising the bar for poptentially adaptive differences
to be detected.

When we later venture into adaptive interpreatations of the observed
gene-expression differences it has to be remembered however, that
these constitute nothing more but a molecular phenotype. This
phenotype is not necessarily a fitness-component. It is one of the
dangers of genomic data to forget the fundamental lession from the
dabate initiated by Gould and Lewontin in 1979
\cite{gould_spandrels_1979}. Briefly, while functional changes are
often caused by selection, differences in function do not necessarily
demonstrate the past or present action of selection. There is no way
to infere the action of selection based on functional considerations,
and even if selection can be inferred otherwise, it is not necessarily
a certain observed differing trait that selection acted on
\cite{pmid19744124}.

No surprise was the abundance of differential expression between male
and female worms of roughly one third of genes. A lage number of genes
are known to be sex-specific, regulating ovulation and spermatogenesis
throughout the metazoa and especially in nematodes
\cite{pmid15371532}. On top of these sex-specific genes there is large
number of genes differently expressed due to differences in
metablolism between males and females. Estimates for
\textit{Drosophila} range between one- and two-thirds of the
transcriptome showing sex-biased expression \cite{pmid11726925}. In
the liver transcriptome of \textit{Mus musculus} even 70\% of
transcripts have shown to differ between sexes \cite{pmid16825664}
(note however, that this study used 169 female and 165 male mice to
guarantee the finding of even the most subtle differences). Given the
scale of these differences in other species our estimate of roughly a
third of the transcripts in \textit{A. crassus} showing differential
expression according to the sex of the worms implies conservative
thresholds used in the statistical analysis.

Nearly the same proportion (roughly 30\%) of contigs was confirmed
through summation and analysis of contigs for orthologs in
\textit{B. malayi} and \textit{C. elegans}. Development of this
orthologous confirmation method was neccesitated by the possibly
fragmented and chimeric transcriptome assembly. It introduces
stringent conditions for the detection of significance, as p-value
correction for multiple testing is employed during each analysis (once
for raw counts and twice for orhtologous counts). Although the
underlying tests are surley not independent, the false discovery rate
of 5\% for raw contigs can be expected to be immensly lowered by
applying an FDR of 10\% twice.

Also biological implication could produce false negatives in such an
evaluation: All genes duplicated in \textit{A. crassus} (a) and
follwing antithetic expression patterns will be evalutated negatively
as well as duplicated genes in any of the model-species (b) following
such a pattern. However there is no other choice then applying these
stringent conditions to screen for artefacts producing the same
patterns based on mapping to fragmented (a) or chimeric (b) reference
contigs. We think that an evaluation based on this scrutinised
conficence in an assembly previously computed from 454-data is even
more appropriate then an analysis solely based on counts collapsed for
orhtologs excluding only possible fragmentation artifacts (as used
eg. in \cite{pmid22084086}).

In general, our statistical analysis aimed to minimize false positives
(type I error) at expenses of possible false negatives (type II error)
and is thus not fully suited to address the proportions of
differentially regulated genes.

Nevertheless it is surprising, that less than 1\% of transcripts were
detected differentially expressed between worms in different
host-species and less then 0.3\% were confirmed with the orthologous
method. This was an unexpected finding, as the different immune
resopnses of the host species have a big influence on other phenotypes
of worms \cite{knopf_swimbladder_2006}. In addition to the low number
of genes, multifactorial analysis revealed below 10\% of the variance
to be explained by host-species effect, even in significantly
differential regulated genes for this factor.

Although these differences between worms in different host species
were the most marginal of any of the factors, it is possible to
connect some (at least two) of the genes to a prominent physiological
difference: The digestion of hemoglobin. Two different aspartic
proteases (both confirmed thourough orthologs, one of them differing
for all three main effects, the other for an interaction of worm-sex
and host species) known to be involved in the first steps of digestion
of hemoglobin from other nematodes \cite{pmid12782060} were
overexpressed in worms in \textit{An. anguilla}. This expression
phenotype could potentially be linked to the often observed phenotye
of bigger size of \textit{A. crassus} in this host
\cite{knopf_swimbladder_2006}, as the main contribution to this
increase in size is the larger volume of host-blood taken up by the
parasite. Accordingly the parasite does probably digest hemoglobin at
a higher rate.

Close to 1\% of contigs were significantly differeing in expression
between European and Asian \textit{A. crassus}, making this difference
significant for a higher number of contigs than the
host-differences. For this contrast also the proportion of orhtolgous
confirmation was lower than for sex differences but higher than for
host differences. Additionally multivariate analysis of all
differently expressed transcripts for worm-population revealed that
the variance contributed by the population-factor was higher than 10\%
or even 20\% for orthologous confirmed contigs.

Another important finding was the large overlap in contigs expressed
differentially depending on worm-sex and worm-population. Such an
overlap is expected if genes expressed differentially according to sex
are evolving faster towards a differential expression according to
other factors. Faster evolution of reproductive (and especially male
specific traits) have been shown in many species at a phenotypic and
at a sequence level \cite{pmid15795858}. In \textit{Drosophila} male
reproductive proteins have been shown to evolve at elevated levels and
under positive selection \cite{pmid11404480}. In \textit{C. elegans}
genes expressed after reproductive maturity evolve faster than genes
expressed earlier in development \cite{pmid15371532}, suggesting a
model of elevated pleiotropic effects in genes expressed at earlier
stages of development. And also gene expression should evolve at a
higher rate in sex-specific genes, indeed the transcriptomes of
\textit{Drosophila} species show that interspecific expression
divergence is sex dependent and the action of sex-dependent natural
selection during species divergence has been inferrred from this
\cite{pmid15034135,pmid19720861}.

Taken together our findings strongly support a larger influence of the
genetic differences between European and Asian populations of
\textit{A. crassus} than of the modification in the different
host-species on gene-expression. When additive and interaction effects
are considered, the influence of host-species even vanishes completely
in favour of a combination of effects combining parasite population
and sex of the worms.

From a functional perspective genes identified to differ between
popualatins can be rather categorized as important in general
metabolic processes instead of specific host-parasite intereaction.
This constitutes a negative evalutation of one of our \textit{a
  priori} hypotheses to find parasite-specific genes, identified as
vaccine candidates in a number of nematodes, within the genes modified
or diverged in our study (\ref{sec:dna-sequ-nemat}).

\figuremacro{tGO_pop_MF_classic_10_all}{GO molecular function graph
  for enriched terms in DE according to worm-population}{Subgraph of
  the GO-ontology molecular function category induced by the top 10
  terms identified as enriched in DE genes between different parasite
  populations. Boxes indicate the 10 most significant terms. Box color
  represents the relative significance, ranging from dark red (most
  significant) to light yellow (least significant). In each node the
  categoy-identifier, a (eventualy truncated) description of the term,
  the significance for enrichment and the number of DE / total number
  of annotated gene is given. Black arrows indicate a is ``is-a''
  relationship.}

Instead especially enzymes ane enzyme subunits important for aerobic
respiration are expressed at lower levels in European
\textit{A. crassus}. In fact most transcritpst significantly differing
between populations were annotated as ``oxidoreductase'' in
gene-ontology. While at face values of single genes have to be
interpreted with care such trends in the data are very likely to point
to improtant differences.

Downregulatin of Cytochrome C oxidase subunit 2 (COXII) in the
European population of \textit{A. crassus} was the most persistent
finding. Cyotchorme C oxidase subunit 1-3 are are essential components
of complex IV, the cytochrome c oxidase. They are encoded in the
mitochondrial genenome and coordinate catalytic heme and copper
cofactors.

In fact not only enrichment anylysis highlighted oxidoreductases, but
expression values of COX II clustered with other metabollically
interesting enzymes: Two lecitin:cholesterol acyltransferase
transcipts are putative recently duplicated genes. They were showing
slighly divergent protein sequences but hit the same orthologs in
\textit{C. elegans} and \textit{B. malayi}. They were also sharing
very similar expression profiles. Expression of different colesterol
acyltransferases has been shown to vary in response of to the presence
of heme and anaerobiosis in yeast
\cite{pmid11786267}. 3-hydroxyacyl-CoA dehydrogenase (involved
fatty-acid $\beta$-oxidation \cite{pmid8454629}), Malate/L-lactate
dehydrogenase (involved in the citric acid cycle
\cite{sturm1969vergleichende}) and aspartyl proteases (involved in the
digestion of host hemoglobin in helminths \cite{pmid12782060})
completed this particular cluster.

The pattern observed for the metablolic enzymes would be interpretable
as a change to use more anaerobic metabolism in the European
population of \textit{A. crassus}. In one possible scenario in Europan
worms one of the core enzymes of the respiratory chain would have
evolved a genetically fixed lower level of expression. This
interpretation follows the logic, that the most differentiall
expressed gene could be the driver of observed change. Other enzymes
(e.g. in lipid metabolism) indirectly and partially controled by
feedback mechanisms from the respiratory chain would show similar
patterns of altered expression in European worms. However the
expression of some these indirectly and also by more additional
environmenal factors controlled genes would be perturbed when worms
are applied back to their Asian host.

\figuremacro{pop_ortho_heat}{Heatmap for OC contigs DE between worm
  populations}{Orthologous confirmed contigs, differentially expressed
  between the European and Asian populations of
  \textit{A. crassus}. Expression levels are clustering mainly
  according to sex of the worm.}

Such a scenario also provokes some speculation about the adaptive
value of such a change in a core metabolic process:\\
Aerobic respiration is potential source for oxidative stress providing
a steady source of reactive oxigen species (ROS) as electrons are
leaking from the respiratory chain as superoxide anions. It is well
established that such ROS production is especially harmful to
blood-feeding parasites, as free inorganic iron as well as heme have
the potential to generate additional ROS
\cite{pmid21087517}. Anaerobic metabolism is thus thought to occur in
many haematophagous parastites as a counter-measure against oxidative
stress from hemoglobin catabolism \cite{pmid12163151}. It could thus
be hypothesized that the bigger size and the larger amount of
eel-blood ingested leadign to a higher rate of hemoglobin digestion
provided the selective pressure to reduce aerobic respiration.

Additionally helminths can simply get too large to maintain oxygen
diffusion to mitochondriae in the abscence of a cardiovascular
system. As yet proton-pumping electron transport constitutes the most
profitable metabolic process, the mitochondria of anaerobic helminths
produce and electorn gradient for the use of ATPase with the help of
acceptors other than O$_2$ \cite{pmid12417132}. Such an alternate
electron sink is fumarate used in many helminths in a process called
malat dismutation \cite{pmid15275412}.

A second group of genes differentially expresse in populations of
\textit{A. crassus} emerged from both cluster and enrichment
analyses. Two transcripts in this cluster were significant for
interaction effects between host-speacies and parasite-population,
they annotated as collagens. For both genes this meant an
``adujusted'' (to avoid the suggestive ``adapted'') expression
difference leading to a lower experession in
host-species/parasite-population pairs found in nature. Cuticle
collagens are a large multigene family of proteins (Interpro lists 164
entries for ``Nematode cuticle collagen, N-terminal''
(\href{http://www.ebi.ac.uk/interpro/ISpy?ipr=IPR002486&tax=6239}{\textit{C. elegans}
}) and 51 for
\href{http://www.ebi.ac.uk/interpro/ISpy?ipr=IPR002486&tax=6279}{\textit{B. malayi}}),
additionally they contain extensive repeat regions. In the genome of
\textit{B. malayi} 82 genes that encode for a collagen repeat have
been found \cite{ghedin_draft_2007}. It was thus very important to
have orhtologous confirmation for these two contigs, as misassembly
could have easiesly lead false positives here.

The two collagens were clustered with a third contig sharing a
collagen-annotation and (failing to be significant for the interaction
term probably because of low overall expression), a contig annotated
as Matrixin (an enzyme assumed to be involved in remodelig the
extracellular matrix \cite{mealloprot}) and a ABC-transporter family
protein. This can be reagarded a confirmation via clutering with
funcionally related proteins.

Functional speculations are more difficult for collagen, than for the
respiratory chain enzymes. The cuticle constitutes an exoskeleton and
a barrier between the worm and its host-environment.

Synthesis of most collagens is beleived to occur at negligible levels
in adult male worms. 

% CHANGE
% four larval stages before moulting5. During each synthesis, different
% collagen genes are expressed in discrete temporal periods,
% CHANGE

\section{Outlook}

Population genomics


%%% Local Variables: ***
%%% mode:latex ***
%%% TeX-master: "../thesis.tex"  ***
%%% tex-main-file: "../thesis.tex" ***
%%% End: ***
