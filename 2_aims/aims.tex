% this file is called up by thesis.tex
% content in this file will be fed into the main document

\chapter{Aims of the project} % top level followed by section, subsection

% ----------------------- contents from here ------------------------


\section{Preliminary aims}

In order to investigate the response of the transcriptome to
environmental stimuli or alternatively a genetic fixation of such a
response, the responding units (transcripts) had to be established
first. Ensuring the quality of these computationally constructed
transcript-models (contigs) and screening for host- and other
xenobiont-derived sequences were central aims of this preparatory part
of the project. These goals were pursued using bioinformatic analysis
of Sanger- and pyro-sequencing data, whith the aim guarantee relieable
inferrence based on this reference.

\section{Final aim}

Not only gene-expression sudies were enabled based on the sequence of
this reference transcriptome but also questions could be adressed
regarding general aspects of evolutionary biology of
\textit{A. crassus}. Aims addressble at the sequence-level were the
characterisation of the transcriptome in relations to related
parasitic nematodes and inferrence of positive selection using data on
polymorphism.

The genetic component of expression differences was then elucidated in
reciprocal transplant experiments. As final aim of these experiments
the relative contributions of physiological plasticity of
gene-expression versus rapid, heritable, evolutionary change should be
illuminated. If present, divergent expression phenotypes between
European and Asian populations should be found.



% ---------------------------------------------------------------------------
% ----------------------- end of thesis sub-document ------------------------
% ---------------------------------------------------------------------------

%%% Local Variables: ***
%%% mode:latex ***
%%% TeX-master: "../thesis.tex"  ***
%%% tex-main-file: "../thesis.tex" ***
%%% End: ***
     