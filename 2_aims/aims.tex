% this file is called up by thesis.tex
% content in this file will be fed into the main document

\chapter{Aims of the project} % top level followed by section, subsection

% ----------------------- contents from here ------------------------


\section{Preliminary aims}

In order to investigate transcriptomic response to environmental
stimuli or genetic differences, the responding unit, the transcripts
had to be established first to ensure relieable inferrence based on
this reference. Ensuring the quality of the computions constructing
transcripts (contigs) and screening for host- and other
xenobiont-derived sequences were central aims of this part of the
project. These goals were pursued using bioinformatic analysis of
Sanger- and primarily pyrosequencing data.

\section{Final aim}

Not only gene-expression sudies were enabled based on the sequence of
this reference transcriptome but also questions could be adressed
regarding general aspects of evolutionary biology of
\textit{A. crassus}. Aims addressble at the sequence-level were the
characterisation of the transcriptome in relations to related
parasitic nematodes and inferrence of positive selection using data on
polymorphism.

The genetic component of expression differences was then elucidated in
reciprocal transplant experiments.



% ---------------------------------------------------------------------------
% ----------------------- end of thesis sub-document ------------------------
% ---------------------------------------------------------------------------

%%% Local Variables: ***
%%% mode:latex ***
%%% TeX-master: "../thesis.tex"  ***
%%% tex-main-file: "../thesis.tex" ***
%%% End: ***
     