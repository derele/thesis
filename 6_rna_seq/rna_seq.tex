% this file is called up by thesis.tex
% content in this file will be fed into the main document

%: ----------------------- name of chapter  -------------------------
\chapter{Transcriptomic divergence in a common garden experiment}
% top level followed by section, subsection
\label{cha:tra-diff}

%: ----------------------- paths to graphics ------------------------

% change according to folder and file names
\ifpdf
    \graphicspath{{6_rna_seq/figures/PNG/}{6_rna_seq/figures/PDF/}{6_rna_seq/figures/}}
\else
    \graphicspath{{6_rna_seq/figures/EPS/}{6_rna_seq/figures/}}
\fi


%: ----------------------- contents from here ------------------------

\section{Overview}


\section{Infection experiments}

Dissection of eels 55-57 after infection (dpi) showed higher recovery
of European worms in \textit{An. anguilla} and higher recovery of
Taiwanese worms in \textit{An. japonica}, compared to the other
parasite populations. In other words, in host-parasite combinations of
matching origin, more parasites were recovered.

\figuremacro{stages_plot}{Recovery of worms in coinoculation
  experiment}{Mean numbers of worms recovered after 55-57 dpi for
  sample sizes given as n=x. Error-bars indicate the standard error
  (s.e.) of the mean. Recovered lifecycle stages of the parasite are
  listed separately as L3-larvae (l3), L4-larvae (l4), adult females
  (adult.f) and adult males (adult.m).}

In the host-species/parasite-population pairs found in nature roughly
eight or nine adult worms could be recovered per eel. In the
transplanted host/parasite combinations only two or three adult worms
were recovered on average (see figure \ref{stages_plot}). In
\textit{An. anguilla} no differences in the recovery of larval stages
was recorded. In \textit{An. japonica} however, roughly two
individuals more were recorded from both larval stages in the
host/parasite combination found in nature.

Recovery as a proportion of the 50 larvae eels were inoculated with,
was thus roughly 30\% for the adapted pairs compared to only roughly
10\% in non-adapted host-parasite pairs.

These differences are highly significant especially for adult worms
(see table \ref{tab:ad-sig}) and are interpretable as a sign of local
adaptation, as adult survival and recovery can be regarded as a
fitness component.


\begin{table}[h]
\begin{center}
\begin{tabular}{rrrrr}
  \hline
 & Estimate & Std. Error & t value & Pr($>$$|$t$|$) \\ 
  \hline
  (Intercept) & 9.5000 & 1.1109 & 8.55 & 0.0000 \\ 
  host.spec.AJ & -8.0789 & 1.7472 & -4.62 & 0.0000 \\ 
  worm.pop.T & -5.2222 & 1.3689 & -3.81 & 0.0002 \\ 
  host.spec.AJ:worm.pop.T & 11.7345 & 2.2010 & 5.33 & 0.0000 \\ 
   \hline
\end{tabular}
\caption[Linear model for recovery]{Linear model for recovery of adult
  worms. The estimate gives the mean of the distribution of adult
  worms for the factor values in the rows. The intercept is set to
  "Aa. R" (\textit{An. anguilla} and the European populations) further
  rows give variations for each factor. Std. Error is the standard
  error of this value. Additionally the probability of a t-value as
  small or smaller than the observed t-value are given. The signature
  of local adaptation is visible in the highly significant interaction
  term.}
\label{tab:ad-sig}
\end{center}
\end{table}

\afterpage{\clearpage}




\section{Sample preparation and sequencing }

Three biological replicates were obtained from each of the two worm
populations in each of the two eel-host species for each of the two
sexes of worms. This resulted in a total of 24 RNA-extractions
prepared for sequencing: 3 individual female worms from each
experimental group were chosen randomly to give in total twelve
females. Additionally, from three individual male worms, and from 9
pools of male worms RNA was extracted (see table
\ref{tab:lib-prep}). Pools consisted of worms from one infected eel
individual each. All worms or worm-pools were derived from infections
of different eel individuals, with one small exception from this form
of statistical independence: from eel AJ/R3 a male worms as well as a
female worm had to be prepared. It was impossible to extract enough
RNA from all but the biggest male worms especially of the Japanese
eel/European worm combination, leaving no other choice. Because of the
small size of male worms it was generally not possible to randomly
choose individuals. Preparation of sufficient amounts of RNA was only
achieved in pools of the biggest individuals. All male worms were thus
chosen for preparation based on large size, even when pools of worms
were used.

% latex table generated in R 2.13.0 by xtable 1.5-6 package
% Tue Nov 29 10:05:46 2011
\begin{table}[h]
\begin{center}
\begin{tabular}{llllrrr}
  \hline
  label & sex & host & population & intensity & worms in prep & conc in prep \\ 
  \hline
  AA/T20F & female & \textit{An. anguilla} & Taiwan (K) & 1 & 1 & 5.60 \\ 
  AA/T12F & female & \textit{An. anguilla} & Taiwan (K) & 14 & 1 & 6.80 \\ 
  AA/T45F & female & \textit{An. anguilla} & Taiwan (Y) & 5 & 1 & 8.00 \\ 
  AA/T24M & male & \textit{An. anguilla} & Taiwan (K) & 6 & 3 & 4.80 \\ 
  AA/T42M & male & \textit{An. anguilla} & Taiwan (Y) & 11 & 1 & 5.60 \\ 
  AA/T3M & male & \textit{An. anguilla} & Taiwan (Y) & 5 & 4 & 4.88 \\ 
  AA/R18F & female & \textit{An. anguilla} & Europe (R) & 4 & 1 & 4.80 \\ 
  AA/R28F & female & \textit{An. anguilla} & Europe (R) & 10 & 1 & 5.20 \\ 
  AA/R8F & female & \textit{An. anguilla} & Europe (B) & 27 & 1 & 5.20 \\ 
  AA/R16M & male & \textit{An. anguilla} & Europe (R) & 10 & 4 & 5.20 \\ 
  AA/R11M & male & \textit{An. anguilla} & Europe (R) & 25 & 14 & 6.40 \\ 
  AA/R2M & male & \textit{An. anguilla} & Europe (B) & 10 & 4 & 6.60 \\ 
  AJ/T8F & female & \textit{An. japonica} & Taiwan (Y) & 10 & 1 & 5.91 \\ 
  AJ/T5F & female & \textit{An. japonica} & Taiwan (K) & 2 & 1 & 4.80 \\ 
  AJ/T26F & female & \textit{An. japonica} & Taiwan (Y) & 2 & 1 & 2.40 \\ 
  AJ/T25M & male & \textit{An. japonica} & Taiwan (Y) & 24 & 5 & 4.05 \\ 
  AJ/T19M & male & \textit{An. japonica} & Taiwan (Y) & 24 & 7 & 3.50 \\ 
  AJ/T20M & male & \textit{An. japonica} & Taiwan (Y) & 20 & 8 & 3.80 \\ 
  AJ/R1F & female & \textit{An. japonica} & Europe (R) & 3 & 1 & 5.92 \\ 
  AJ/R3F & female & \textit{An. japonica} & Europe (R) & 3 & 1 & 6.90 \\ 
  AJ/R5F & female & \textit{An. japonica} & Europe (B) & 10 & 1 & 4.04 \\ 
  AJ/R1M & male & \textit{An. japonica} & Europe (R) & 3 & 1 & 2.50 \\ 
  AJ/R3M & male & \textit{An. japonica} & Europe (R) & 3 & 2 & 2.60 \\ 
  AJ/R5M & male & \textit{An. japonica} & Europe (B) & 10 & 1 & 2.23 \\ 
  \hline
\end{tabular}
\caption[Summary of RNA preparation]{\textbf{A summary of 24 samples
    prepared for RNA-seq -} The label of the RNA preparation follows a
  convention based on the eel species (host; first two letter of
  label, AA for \textit{An. anguilla} AJ for \textit{An. japonica}),
  worm population (population - R for European, T for Taiwanese) and
  sex of worm(s) in preparation (F for female, M for male; last letter
  in label). The European samples were from two locations: river Rhine
  (R,) and M\"uggelsee near Berlin (B), the Taiwanese samples were from
  from Kao Ping River (K) and Yunlin county (Y). Additionally the
  intensity of infection (number of adult worms found in the infected
  eel; intensity) and the number of worms pooled in the preparation
  (only male worms are pooled for RNA extraction, individual female
  worms were used). Finally RNA-concentration in the preparation (conc
  in prep) is given in $\mu$g per ml.}
\label{tab:lib-prep}
\end{center}
\end{table}

%% \afterpage{\clearpage}

Sequencing was performed in three multiplexed pools of eight libraries
each. The samples were partitioned into these pools spreading
replicates for each condition over all three pools to further
guarantee statistical independence from sequencing-lane effects. Each
pool of eight was sequenced on two lanes, giving in total six lanes of
data and two technical replicates for each library. Sequencing
resulted in a total of 263,668,952 raw sequencing read-pairs, each
read having a length of 60 bases and 270 bases mean insert size
between the read pairs.

\section{Examination of data-quality}

Reads were mapped against the fullest pyrosequencing-assembly (see
\ref{sec:final-full-assembly}) using \texttt{BWA}
\cite{pmid20080505}. Of the 263,668,952 raw read-pairs 173,602,387
mapped uniquely to the assembly and were counted on a per-library
base. 

The technical replicates demonstrated very low differences as inferred
from a clustering analysis using variance stabilised data and
transposed euclidean distances between samples (see figure
\ref{heat_all} a).

\figuremacroW{heat_all}{Distances between RNA-seq read-count for
  different samples}{Euclidean distance (square distance between the
  two count vectors) for variance stabilised read-counts for all
  libraries including technical replicates; Red indicates low distance
  (high similarity), blue high distance (low similarity). a) Data
  before screening and summation of technical replicates. All
  technical replicates are clustered very closely, the distance between
  an outlier female sample (AJ\_T26F) is high. b) Same illustration
  after summation of technical replicates and screening. Distance
  between outlier-sample and other female samples is reduced.}{0.5}

158,232,523 read-pairs were left after removal of hits to contigs for
which non-\textit{A. crassus} origin had been inferred in the
analysis of the 454-transcriptome assembly.

% latex table generated in R 2.13.0 by xtable 1.5-6 package
% Tue Nov 29 14:12:43 2011
\begin{table}[h]
\begin{center}
\begin{tabular}{llrrr}
  \hline
library & raw.reads & raw.mapped & tax.mapped & screened \\ 
  \hline
AA\_R11M & 11986442 & 8628520 & 7868814 & 6889551 \\ 
  AA\_R16M & 10810349 & 6858585 & 6217540 & 5276284 \\ 
  AA\_R18F & 9227615 & 6552527 & 5933235 & 5200958 \\ 
  AA\_R28F & 10135670 & 6665381 & 6005399 & 5171806 \\ 
  AA\_R2M & 12469746 & 7628428 & 6929651 & 5906422 \\ 
  AA\_R8F & 15270570 & 11527867 & 10758535 & 9453468 \\ 
  AA\_T12F & 11299438 & 7842479 & 7195621 & 6332396 \\ 
  AA\_T20F & 11740839 & 7744179 & 7114349 & 6323422 \\ 
  AA\_T24M & 8552723 & 5254194 & 4662053 & 3969305 \\ 
  AA\_T3M & 11031751 & 6460836 & 5800042 & 4993726 \\ 
  AA\_T42M & 11573501 & 7567845 & 6787375 & 5694801 \\ 
  AA\_T45F & 10646847 & 7714472 & 7173709 & 6283585 \\ 
  AJ\_R1F & 9855005 & 6400558 & 5890748 & 5167912 \\ 
  AJ\_R1M & 10211903 & 5851063 & 5313544 & 4506254 \\ 
  AJ\_R3F & 9897937 & 6425201 & 5948079 & 5124077 \\ 
  AJ\_R3M & 8775211 & 4562324 & 4073621 & 3422526 \\ 
  AJ\_R5F & 11949105 & 8442537 & 7830247 & 6882280 \\ 
  AJ\_R5M & 11231532 & 7504494 & 6772010 & 5913016 \\ 
  AJ\_T19M & 9195576 & 4798404 & 4293123 & 3635843 \\ 
  AJ\_T20M & 10862591 & 6880937 & 6251674 & 5280529 \\ 
  AJ\_T25M & 11195315 & 7162880 & 6480185 & 5645097 \\ 
  AJ\_T26F & 11195335 & 7439917 & 6641973 & 6031374 \\ 
  AJ\_T5F & 10357569 & 7413685 & 6794507 & 6007930 \\ 
  AJ\_T8F & 14196382 & 10275074 & 9496489 & 8364594 \\ 
   \hline
\end{tabular}
\caption[Mapping Summary]{\textbf{Mapping Summary} - Mapping is
  summarised for all 24 libraries. Rows indicate different libraries
  (worms or worm-pools as indicated in \ref{tab:lib-prep}) raw.reads
  gives the number of read-pairs sequenced, raw.mapped the number of
  reads mapping uniquely with their best hit, tax.mapped the number of
  reads after subtraction of reads to putative eel-host derived
  contigs and screened after subtraction of all reads mapping not to
  the highCA-derived assembly or to contigs with overall counts less
  than 32.}
\label{tab:read-clean}
\end{center}
\end{table}

%% \afterpage{\clearpage}

After another screening for spurious read-counts to low coverage
transcripts and to transcripts of low reliability (lowCA in the
454-assembly; see \ref{sec:final-full-assembly}) 137,477,156
read-pairs were left for further analysis. Distribution of these
read-pairs over libraries showed roughly 2.7-fold differences, with a
mean of 5,728,215 reads and a range from 3,422,526 read-pairs for
library AJ\_R3M to 9,453,468 read-pairs for library AA\_R8F (see
\ref{tab:read-clean}).

\figuremacro{mds}{Principle coordinate plot for expression in RNA-seq
  libraries}{Distance between sample-pairs is the root-mean-square
  deviation (Euclidean distance) for the most differentially expressed
  (DE) genes. Distances can be interpreted as the log2-fold-change of
  the genes with the biggest changes, i.e. the log2-fold-change for
  the genes that distinguish the samples.}

\afterpage{\clearpage}

These reads mapped to 7,520 contigs from our 454 assembly, making them
the basis for all further investigations.

In addition to hierarchical cluster analysis, also principal
component analysis grouped libraries according to the sex of worms
(the largest effect), but was unable to identify libraries with
expression correlated in more subtle ways (see figure \ref{heat_all}
b). Between-sample distance confirmed the hierarchical library
clustering. Sex of the worms defined the overall distances between
libraries, host- or population-differences were not visible in an
overall effect in the top differentially expressed (DE) genes (see
figure \ref{mds}). Male samples showed a smaller distance in
congruence due to the fact that they were made from pooled individuals
balancing expression differences for individual worms.

\section{Orthologous screening for expression differences}

For the 7,520 contigs with expression values 4,382
\textit{C. elegans}-orthologs and 4,292 \textit{B. malayi}-orthologs
were determined based on the annotation of our pyrosequencing-assembly
(see \ref{454-annot}). This resulted in 3,596 contigs with an
expression measurement, having a measurement also for both
corresponding orthologs (or group of orthologs) in both model-species
and thus being available for analysis.

For all further evaluations the congruence of the basic contig-based
statistics with orthologous-confirmed (OC) statistics is considered.

\section{Expression differences in generalised linear models}

Generalised linear models (GLMs) were used as implemented in the
R-package \texttt{edgeR}. Using these models I obtained 2,588 contigs
(34\% of total) DE between male and female worms at a false discovery
rate (FDR) of 5\%. 1,101 (31\% of total orthologous available) of
these contigs of were confirmed by contigs in the orthologous
evaluation. 1,425 (556 OC) of these were upregulated in male worms
1,163 (545 OC) in female worms.

At the same threshold, 55 contigs (0.7\% of total; 9, 0.25\% OC)
showed significant differential response to the host-species. 38 (5
OC) were upregulated in \textit{An. japonica}, 17 (4 OC) in
\textit{An. anguilla}.

68 contigs (0.9\% of total; 15, 0.42\% OC) showed differences
according to the population of the worm. 39 (11 OC) of these were
upregulated in the Taiwanese population, 29 (4 OC) in the European
populations.

An important observation in these models is the prevalence of
co-occurring significance of simple main effects. Expression changes
overlapping for two main effects mean a significant difference in
expression according to both factors. These differences are in the
same direction for a combination of the factors. Most contigs DE
according to the main effects of host-species or worm-population were
also DE according to the sex of the worm. There was also a number of
contigs differing for all three predictors in the same way. No contigs
were observed DE in both the host-species and worm-population in the
same direction but not according to worm-sex. From the 68 contigs DE
in different \textit{A. crassus}-populations, 38 were also DE
according to worm sex and 16 according to all three main effects (see
figure \ref{venn_mod_man}).

In addition, interaction-effects were also observed. The benefit of
also allowing contrasting significant differences in interaction terms
highlights the power of the GLM-approach. In these interactions a
difference according to both focal factors in different directions for
factor combinations is indicated. For interactions between
host-species and parasite-population (eel/pop), for example, this
mirrors the result of adult recovery i.e. a differential regulation
according to sympatric host-species/parasite-population combinations
as found in nature: 7 contigs (0 OC) showed differential expression
according to the worm-sex/eel-species interaction, 12 (3 OC) to
worm-sex/parasite-population, 13 (2 OC) to
host-species/parasite-population, 1 (0 OC) contig showed significance
for the 3-way interaction (see figure \ref{venn_mod_man}). It should
be noted, that conclusions drawn from of simple main effects do not
necessarily hold for contigs with significant interaction effects
(e.g. significantly higher expression in European population can then
mean higher values only in one of the host-species).

\figuremacroW{venn_mod_man}{Venn diagram of contigs significant for
  different terms in \texttt{edgeR}-GLMs}{Overlap between differences
  in simple main effects are given as black numbers in the
  Venn-Diagram. Numbers outside the circles in the lower left corner
  indicated non-significant contigs. The number of significant contigs
  for interaction effects are indicated in red for comparison. In (a)
  values for all contigs are given in (b) for ortholog-confirmed (OC)
  contigs.}{0.6}

\afterpage{\clearpage}

In summary, a low amount of overlap in main effects between
populations and host-species compared to the other main-effect
overlaps and in relation a higher proportion of interaction effects
between these two conditions was observed.

\section{Confirmation of contig categories through principal component
  analysis}

I performed constrained redundancy analysis for the effects of
eel-host and worm-population. This technique, similarly to principal
components analysis, can partition the variance into orthogonal
components, and additionally constrain one of the components to the
factor of interest. I found that 7\% of the variance in contigs DE
between eel-hosts and 11\% of the variance in contigs DE between
worm-population explained by the corresponding factor. In both
evaluations more than 50\% of the remaining variance could be
explained by a single principal component, to which sex contributed
over 99\% (loading) (see figure \ref{pca_eel} a and \ref{pca_pop}
a). When only OC-DE contigs were considered the explained variance for
difference between eel-host dropped to 3.3\% and the explained
variance for differences between worm-population was raised to 23\%,
while the sex-effect explained 70\% and 50\% of the variance (see
figure \ref{pca_eel} b and \ref{pca_pop} b). Significance of the
constrained component evaluated by a permutation-test could be
established at a p $<$ 0.05 threshold for all but the OC eel-host DE
subset.

\figuremacroW{pca_eel}{Constrained redundancy analysis for host-DE
  contigs}{Eel-host differences are displayed as constrained component
  on the x-axis, the sex contributed >99\% (loading) to the principal
  component on the y-axis. (a) Host differences partition the variance
  in samples in like expected for all contigs, the constrained
  component showed significance. (b) For OC contigs the constrained
  component fails to to partition the variance as expected, the
  component showed no significance for this subset of the data.}{0.60}

\figuremacroW{pca_pop}{Constrained redundancy analysis for
  population-DE contigs}{Population differences are displayed as
  constrained component on the x-axis, the principal component on the
  y-axis corresponds to the sex of the worm. Host differences partition
  the variance in samples like expected for all contigs (a) as well as
  for OC-contigs (b). The constrained component showed significance in
  both subsets.}{0.60}

\afterpage{\clearpage}

\section{Biological processes associated with DE contigs}

I employed tests for over-representation of categories in
gene-ontology (GO). These tests respect the structure of the ontology
and also consider over-representation of higher level (ancestor-)
terms. Summarising annotations at higher levels it is therefore
possible to conceive higher-order responses to the conditions
investigated.

For the differences between male and female worms enriched annotations
can be summarised into three broad categories: Terms over-represented
due to spermatogenesis (e.g. PP1-phosphatase and ester hydrolase are
important for spermatogenesis in \textit{C. elegans}
\cite{wormbook_sperm, fardilha2011protein}) embryo development (many
obvious terms) and terms for other processes more related to metabolic
differences between males and females (such as oxidoreductase
activity; see table \ref{func-sex} but also additional figures
\ref{tGO_sex_BP_classic_10_all}, \ref{tGO_sex_CC_classic_10_all} and
\ref{tGO_sex_MF_classic_10_all}).

\begin{longtable}{lp{4cm}rrrl}
\hline
GO.ID & Term & Annotated & Significant & Expected & p-value \\ 
\endfirsthead
\multicolumn{6}{c}%
{{\bfseries \tablename\ \thetable{} -- continued from previous page}} \\
\hline
GO.ID & Term & Annotated & Significant & Expected & p-value \\ 
\hline 
\endhead
\hline
\multicolumn{6}{|r|}{{Continued on next page}} \\ 
\hline
\endfoot
\endlastfoot
\hline
  \multicolumn{6}{l}{Molecular function} \\ 
  GO:0042578 & phosphoric ester hydrolase activity &  99 &  59 & 31.99 & 1.2e-08 \\ 
  GO:0016791 & phosphatase activity &  88 &  53 & 28.44 & 4.2e-08 \\ 
  GO:0004721 & phosphoprotein phosphatase activity &  65 &  42 & 21.00 & 6.5e-08 \\ 
  GO:0004722 & protein serine/threonine phosphatase act... &  34 &  24 & 10.99 & 4.8e-06 \\ 
  GO:0005509 & calcium ion binding &  78 &  43 & 25.21 & 2.1e-05 \\ 
  GO:0046873 & metal ion transmembrane transporter acti... &  32 &  21 & 10.34 & 0.00010 \\ 
  GO:0003824 & catalytic activity & 1354 & 482 & 437.55 & 0.00015 \\ 
  GO:0016614 & oxidoreductase activity, acting on CH-OH... &  46 &  27 & 14.86 & 0.00018 \\ 
  GO:0016616 & oxidoreductase activity, acting on the C... &  42 &  25 & 13.57 & 0.00023 \\ 
  GO:0017018 & myosin phosphatase activity &  10 &   9 & 3.23 & 0.00027 \\ 
  \hline
  \multicolumn{6}{l}{Biological process}  \\ 
  GO:0050896 & response to stimulus & 1535 & 583 & 504.78 & 1.7e-10 \\ 
  GO:0006470 & protein dephosphorylation &  63 &  41 & 20.72 & 1.2e-07 \\ 
  GO:0007391 & dorsal closure &  32 &  25 & 10.52 & 1.7e-07 \\ 
  GO:0016476 & regulation of embryonic cell shape &  13 &  13 & 4.27 & 5.0e-07 \\ 
  GO:0001700 & embryonic development via the syncytial ... &  49 &  33 & 16.11 & 6.7e-07 \\ 
  GO:0007392 & initiation of dorsal closure &  15 &  14 & 4.93 & 1.7e-06 \\ 
  GO:0046664 & dorsal closure, amnioserosa morphology c... &  15 &  14 & 4.93 & 1.7e-06 \\ 
  GO:0016311 & dephosphorylation &  86 &  49 & 28.28 & 2.6e-06 \\ 
  GO:0042221 & response to chemical stimulus & 864 & 337 & 284.12 & 3.1e-06 \\ 
  GO:0007394 & dorsal closure, elongation of leading ed... &  11 &  11 & 3.62 & 4.7e-06 \\ 
  \hline
  \multicolumn{6}{l}{Cellular compartment}  \\ 
  GO:0031224 & intrinsic to membrane & 372 & 164 & 118.85 & 8.4e-08 \\ 
  GO:0016021 & integral to membrane & 368 & 162 & 117.58 & 1.2e-07 \\ 
  GO:0005576 & extracellular region & 250 & 115 & 79.88 & 7.7e-07 \\ 
  GO:0031226 & intrinsic to plasma membrane & 176 &  86 & 56.23 & 1.0e-06 \\ 
  GO:0005887 & integral to plasma membrane & 172 &  84 & 54.95 & 1.4e-06 \\ 
  GO:0030054 & cell junction & 145 &  72 & 46.33 & 3.9e-06 \\ 
  GO:0000267 & cell fraction & 435 & 179 & 138.98 & 6.4e-06 \\ 
  GO:0016020 & membrane & 1154 & 417 & 368.70 & 3.6e-05 \\ 
  GO:0000164 & protein phosphatase type 1 complex &  14 &  12 & 4.47 & 4.9e-05 \\ 
  GO:0072357 & PTW/PP1 phosphatase complex &  14 &  12 & 4.47 & 4.9e-05 \\ 
  \hline\\
\caption[GO-terms enriched in DE between male and
female]{\textbf{GO-terms enriched in DE between male and female worms
    -} The top 10 enriched GO-categories are given for genes DE
  between the different male and female worms.}
\label{func-sex}
\end{longtable}

For the lower number of contigs DE between host-species inference of
higher order terms was obviously only possible to a limited extent and
in part also unnecessary, because annotations can be interpreted at
face value. However, annotations for contigs DE between eel-hosts
highlighted redundant terms associated with ``antigen processing and
presentation'' proteins which are in mammals usually involved in
antigen processing and cleavage of the invariant chain of the MHCII
complex. These terms led to Contig566 and Contig26 and their
\textit{B. malayi}-orthologs ``aspartic protease BmAsp-1, identical''
and ``eukaryotic aspartyl protease family protein''. In blood feeding
helminths these enzymes are in contrast usually involved in early
cleavage events during the digestion of host haemoglobin
\cite{pmid12782060}.


\figuremacro{tGO_pop_BP_classic_10_all}{GO biological process graph
  for enriched terms in DE according to worm-population}{Subgraph of
  the GO-ontology biological process category induced by the top 10
  terms identified as enriched in DE genes between different parasite
  populations. Boxes indicate the 10 most significant terms. Box
  colour represents the relative significance, ranging from dark red
  (most significant) to light yellow (least significant). In each node
  the category-identifier, a (eventually truncated) description of the
  term, the significance for enrichment and the number of DE / total
  number of annotated gene is given. Black arrows indicate a is
  ``is-a'' relationship.}

For contigs DE between worm populations despite the limited number of
DE contigs, enrichment analysis identified ``oxidoreductase activity''
as an informative significantly enriched higher level term (see figure
\ref{tGO_pop_MF_classic_10_all}). The biological processes ``response
to metal ion'' and ``mitochondiral electron transport'' (see
figure\ref{tGO_pop_BP_classic_10_all}) confirmed an evaluation linking
these mainly to enzymes used in respiratory processes and highlighted
additionally enzymes from lipid metabolism (especially
$\beta$-oxidation of fatty acids) related to respriration and the
availability of oxygen.

\section{Clustering analysis}

For the remainder of the text I will concentrate on these differences
of the European and Taiwanese populations and mention the other
differences only as far as they are related to this focal factor. In
\ref{add-figures} however, graphical analyses of the same type are
presented for other factors.

Clustering analysis uses distance measurements between samples as well
as genes (or transcripts) to highlight patterns of similarity. The
classical distance measure used in hierarchical clustering throughout
this document is Euclidean distance. Grouping of genes regulated in
parallel in combination with annotation, the status of cellular
processes can support notions based on single genes.

Hierarchical clustering analyses of genes DE between populations
confirmed the results of principal component based multivariate
analysis. The main factor grouping libraries was the sex of the
worm. A sub-grouping of samples fully according to European and
Taiwanese populations was only observed for male worms. In female
worms other unmeasured co-factors were preventing a clustering fully
according to this factor. In male worm however, library clustering
even followed a pattern of similar expression in according to the
second factor of eel-host. These statements are true for both the full
set of contigs (see figure \ref{pop_all_heat}) and OC contigs (see
\ref{pop_ortho_heat}).

Clustering of genes revealed three co-regulated groups in the full set
of contigs and the OC set. The first of gene-clusters (top in
\ref{pop_all_heat} and \ref{pop_ortho_heat}) was in sex-subgroups
mainly following an expression pattern differing between
populations. The second gene-group was much larger in the full set
than in the OC set of contigs (middle in \ref{pop_all_heat}). It was
only very weakly reacting to any other factor but sex and was very
sparsely annotated (therefore this group was much smaller in the OC
set \ref{pop_ortho_heat}). The third gene-group found again in both
the full and OC contigs (bottom in \ref{pop_all_heat} and
\ref{pop_ortho_heat}) was reacting on both the host and population
factor in a converse way. Contigs in this cluster were mainly found to
be significant for interaction effects.

\figuremacroW{pop_all_heat}{Clustering of expression values for contigs
  DE between populations}{A heatmap of variance/mean stabilised
  expression values. Deprograms are based on hierarchical
  clustering. Green indicates expression below the mean, red above the
  mean. Experimental conditions are indicated by black bars for groups
  of samples (columns) below the plot. Presence GO-term annotation for
  contigs (rows) are given as black bars right to the plot:
  isOxidoreductase = GO:0016491, oxidoreductase activity;
  isMitochondrial = GO:0005739, mitochondrion; isELDevelopment =
  GO:0002164, larval development or GO:0009791, post-embryonic
  development; isResponsetoStim = GO:0050896, response to stimulus;
  isPhosphatase = GO:0016791, phosphatase; isMembrane = GO:0016020,
  membrane; isAntigenProc = GO:0002478, antigen processing and
  presentation of exogenous peptide antigen; isEndosome = GO:0005768,
  endosome; isProtLipComp = GO:0032994, protein-lipid complex. Grey
  bars indicate no annotation available.}{1.5}

Consolidating the clusters with annotation and annotation-enrichment,
the first cluster of genes was very well annotated and contained
mostly catalytic enzymes involved in oxidation and reduction, the
bottom cluster contained more unannotated genes and structural
(cuticular collagen) genes.

\section{Single gene differences}
\label{sec:single-gene-diff}

Tables on single transcript values of OC contigs DE between eel-hosts
and populations can be found in additional tables \ref{eel.sing.diff}
and \ref{pop.sing.diff}. Obviously for some contigs differences
significant in the model are rendered inaccessibly by comparing simple
mean values because of superposed interaction effects or overwhelming
general effects of worm sex.

Cytochrome C oxidase subunit 2 (COXII) shows the clearest of all
expression patterns for any of the observed genes. It differed
significantly only between populations (showed no reaction an any
other factor) and was on average over 1,000-fold stronger expressed in
the Taiwanese population. At face values differed for every single
individual (of the 12 investigated in each populations) at least
20-fold (highest normalised expression was 350 counts in a European
worm, lowest normalised expression in any Taiwanese worm was 7,500
counts). Counts summed for orthologs were also significant only for
this factor and showed over 10-fold stronger expression in the same
direction. This accounts to the fact, that misassembled contigs
containing fragments of COXII were only adding experimental noise.

% Malate/L-lactate dehydrogenase and a 3-hydroxyacyl-CoA dehydrogenase
% were also following more complicated expression patterns two
% cholesterol acyltransferase transcipts

% \section{Sequence polymorphism in RNAseq data}
% \label{sec:pol-rnaseq}


% Because of the clear Taiwanese/European distinction for COXII
% expression in the 24 investigated worms a variant would be expected to
% be private to rather the twelve Taiwanese or twelve European
% individuals. Preliminary results on such private sequence variation
% have not shown polymorphism in COXII or in the mitochondrial genome in
% general.


%%% Local Variables: ***
%%% mode:latex ***
%%% TeX-master: "../thesis.tex"  ***
%%% tex-main-file: "../thesis.tex" ***
%%% End: ***
