% this file is called up by thesis.tex
% content in this file will be fed into the main document

%: ----------------------- name of chapter  -------------------------
\chapter{Evaluation of an assembly strategy for pyrosequencing
  reads} % top level followed by section, subsection


%: ----------------------- paths to graphics ------------------------

% change according to folder and file names
\ifpdf
    \graphicspath{{4_eval_ass/figures/PNG/}{4_eval_ass/figures/PDF/}{4_eval_ass/figures/}}
\else
    \graphicspath{{4_eval_ass/figures/EPS/}{4_eval_ass/figures/}}
\fi

%: ----------------------- contents from here ------------------------


\section{Overview}
\label{sec:over-eval}

This chapter reports on an important methodical detail of
\ref{cha:pyro}: the sequence-assembly. The quality of this sequence
assembly constitues a fundamental foundation of the later
chapters. 

The pre-processed \textit{A. crassus} data-set consisting of 100491819
bases in 353055 reads (58617 generated using ``FLX-chemistry'', 294438
using ``Titanium-chemistry'') was assembled following an approach
proposed by \cite{pmid20950480}: Two assemblies were generated, one
using \texttt{newbler v2.6} \cite{pmid16056220}, the other using
\texttt{mira v3.2.1} \cite{miraEST}. The resulting assemblies
(referred to as first-order assemblies) were merged with \texttt{Cap3}
\cite{Cap3_Huang} into a combined assembly (referred to as
second-order assembly). 

Summary statistics for the assemblies, demonstrating the superiority
of the second-order assembly are reported as well as summary
statistics for single contigs. These metadata on contigs are important
in evalutation of downsteam results. As a perfect assembly with each
contig represtenting a single full transcript is illusive and every
contig constitutes an hypothesis, it becomes important to validate and
question analyses based on as much information as possible.

\section{The \texttt{newbler} first-order assembly}
\label{sec:new-fist}

During transcriptome-assembly \texttt{newbler} can split individual
reads spanning the breakpoints of alternate isoforms, to assemble
e.g. the first portion of the reads in one contig, the second portion
in two different contigs. Later multiple so called isotigs would be
constructed and reported, one for each putative
transcript-variant. While this approach could be helpful for the
detection of alternate isoforms, it also produces short contigs
(especially at error-prone edges of high-coverage transcripts) when
the building of isotigs fails. The read-status report and the assembly
output in \texttt{ace}-format the program provides are including short
contigs only used during the assembly-process, but not reported in the
contigs-file used in transcriptome-assembly projects
(\texttt{454Isotigs.fna}). Therefore to get all reads not included in
contigs (i.e. a consistent definition of ``singleton'') it was
necessary to add all reads appearing only in contigs not reported in
the fasta-file to the reported singletons. The number of singletons
increased in this step form the 26211 reported to 109052. We later
also address the usefulness of \texttt{newbler's} report vs. the
expanded singleton-category, but for the meantime we define singletons
as all reads not present in a given assembly.

\figuremacro{split_454}{Number of contigs/isotigs splitted}{A
  histogram of the number of contigs or isotigs \texttt{newbler}
  splited a single read into}

As mentioned above, the splitting of reads in the \texttt{newbler}
assembly can give useful information on possible isoforms, however the
the number of contigs \texttt{newbler} splitted one read into (in some
cases mort than 100 contigs) seems artificially inflated (see figure
\ref{split_454}). If information would correspond to real isoforms it
should be about an order of magnitude lower. This fact emphasises the
need for further processing of the contigs. The maximal number of
read-splits in a given contig and it's usefulness will be discussed
later in greater detail.

\section{The \texttt{mira}-assembly and the second-order assembly}
\label{sec:assembly-sec}

The \texttt{mira} assembly provided a second estimate of the
transcriptome. In this assembly individual reads are not split. The
number of reads not used in the \texttt{mira}-assembly was 65368.

To combine the two assemblies \texttt{cap3} was used with default
parameters and including the quality information from first-order
assemblies. The reminder of this text deals with the exploratory
analysis of how information from both estimates of the transcriptome
are integrated into the final second-order assembly.

% latex table generated in R 2.13.0 by xtable 1.5-6 package
% Tue Nov 15 23:52:26 2011
\begin{table}[ht]
\begin{center}
\begin{tabular}{rrrr}
  \hline
 & Newbler & Mira & Second-order(MN) \\ 
  \hline
Max length & 6300 & 6352 & 6377 \\ 
  Number of contigs & 15934 & 22596 & 14064 \\ 
  Number of Bases & 8085922 & 12010349 & 8139143 \\ 
  N50 & 579 & 579 & 662 \\ 
  Number of congtigs in N50 & 4301 & 6749 & 3899 \\ 
  non ATGC bases & 375 & 29962 & 5245 \\ 
  Mean length & 508 & 532 & 579 \\ 
   \hline
\end{tabular}
\caption[Statistics for the first-order assemblies]{\textbf{Statistics
    for the first-order assemblies -} Basic statistics for the
  first-order assemblies and the second-order assemly (for which only
  the most relieable category of contigs (MN) is shown see
  ref{sec:data-categ-second})}
\label{tab:pc}
\end{center}
\end{table}


Table \ref{tab:pc} gives basic summary-statistics of the different
assemblies. \texttt{mira} clearly produced the biggest assembly, both
in terms of number of contigs and bases), the second-order assembly is
slightly smaller size than the \texttt{newbler} assembly.  The
second-order assembly had on average longer contigs than both
first-order assemblies and a higher weighted median contig size (N50).

\section{Data-categories in the second-order assembly}
\label{sec:data-categ-second}

Three main categories of assembled sequence data can be distinguished
in the second-order assembly, each one with different reliability and
purpose in downstream applications: The first category of data
obtained are the singletons of the final second-order assembly. It
comprises raw sequencing reads that neither of the first-order
assemblers used. It is therefore the intersecion of the
\texttt{newbler}-singletons (as defined in \ref{sec:new-fist}) and the
\texttt{mira}-singletons. 47669 reads fell in this category. A second
category of sequence contains the first-order contigs, that could not
be assembled in the second-order assembly (the singletons in the
\texttt{cap3}-assembly; M\_1 and N\_1 in table
\ref{tab:categ}). Furthermore second-order contigs in which
first-order contigs from only one assembler are combined (M\_n and
N\_n in table \ref{tab:categ}) also have to be included in this
category. Sequences in this category should be considered only
moderately reliable as they are supported by only one assembly
algorithm.

Finally the category of contigs considered most reliable contains all
second-order contigs with contribution from both first-order
assemblies (MN in table \ref{tab:categ}).

% latex table generated in R 2.13.0 by xtable 1.5-6 package
% Tue Nov 15 23:52:34 2011
\begin{table}[ht]
\begin{center}
\begin{tabular}{rlllll}
  \hline
 & M\_1 & M\_n & MN & N\_n & N\_1 \\ 
  \hline
Snd.o.con &   & 164 & 13887 & 13 &   \\ 
  Fst.o.con & 2347 & 897 & mira=19352/newbler=14410 & 40 & 1484 \\ 
  reads & 42172 & 21153 & one=269868/both=193308 & 1538 & 13100 \\ 
   \hline
\end{tabular}
\caption[number of reads in assemblies]{\textbf{Number of reads in
    assemblies} for first-order contigs (Fst.o.con) and second-order
  contigs (Snd.o.con) numbers for for different categories of contigs
  are given: M\_1 and N\_1 = first-order contigs not assembled in
  second-order assembly, from mira and newbler respectively; M\_n and
  N\_n = assembled in second-order contigs only with contigs from the
  same first-order assembly; MN = assembled in second-order contigs
  with first order contigs from both first order assemblies.}
\label{tab:categ}
\end{center}
\end{table}

For the last, most reliable (MN) category, reads contained in the
assembly can be categorized depending on whether they entered the
assembly via both or only via one first-order assembly.

\figuremacro{read_origin}{Origing of reads}{Reads in the most reliable
  (MN) assembly-category are categorized by the way they entered the
  assembly: Although they are in a highly credible contig, reads can
  still have entered from only one first order assembly (Mira\_in\_MN
  or Newbler\_in\_MN). The intersection gives the reads which entered
  via both routes. The duplicated category gives the number of reads
  splitted by Newber and the intersection reads, which were splitted
  and entered the assembly.}

Figure \ref{read_origin} gives a more detailed view of the fate of the
reads \texttt{newbler} splited during first-order
assembly. Interestingly most reads \texttt{newbler} splited ended in
the high-quality category of the second order assembly only 

\section{Contribution of first-order assemblies to second-order contigs}
\label{sec:contr-firs-order}

Looking at the contribution of contigs from each of the assemblies to
one second-order contig in figure \ref{ass_contributions} a it becomes
clear, that the \texttt{mira}-assembly had a high number of redundant
contigs. These were assembled into the same contig by \texttt{newbler}
and finally also in one second-order contig by \texttt{Cap3}.

\figuremacro{ass_contributions}{Contribution to second-order assembly}
{Number of first-order contigs from both first-order assemblies for
  each second order contig (a) number of reads through
  \texttt{newbler} and \texttt{mira} for each second-order contig (b)}

A different picture emerges from the contribution of reads through
each of the first-order assemblies (figure \ref{ass_contributions}
b). Here for most second-order contigs many more reads are contributed
through \texttt{newbler}-contigs. This is because \texttt{newbler} has
more reads summed over all contigs caused by the duplication due to
splitting of reads.

\section{Evaluation of the assemblies}
\label{sec:eval-three-assembl}

To further compare assemblies (\texttt{mira, newber} first-order
assemblies including or excluding their singletons) and the
second-order assembly (including different contigs-categories and
singletons) we evaluated the number of bases or proteins their contigs
and singletons (partially) cover in the related model-nematodes,
\textit{Caenorhabditis elegans} and \textit{Brugia malayi}. 

In addition, the size of the assembly can give an indication of
redundancy or artificially assembled data: If it increases without
improving the reference-coverage the dataset is likely to contain
more redundant or artificial information, a more parsimonious assembly
should be preferred.

The database-coverage for the two reference species can then be
plottedf against the size of the assembly-dataset to estimate the
completeness conditional to the size of the assembly (figures
\ref{base_ref_b}, \ref{base_ref_p}, \ref{base_ref_p}) .

From the assemblies excluding singletons (in the lower left corner
with lower size and database-coverage) the highly reliable
contig-category of the second-order assembly produced the highest
per-base coverage in both reference-species, with the \texttt{newbler}
assembly on a second place and \texttt{mira} producing the lowest
reference-coverage. When adding the contigs considered lower quality
supported by only one assembler to the second-order assembly the
reference-coverage increased moderately.

Including singletons the \texttt{mira} and \texttt{newber} assemblies
were of increased size. A comparison of the \texttt{newbler's}
reported singletons with all singletons addet to the
\texttt{newbler}-assembly shows, that the reported singletons
increased reference-coverage to the same amount than all singletons,
while the non-reported singletons only increased the size of the
assembly. It can be concluded, that the latter contain hardly any
additional information but only error-prone or variant reads.

The second-order assembly including the intersection of first-order
singletons performed similar to the \texttt{newbler} assembly for the
number of bases coverd, but was larger in size. Adding the less
reliable set of one-assembler supported second-order-contigs the
assembly coverd the most bases in both references. When not the
singleton of the second-order assembly (as defined in
\ref{sec:new-fist}) but the intersection of \texttt{newbler's}
``reported singletons'' and \texttt{mira's} singletons were considered
a very parsimonious assembly with high reference-coverage (termed
fullest assembly; and labeled FU in the plots above) was obtained.


\figuremacro{base_ref_b}{Base-content and reference-transcriptome
  coverage in percent of bases}{for different assemblies and
  assembly-combinations; M = \texttt{mira}; N = \texttt{newbler};
  $M+S$ = \texttt{mira} + singletons; $N+S$ = \texttt{newbler} plus
  singletons; $N+Sr$ = \texttt{newbler} plus singletons reported in
  readstatus.txt; MN = second-order contigs supported by both
  first-order; $MN+N\_x$ = second-order MN plus contigs only supported
  by \texttt{newbler}; $MN+M\_x$ = same for
  \texttt{mira}-first-order-contigs; $MN+M\_x+S$ and $MN+N\_x+S$ same
  with singletons; FU = second-order contigs supported by both or one
  assembler plus the intersection of \texttt{newbler} reported
  singletons and \texttt{mira}-singletons = the basis for the
  ``fullest assembly'' used in later analyses}


\figuremacro{base_ref_p}{Base-content and reference-transcriptome
  coverage}{in percent of proteins hit for different assemblies and
  assembly-combinations (for category-abrevations see figure
  \ref{base_ref_b})}

Considering the reference-database with any kind of coverage the
second-order assembly performed less preferable. Excluding singletons
it was covering similar numbers of database-proteins than the
\texttt{newber}-assembly and and was outperformed by the
\texttt{mira}-assembly, although the latter showed again to be least
parsimonious. The same general picture emerged from this analysis when
singletons were considered additionally. \texttt{newbler} and
second-order assemblies coverd similar amounts of reference-data.

\figuremacro{base_ref_8}{Base-content and reference-transcriptome
  coverage in percent of proteins coverd to at least 80\%}{of their
  length for different assemblies and assembly-combinations (for
  category-abrevations see figure \ref{base_ref_b})}

When database-proteins covered for at least to 80\% of their length
are considered the second-order assembly showed it's superiority: Both
ex- and including singletons the second-order assembly outperformed
the first-order assemblies. Moderate gains in reference coverage were
made again for the addition of dubious single-assembler supported
second-order contigs. We give most weight in our analysis to these
results as in average longer correct contigs will allow finding the
highest number of putative full-length genes.

Given this evaluation we defined a ``minimal adequate'' assembly as
the subset of contigs of the second-order assembly supported by both
assemblers (labeled MN above).

Given the performance of the singletons \texttt{newbler} reported we
defined a ``fullest-assembly'' as all second-order contigs (including
those supported by only one assembler) plus the intersection of
reported \texttt{newbler}-singletons and \texttt{mira} singletons.

\section{Measurments on second-order assembly}

Based on the following reads through the complicated assembly process,
we calculated the following for each contig in the second-order
assembly, to report it to for use in later analysis.

\begin{itemize}
\item number of \texttt{mira} and \texttt{newbler} first-order contigs
\item number of reads through \texttt{mira} and reads through \texttt{newbler}
\item number of reads being split by \texttt{newbler} in first-order
  assembly 
\item number of read-split events in the first-order assembly (equals
  the sum of reads multiplied by number of contigs a read has been
  split into)
\item maximal number of first-order contigs a read in the contig has
  been split into during \texttt{newbler}-assembly 
\item the number of reads same-read-paires from the \texttt{newbler}
  and \texttt{mira} first order-assembly merged in a second order
  contig
\item cluster-id of the contig: All contigs ``connected'' by sharing
  reads (similar to the graph clustering reported in
  \cite{pmid21138572}). 
\item number of other second order contigs containing the same read
  (size of the cluster)
\end{itemize}


\subsection{Contig coverage}

As well defined coverage-information is not readyly avaiable from the
output of this combined assembly aproach (although we followed
individual reads through the process) we inferred coverage by mapping
the reads used for assembly against the fullest assembly using ssaha2
\cite{pmid11591649} with parameters (-kmer 13 -skip 3 -seeds 6 -score
100 -cmatch 10 -ckmer 6 -output sam -best 1). We converted the
\texttt{sam}-ouput via a sorted \texttt{bam}-file to
\texttt{pileup}-format using \texttt{samtools}
\cite{journals/bioinformatics/LiHWFRHMAD09}.

For a second evalutation we excluded best-hits mapping to multiple
contigs before converting the \texttt{sam}-file. 
\begin{itemize}
\item mean per base coverage
\item mean unique per base coverage

\end{itemize}

\subsection{Example use of the contig-measurements}

Based on these measurements the emergence of a given contig from the
assembly process can be reconstructed. Table \ref{tab:ex-me} gives an
excerpt of the contig-measurements reported in additional-file
\texttt{contig-data.csv}. The example contigs are all from large
contig-clusters (cluster.size), where interpretation of the assembly
history is complicated, but not impossible:

% latex table generated in R 2.13.0 by xtable 1.5-6 package
% Tue Nov 15 23:54:13 2011
\begin{table}[ht]
\begin{center}
\begin{tabular}{rllll}
  \hline
 & Contig1047 & Contig10719 & Contig104 & Contig13672 \\ 
  \hline
reads\_through\_Newbler &   16 & 1351 &    0 &   14 \\ 
  reads\_through\_Mira &  26 & 651 & 135 &   0 \\ 
  Newbler\_contigs & 1 & 5 & 0 & 2 \\ 
  Mira\_contigs & 1 & 9 & 4 & 0 \\ 
  category & MN & MN & M\_n & N\_n \\ 
  num.new.split &    8 & 1314 &    0 &    0 \\ 
  sum.new.split &   16 & 2628 &    0 &    0 \\ 
  max.new.split & 2 & 2 & 0 & 0 \\ 
  num.SndO.pair &  13 & 644 &   0 &   0 \\ 
  cluster.id & CL62 & CL6 & CL176 & CL235 \\ 
  cluster.size & 24 & 18 &  5 &  5 \\ 
  coverage &   4.200342 & 267.495458 &  41.003369 &   2.920755 \\ 
  uniq\_coverage & 4.248960 & 7.425507 & 2.568000 & 1.196078 \\ 
   \hline
\end{tabular}
\caption[Example for assembly-measurements]{\textbf{Example table for
    assembly-measurements - } measurements on contigs (as given in
  additional file \texttt{contig-data.csv}), row-labels are explained
  in a detailed example in the main text}
\label{tab:ex-me}
\end{center}
\end{table}
\textbf{Contig1047} is in the well trusted MN category of contigs. It
consists of only one contig from each first-order assembly
(newbler\_contigs and mira\_contigs), each containing a set of reads
of moderate size: 16 from \texttt{newbler} (reads\_through\_newbler)
26 from \texttt{mira} (reads\_through\_mira). 8 of the 16 reads
\texttt{newbler} used in its one assembled contig were also assembled
to a differnt \texttt{newbler}-contig (num.new.split). That each of
the 8 reads was only appearing in one other \texttt{newbler}-contig is
visible from the fact, that the number of split events is 16
(sum.new.split) and the maximal number of splits for one read is 2
(max.new.split). 13 (num.SndO.pair) same-read-pairs from the tow
different first-order assemblies were merged in this second-order
contig, leaving 3 (16-13) reads in \texttt{newbler}-contigs and 13
(26-13) reads in \texttt{mira} contigs, which all could potentially
have ended up in other contigs. The contig is in a cluster (CL62),
which contains in total 24 contigs (cluster.size). It has to be
addmitted that the whole graph-structure linking this 24 contigs can't
be reconstructed from this contig summary data. On the other hand the
summary data makes clear, from what source the links for
cluster-affiliation have resulted: In this case from 3 and 13 unlinked
read-paires from both first-order assemblies and 8 split-reads from
\texttt{newbler}-fistr order contigs.

A comprehensive interpretation of the other example-contigs depicted
is left to the reader. It should just be remarked, that in case of
one-assembler supported contigs, all reads in that contig could
potentially be represented in other contigs, making average
cluster-size in these contigs bigger than in the MN category.

One of the most interesting measurement calculated for each contig is
the cluster-membership and cluster-size. Such clusters can represent
close paralogs, duplicated genes, isoforms from alternative splicing
or allelic variants.

These measurements can be used in later analysis to e.g. reevaluate
the likelihood of misassembly in a given set of contigs. An evaluation
of other cluster members, when biologically interesting properties are
inferred for a contig in a cluster is e.g. advised and will be
demonstrated in later in the manuscript.


\section{Finalising the fullest assembly set}
\label{sec:final-full-assembly}

In order to minimize the amount of sequence with artificially inferred
isoform-breakpoints we used the unique-mapping-information described
above to detect contigs and singletons not supported by any raw data
(reads). Table \ref{tab:cov.ex} gives a summary of these unsupported
data by contig-category. For all downstream-analysis we removed all
well trusted MN-category contigs having no coverage at all and the
contigs (and singletons) from other categories having no unique
coverage.

% latex table generated in R 2.13.0 by xtable 1.5-6 package
% Tue Nov 15 23:54:14 2011
\begin{table}[ht]
\begin{center}
\begin{tabular}{rrrrrrr}
  \hline
 & singletons & M\_1 & M\_n & MN & N\_1 & N\_n \\ 
  \hline
coverage == 0 & 546 &  34 &   2 &  36 & 158 &   0 \\ 
  unique coverage == 0 & 584 &  48 &   2 &  42 & 210 &   3 \\ 
   \hline
\end{tabular}
\caption[finalizing the assembly]{\textbf{Final filtering of the
    assembly -} Number of contigs with a coverage and unique-coverage
  of zero, inferred from mapping of raw reads, listed by
  contig-category}
\label{tab:cov.ex}
\end{center}
\end{table}

Thereby we reduced our dataset to 40187 tentative unique genes (TUGs),
redifining the ``fullest assembly'' dataset. Based on the above
evaluation we decided to treat the MN-category of contigs as high
credibility assembly (highCA) and to subsume the M\_n, N\_n, M\_1,
N\_1 and \texttt{Newbler's} reported singletons as additional low
credibility assembly (lowCA).

%%% Local Variables: ***
%%% mode:latex ***
%%% TeX-master: "../thesis.tex"  ***
%%% tex-main-file: "../thesis.tex" ***
%%% End: ***
