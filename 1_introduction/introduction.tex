
% this file is called up by thesis.tex
% content in this file will be fed into the main document

%: ----------------------- introduction file header -----------------------
\chapter{Introduction}

% the code below specifies where the figures are stored
\ifpdf
    \graphicspath{{1_introduction/figures/PNG/}{1_introduction/figures/PDF/}{1_introduction/figures/}}
\else
    \graphicspath{{1_introduction/figures/EPS/}{1_introduction/figures/}}
\fi

% ----------------------------------------------------------------------
%: ----------------------- introduction content ----------------------- 
% ----------------------------------------------------------------------



%: ----------------------- HELP: latex document organisation
% the commands below help you to subdivide and organise your thesis
%    \chapter{}       = level 1, top level
%    \section{}       = level 2
%    \subsection{}    = level 3
%    \subsubsection{} = level 4
% note that everything after the percentage sign is hidden from output



\section{The study organism: \textit{Anguillicola crassus}} 

\subsection{Ecological significance} 

\textit{Anguillicola crassus} Kuwahara, Niimi and Ithakagi 1974
\cite{kuwahara_Niimi_Itagaki_1974, moravec_anguillicoloides} is a
swimbladder nematode naturally parasitizing the Japanese eel
(\textit{Anguilla japonica}) indigenous to East-Asia. After a single
introduction \cite{wielgoss_population_2008} to Germany in the early
1980s \textit{A. crassus} has colonized almost all populations of the
European eel (\textit{Anguilla anguilla})
\cite{kirk_impact_2003}. Since the 1990s populations of the American
eel (\textit{Anguilla rostrata}) have been colonized as novel hosts
\cite{fries_notes:_1996,barse_exotic_1999, barse_swimbladder_2001} and
finally it has been detected in three indigenous \textit{Anguilla}
species on the island of Reunion near Madagascar
\cite{sasal_parasite_2008}.\



In Asia, as well as in the introduced ranges, copepods and ostracods
serve as intermediate hosts of \textit{A. crassus}
\cite{moravec_first_2005}, in which L2 larvae develop to L3 larvae,
infective to the final host. Once ingested by an eel they migrate
through the intestinal wall and via the body cavity into the
swimbladder wall \cite{haenen_effects_1996}, i.a. using a trypsin-like
proteinase\cite{polzer_identification_1993}. In the swimbladder wall
L3 larvae hatch to L4 larvae. After a final moult from L4 to preadult
the parasites inhabit the lumen of the swimbladder, where they
eventually mate. Eggs containing L2 larvae are released via the ductus
pneumaticus into the eels gut and finally into the
water\cite{de_charleroy_life_1990}.\

Within the novel range and hosts, conspicuously elevated prevalences
and intensities of infection occur (reviewed in
\cite{kirk_impact_2003} and \cite{taraschewski_hosts_2007}). These
differences in abundance of \textit{A. crassus} in East Asia compared
to Europe are commonly attributed to the different host-parasite
relations in the final eel host permitting a differential survival of
the larval and the adult parasites
\cite{knopf_differences_2004}. Recently, data from experimental
infections of European eels with \textit{A. crassus} have been
published \cite{fazio_regulation_2008}. They show that the parasite
undergoes (under experimental conditions) a density-dependent
regulation keeping the number of worms within a certain range.\
  

\subsection{Evolutionary significance}

\subsubsection{Divergence of \textit{A. crassus} populations}

\figuremacro{ula_stages.png}{Differences in developmental speed}{data
  courtesy of Urszula Weclawski
  %% \href{http://someurl}{urlanchor<}.
  %% \figuremacroW{ula_stages.png}{Title}{Caption}{0.8}
  % variation of the above macro with a width setting
}

\subsubsection{Interest in \textit{A. crassus based on its
    phylogenetic position in the phylum nematoda}}



\subsection{Functional insights from other nematodes used to formulate
  hypotheses for \textit{A.crassus}}


\section{Advances in sequencing technology enabeling this study}

Recent advances in DNA-sequencing


\figuremacro{sequencing_costs.png}{Falling seqeuncing costs}{Falling into bottomless, Data
  provided by \href{http://www.genome.gov/sequencingcosts/}{National
    Human Genome Research Institute, NHGRI}.}



\subsection{Pyro-sequencing}


\subsection{Illumina-Solexa sequencing}



\begin{table}[htdp]
\centering
\begin{tabular}{ccc} % ccc means 3 columns, all centered; alternatives are l, r

{\bf Gene} & {\bf GeneID} & {\bf Length} \\ 
% & denotes the end of a cell/column, \\ changes to next table row
\hline % draws a line under the column headers

human latexin & 1234 & 14.9 kbps \\
mouse latexin & 2345 & 10.1 kbps \\
rat latexin   & 3456 & 9.6 kbps \\
% Watch out. Every line must have 3 columns = 2x &. 
% Otherwise you will get an error.

\end{tabular}
\caption[title of table]{\textbf{title of table} - Overview of latexin genes.}
% You only need to write the title twice if you don't want it to appear in bold in the list of tables.
\label{latexin_genes} % label for cross-links with \ref{latexin_genes}
\end{table}

% There you go. You already know the most important things.


% ----------------------------------------------------------------------

%%% Local Variables: ***
%%% mode:latex ***
%%% TeX-master: "../thesis.tex"  ***
%%% End: ***
     