
% this file is called up by thesis.tex
% content in this file will be fed into the main document

%: ----------------------- introduction file header -----------------------
\chapter{Introduction}
\label{intro}
% the code below specifies where the figures are stored
\ifpdf
    \graphicspath{{1_introduction/figures/PNG/}{1_introduction/figures/PDF/}{1_introduction/figures/}}
\else
    \graphicspath{{1_introduction/figures/EPS/}{1_introduction/figures/}}
\fi

% ----------------------------------------------------------------------
%: ----------------------- introduction content ----------------------- 
% ----------------------------------------------------------------------

\section{The study organism: \textit{Anguillicola crassus}} 
\label{stud-org}

\subsection{Ecological significance} 
\label{eco-sig}

\textit{Anguillicola crassus} Kuwahara, Niimi and Ithakagi 1974
\cite{kuwahara_Niimi_Itagaki_1974, moravec_anguillicoloides} is a
swimbladder nematode naturally parasitizing the Japanese eel
(\textit{Anguilla japonica}) indigenous to East-Asia. In the last 30
years anthropogenic expansions of its geographic- and host-range to
new continents and host-species attracted interest of limnologists and
ecologists. The newly accquired hosts are, like the native host,
freshwater eels of the genus \textit{Anguilla} (and the use of the
final host seems to be limited to this genus
\cite{sures_development_1999}), but the nematode displayed a high
versatility and plasticity in most other aspects of it's life to
successfully invade new continents \cite{taraschewski_hosts_2007}.

First \textit{A. crassus} colonized Europe in the eraly 1980ies and
spread through almost all populations of the European eel
(\textit{Anguilla anguilla}) during the following decades (reviewed in
\cite{kirk_impact_2003}). This spread includes populations of the
European eel in North
Africa\cite{gargouri_ben_abdallah_spatio-temporal_2006,
  loukili_dynamics_2007}. At the present day \textit{A. crassus} is
found in all but the northernmost population of the European eel in
Iceland \cite{kristmundsson_parasite_2007}. It has to be noted
however, that low water temperature \cite{knopf_impact_1998} and
salinity \cite{kirk_effect_2000} limit the dispersal of
\textit{A. crassus} larvae and high epidemiological prameters are
rather expected in freshwater and in southern latitudes.

Wielgoss et al. \cite{wielgoss_population_2008} studied the population
structure of \textit{A. crassus} using microsattelite markers and
inferred details about the colonization process and history. These
details are in very good agreement with previous knowledge about the
introduction history, and lead to the situation that the process of
introduction and spread can be considered very well described:

\textit{A. crssus} was first recorded in 1882 in North-West Germany,
and this record was published in a German fishery magazine in 1985
\cite{fischer_teichwirt}. The import of Japanese Eels from Taiwan to
the habor of Bremerhaven in 1980, was soon identified as most likely
introduction event \cite{koops_anguillicola-infestations_1989}. Taiwan
as the most likely geographical source of the introduction was in turn
also inferred from population structure by Wielgoss et
al. Furthermore, from the fact that genetic diversity is highest in
northern regions of Germany and gradually declines to the south, they
concluded a single introduction event to Germany as source for all
populations of \textit{A. crassus} in the comprehensive set of
investigated populations of the European eel. This signal was
persistent together with a strong signal for anthropogenic mixing of
eel and parasite populations due to restocking \cite{pmid20646147}.


\figuremacro{world-ac}{Transcontinental dispersal of
  \textit{A. crassus:}}{Invasions of different continents by different
  source-populationa are illsutrated using arrows. Red color indicates
  the range of the eel species targeted by the invasion. Modified form
  \cite{mnderle_kologische_2005}, based on data reviewed in
  \cite{kirk_impact_2003} and newer findings in
  \cite{wielgoss_population_2008} and \cite{sasal_parasite_2008}}

A little less is known about a second colonization of
\textit{A. crassus}, which succeeded in North-America. Since the 1990s
populations of the American eel (\textit{Anguilla rostrata}) have been
invaded as novel hosts \cite{fries_notes:_1996,barse_exotic_1999,
  barse_swimbladder_2001}. Wielgoss et al. identified Japan as the
most likely source of this American population of
\textit{A. crasssus}.

Finally \textit{A. crassus} has been detected in three indigenous
species of freshwater eels on the island of Reunion near Madagascar
\cite{sasal_parasite_2008}.

Copepods and ostracods serve as intermediate hosts of
\textit{A. crassus} in Asia, as well as in the introduced ranges
\cite{moravec_first_2005}. In these L2 larvae develop to L3 larvae,
infective to the final host. Once ingested by an eel they migrate
through the intestinal wall and via the body cavity into the
swimbladder wall \cite{haenen_effects_1996}, i.a. using a trypsin-like
proteinase\cite{polzer_identification_1993}. In the swimbladder wall
L3 larvae hatch to L4 larvae. After a final moult from L4 to adults
(via a short preadult stage) the parasites inhabit the lumen of the
swimbladder, where they eventually mate. Eggs containing L2 larvae are
released via the ductus pneumaticus into the eels gut and finally into
the water\cite{de_charleroy_life_1990}.

\figuremacro{l_cycle}{Live-cycle of \textit{A. crassus}}{Adult females
  deposit already hatched L2 in the lumen of the swimbladder. Larvae
  migrate through the \textit{ductus pneumaticus} and the intestine
  into the open water.Copepodes serve as intermediate host where
  infectiv L3-larvea develop. These can be transported and accumulated
  in paratenic hosts or directly ingested by an eel. They migrate
  through the eel's intestinal wall into the swimbladder wall. After
  the final molt to adults worms arrive in the lumen of the
  swimbladder, feed on blood and reproduce. Modified from
  \cite{mnderle_kologische_2005}.}

One of the possible differences between Asian and European population
of \textit{A. crassus} is the widespread use of paratenic hosts in
European waters
\cite{thomas_paratenic_1992,pietrock_dynamics_2002}. Such a use of
paratenic hosts has not been from the Asian range of the parasite yet
and there are some speculation that the use and availability of
paratenic hosts could be a factor explaining the success of invasion
or even the higher epidemiological parameters in Europe
\cite{thomas_paratenic_1992}. However the lack of evidence for the use
of paratenic host in Asia could as well be a a result of the lack of
appropriate studies in Asian water systems.

The impact of \textit{A. crassus} on the European eel has been a major
focus of research during the past decades. Pathogenic effects on the
eels such as a thinkening \cite{wurtz_tara_2000} and infammation
\cite{beregi_radiodiagnostic_1998} of the swimbladder wall, can lead
to mortality of eels, when combined with co-stressors
\cite{gollock_physiological_2005}. Especially the changes in the
tissue of the swimbladder wall have been shown to influence swimming
behavior and it has been speculated that eel may fail to complete
theier spawning migration
\cite{palstra_swimming_2007}. Anguillicolosis (the condition caused by
\textit{A. crassus}) has therefore been speculated to be a cofactor in
the decline of European eel stocks \cite{sures_science_letter} caused
by overfishing of glass-eels \cite{pmid12713741}.

High prevalences of the parasite of above 70\%
(e.g. \cite{wrtz_distribution_1998}), as well as high intesities of
infections were reported, throughout the newly colonized area
\cite{lefebvre_anguillicolosis:_2004}.

These differences in abundance of \textit{A. crassus} in East Asia
compared to Europe are commonly attributed to the different
host-parasite relations in the final eel host permitting a
differential survival of the larval and the adult parasites
\cite{knopf_differences_2004}. Recently, data from experimental
infections of European eels with \textit{A. crassus} have been
published \cite{fazio_regulation_2008}. They show that the parasite
undergoes (under experimental conditions) a density-dependent
regulation keeping the number of worms within a certain range. As in
the natural host in Asia prevalences and intesities are lower
\cite{mnderle_occurrence_2006}, high epidemiological parameters were
attributed to the inadequate immune-response of the European Eel
\cite{knopf_swimbladder_2006}. Interestingly the differences in the
two host also affect the size and life-history of the worm: In
European eels the nematodes are bigger and develop and reproduce
faster \cite{knopf_differences_2004}.  While the Japanese eel is
capable of killing larvae of the parasite after vaccination
\cite{knopf_vaccination_2008} or under high infection pressure
\cite{heitlinger_massive_2009}.

\figuremacro{worm_diff}{Difference between worms in the swimbladder of
  the European eel and the Japanese eel}{Note the bigger size and
  higher number of worm in a typically infected European eel. In
  comparison in the Japanese eel worms are smaller and intensities of
  infection are much lower. The dark brown matter is ingested
  eel-blood visible through the transparent nematode body- and
  intestinal wall, the white matter are developing eggs and larvae in
  ovaries of female \textit{A. crassus}.}


\subsection{Evolutionary significance}
\label{ev-sig}

A decline of \textit{A. crassus} populations has been hypothised
looking at populations over the last two decades
\cite{lefebvre_anguillicolosis:_2004}. However it should be noted,
that such a decline has not been confirmed in a explicit meta-analysis
and even if present could be explained rather by a lower population
density of the host, than by an evolution of \textit{A. crassus}
towards lower infectivity.

With a view on the potential co-evolution (i.e. adaptation), of the
eel-host to \textit{A. crassus}, that European and American eels are
considered panmictic \cite{pmid20735687}: Signals for population
structure, interpreted as evidence against panmixia first
\cite{pmid11234011}, have been shown to be an artifact of temporal
variation between cohorts of juvinile eels \cite{pmid19417764,
  pmid21299662, pmid16024374}. Such panmixia would severly reduce the
effectiveness of selection, when unifected populations are
participating in reproduction. While


\subsubsection{Divergence of \textit{A. crassus} populations}
\label{div-ac}

Common-garden experiments (also termed ``transplant expreiments'') are
a method to identify genetic components of phenotypic differences
between potentally diverged population of a species, used for almost
as long as scientists investigate evolution
\cite{kerner_classic_common_garden, bonnier_classic_common_garden}. In
the reciprocal version of these experiments, representatives of each
population intented to be studied are raised under experimental
conditions resembling the other population's natural environment.

 ``cross-inoculation experiment under common garden
conditions'' \cite{kaltz_shykoff_rev}


\figuremacro{ula_stages}{Differences in developmental speed}{data
  courtesy of Urszula Weclawski
  %% \href{http://someurl}{urlanchor<}.
  %% \figuremacroW{ula_stages.png}{Title}{Caption}{0.8}
  % variation of the above macro with a width setting
}


Although such experiments have their problems because environmental
factors.



\subsubsection{Interest in \textit{A. crassus} based on its
    phylogeny}
\label{phyl-int}
  In a recent study on the phylogeny of the genus
  \textit{Anguillicola} we identified \textit{A. crassus} as the most
  basal species in the genus.

  The phylogeny is inconsistent with a

  The genus \textit{Anguillicola} holds a phylogenetic position basal
  to the Spirurina (clade III \textit{sensu} Blaxter
  \cite{blaxter_molecular_1998}), one of 5 major clades of nematodes
  \cite{nadler_molecular_2007, wijov_evolutionary_2006}. The Spirurina
  exclusively exhibit a parasitic lifestyle and comprise improtant
  human pathogens as well as prominent parasites of livestock
  (e.g. the Filaroidea and Ascarididae). This phylogenetic position
  makes the Anguillicoloidae an interesting system in the endeavour to
  understand the emergence of parasitism in Spirurina and as an
  ``outgroup'' for functional studies of parasitism in this
  clade. 


  Some functionally interesting genes in this respect are thought to
  be under diversifying selection in an arms-race between host and
  parasite \cite{zang_serine_2001}.

  That positive or diversifying selection on parasite proteins from
  the host-parasite interface can lead to a overabundance of
  non-synonymous changes (altering the protein sequence) over
  synonymous polymorphisms e.g. in \textit{Plasmodium}
  \cite{pmid7630387}.

  Memory componetnt of the vertebrate immune system has been thought
  to be a driving positive selection on antigenes of microorganisms
  \cite{conway_measuring_2002}. The immune systems of teleost has a
  immune system with interstin implications for the eel’s response to
  parastites.


\section{Functional insights from other nematodes used to formulate
  hypotheses for \textit{A.crassus}}
\label{func-ins}

In 1998 \textit{Caenorhabditis elegans} became the first multicellular
organism with a sequenced genome \cite{pmid9851916}. Soon it was
noted, that in addition to it's use as a general model system for the
metazoa, knowledge gained in this species has the potential to be even
more valuable in the phylum Nematoda
\cite{blaxter_caenorhabditis_1998}. The breadth of genomic information
available for \textit{C. elegens} to date is illustrated by a recent
publication of the Gerstein et al. \cite{pmid21177976}: detailed
annotation of the diverse functional genomic elements and their
interactions by the modENCODE consortium.

The complete genome sequence of the nematode \textit{Caenorhabditis
  elegans} \cite{pmid9851916} and \textit{Caenorhabditis briggsae}
\cite{stein_genome_2003}, as well as the draft genome of
\textit{Brugia malayi} \cite{ghedin_draft_2007} provide useful sources
for mining databases for homologous sequences.

Emerging genomes form \textit{Trichinella spiralis}
\cite{pmid21336279}, \textit{Meloidogyne incognita}
\cite{pmid18660804}, \textit{Meloidogyne hapla} \cite{pmid18809916}
\textit{Pristionchus pacificus} the pinewood nematode
\textit{Bursaphelenchus xylophilus} \cite{pmid21909270} 

The lack of genomic information in many species of nematodes promoted
use of ESTs as a tool for gene-discovery and partial genomes
\textit{sensu} \cite{parkinson_partigene--constructing_2004} were
successfully interrogated for a large array of genes interesting for
different scientific communities. In nematode parasites of
vertebrates, pathogenic factors were described as potential vaccine
candidates \cite{pmid11406138}.

Cystein-proteinase inhibitors (cystatins), Serin protenase inhibitors
(serpins) were identifided 

Homologues of mammalian cytokines were identified, which are believed
to interact with receptors of mammalian

The abundant larval transcripts of \textit{B. malay} (Bm-ALT) have
been identified in the transcriptome-projects first CITE! as a gene
family  \cite{gomez-escobar_abundant_2002}

Bm-VAL-1

\cite{blaxter_genes_1996, daub_survey_2000,
  parkinson_transcriptomic_2004, mitreva_gene_2004})

In some of these studies secreted proteins were in the center of
interest. They could potentially be excreted by the nematode to
interact with the host's immune system. The detection of
signal-peptides for secretion using \textit{in silico} analysis of
ESTs has been used to highlight candidate genes for example in
\textit{Nippostrongylus brasiliensis} \cite{harcus_signal_2004}, and
across all nematode ESTs \cite{nagaraj_needles_2008}.

Proteomic analysis in \textit{Brugia malayi}
\cite{pmid19352421,pmid18958170}, \textit{Heligmosomoides polygyrus}
\cite{pmid21722761} and \textit{Haemonchus contortus}
\cite{pmid12576473} was able to find evidence for excretion for some
of the protein-products and to highlight additional candidate genes.



\section{Advances in sequencing technology enabeling this study}
\label{ad-seq}

For almost tree decades the method developed by Frederick Sanger
\cite{pmid271968} was the only practical choice for determining the
sequence of nucleic acid. Although modern machines use the
chain-termitaion method combined with capillary gel electorphoresis
\cite{pmid2326186} in a highly paralized way, costs and labour
constrained sequencing to a well established laboratory-model
organisms (the bacterium \textit{Escherichia coli}, 1997
\cite{pmid9278503}; the baker's yeast \textit{Saccharomyces
  cerevisiae}, 1996 \cite{pmid8849441}; the nematode
\textit{Caenorhabditis elegans} 1998 \cite{pmid9851916}, the fruit fly
\textit{Drosophila melanogaster}, 2000 \cite{adams2000genome}; the
mouse \textit{Mus musculus}, 2002 \cite{pmid12466850}; to name a few
together with the year of publication their genome sequence).

This ``first generation of genomics'' culminated in the publication of
the human genome in 2001 \cite{pmid11181995}.

\figuremacro{sequencing_costs}{Falling seqeuncing costs}{Sequencing
  costs falling due to advances in Solexa-sequencing: Due to improved
  read-length and data-volume on this plattform per base
  sequencing-prices for many applications thumble into free fall. Data
  provided by \href{http://www.genome.gov/sequencingcosts/}{National
    Human Genome Research Institute, NHGRI}.}

In this context a common characteristic of all DNA-sequencing methods
has to be emphasized: Read-length is usually shorter than the length
of the target molecule to be sequenced. This potential problem is
solved by oversampling the target molecule, producing overlapping
sequence. The amount of redundancy of the overlap is termed coverage
(e.g. 10-fold coverage means a base is sequenced 10 times redundantly)
the method as such is referred to as shotgun-sequencing and has -
shortly after sequencing chemestry - been described by Sanger
\cite{pmid6260957}. Soon copmuter programs were necessary to align
sequences, to compute overlaps and consensus sequences
\cite{pmid461197} and the process of computationally reconstructing
the target molecule was termed sequence-assembly \cite{pmid6251542}.

Advances in sequencing technology (often termed ``Next Generation
Sequencing''; NGS), provide the opprotunity for rapid and
cost-effective generation of genome-scale data. The technologies
described as NGS have in common that they use radically new chemistry
comapared to the Sanger-method, up to date all these methods produce
much more, but shorter reads than classical sanger sequencing. This
fostered use and development of new methods to assemble large-scale
shotgun sequences, as higher coverage but shorter read-length (and
also lower accuracy) are increasing the computational complexity of
the assembly-problem (reviewed in \cite{pmid20211242}).

NGS technologies are are increasinly used in studies on organisms with
ecological and evolutionary significance. Such ecological and
evolutionary ``model organisms'' often lack reference genomes to guide
the assembly-process.

``Genome-scale'' sequencing in the broadest context can also mean
sequenching comprehensive transcriptome datasets: Such transcriptomic
datasetes are still less expensive than genomic data-sets in terms
sequencing costs and analytical needs.




\subsection{Pyro-sequencing}
\label{pyro-seq}

see also \ref{nom}

The longer read length of 454-sequencing \cite{pmid16056220} compared
to other NGS technologies, allows \textit{de novo} assembly of
Expressed Sequence Tags (ESTs) in organisms lacking previouse genomic
or transcriptomic data (for a comprehensive list of studies using this
approach before Oct 2010 see \cite{pmid20950480}).

A study on trout in Lake Superior \cite{pmid20331779} used an approach
similar to the appoach in the work presented here: Fish show two
different phenotypes were raised in a common environment,
demonstrating the genetic fixation of the phenotypic trait. 454
sequencing was then used to measure the gene expression levels and to
indentify 40 genes from two pathways being differently expressed and
therefore showing divergent evolution of gene-expression.


\subsection{Illumina-Solexa sequencing}
\label{ill-seq}




As shorter read-length but higher throughput of the Illumina-Solexa
platform provides superior means for gene expression analyis
\cite{pmid21627854}:

RNA-seq \cite{pmid19015660}

Expression-tags (SuperSAGE \cite{pmid20967605}) provide the benefit of
classical SAGE-analysis \cite{pmid7570003} with those of the ulta
hight throughput of Illumina-Solexa sequencing.


\section{Gene-expression and evolutionary divergence}

Today, both theoretical arguments as well as field and laboratory data
suggest that evolution, including divergence of populations, can occur
very rapidly given the right selective pressure. Such situations
provide us with the opportunity of examining how divergence and even
speciation work at the molecular genetic level
\cite{via_ecological_2002} .

Divergence in gene-expression is thought to be a factor for the
establishment of reproductive barriers through hybrid

In \textit{Drosophila} the effect of cis- and trans-regulatory
differences \cite{pmid20354124}

In \textit{Drosophila} hybrid sterility in hybrids between species
\cite{pmid16757655}



% ----------------------------------------------------------------------

%%% Local Variables: ***
%%% mode:latex ***
%%% TeX-master: "../thesis.tex"  ***
%%% tex-main-file: "../thesis.tex" ***
%%% End: ***
     