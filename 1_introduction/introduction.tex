
% this file is called up by thesis.tex
% content in this file will be fed into the main document

%: ----------------------- introduction file header -----------------------
\chapter{Introduction}

% the code below specifies where the figures are stored
\ifpdf
    \graphicspath{{1_introduction/figures/PNG/}{1_introduction/figures/PDF/}{1_introduction/figures/}}
\else
    \graphicspath{{1_introduction/figures/EPS/}{1_introduction/figures/}}
\fi

% ----------------------------------------------------------------------
%: ----------------------- introduction content ----------------------- 
% ----------------------------------------------------------------------



%: ----------------------- HELP: latex document organisation
% the commands below help you to subdivide and organise your thesis
%    \chapter{}       = level 1, top level
%    \section{}       = level 2
%    \subsection{}    = level 3
%    \subsubsection{} = level 4
% note that everything after the percentage sign is hidden from output



\section{The study organism: \textit{Anguillicola crassus}} 

\subsection{Ecological significance} 

\textit{Anguillicola crassus} Kuwahara, Niimi and Ithakagi 1974
\cite{kuwahara_Niimi_Itagaki_1974, moravec_anguillicoloides} is a
swimbladder nematode naturally parasitizing the Japanese eel
(\textit{Anguilla japonica}) indigenous to East-Asia. In the last 30
years anthropogenic expansions of its geographic- and host-range to
new continents and host-species (all freshwater eels of the genus
\textit{Anguilla}) attracted interest of limnologists and ecologists.

First \textit{A. crassus} colonized Europe in the eraly 1980ies and
colonized almost all populations of the European eel (\textit{Anguilla
  anguilla}) in the following decades (reviewed in
\cite{kirk_impact_2003}):

Wielgoss et al. \cite{wielgoss_population_2008} studied the population
structure of \textit{A. crassus} using microsattelite markers and
inferred details about the colonization process. From the fact that
genetic diversity is highest in northern regions of Germany, and
gradually declines to the south they concluded a single introduction
event \cite{wielgoss_population_2008} to Germany. This is in agreement
with the first record of \textit{A. crssus} in 1982 in North-West
Germany \cite{fischer_teichwirt} and the import of Japanese Eels from
Taiwan in 1980 having been identified as most likely introduction
event \cite{koops_anguillicola-infestations_1989}. Taiwan as the most
likely geographical source of the introduction was 


At the present day \textit{A. crassus} is found in all but the
northernmost population of the European eel in Iceland
\cite{kristmundsson_parasite_2007}

A second colonization of \textit{A. crassus} succeeded in
North-America Since the 1990s populations of the American eel
(\textit{Anguilla rostrata}) have been colonized as novel hosts
\cite{fries_notes:_1996,barse_exotic_1999, barse_swimbladder_2001} and
finally it has been detected in three indigenous \textit{Anguilla}
species on the island of Reunion near Madagascar
\cite{sasal_parasite_2008}.\

In Asia, as well as in the introduced ranges, copepods and ostracods
serve as intermediate hosts of \textit{A. crassus}
\cite{moravec_first_2005}, in which L2 larvae develop to L3 larvae,
infective to the final host. Once ingested by an eel they migrate
through the intestinal wall and via the body cavity into the
swimbladder wall \cite{haenen_effects_1996}, i.a. using a trypsin-like
proteinase\cite{polzer_identification_1993}. In the swimbladder wall
L3 larvae hatch to L4 larvae. After a final moult from L4 to preadult
the parasites inhabit the lumen of the swimbladder, where they
eventually mate. Eggs containing L2 larvae are released via the ductus
pneumaticus into the eels gut and finally into the
water\cite{de_charleroy_life_1990}.\

Within the novel range and hosts, conspicuously elevated prevalences
and intensities of infection occur (reviewed in
\cite{kirk_impact_2003} and \cite{taraschewski_hosts_2007}). These
differences in abundance of \textit{A. crassus} in East Asia compared
to Europe are commonly attributed to the different host-parasite
relations in the final eel host permitting a differential survival of
the larval and the adult parasites
\cite{knopf_differences_2004}. Recently, data from experimental
infections of European eels with \textit{A. crassus} have been
published \cite{fazio_regulation_2008}. They show that the parasite
undergoes (under experimental conditions) a density-dependent
regulation keeping the number of worms within a certain range.\

The impact of \textit{A. crassus} on the European eel has been a major
focus of research during the past decades. High prevalences of the
parasite of above 70\% (e.g. \cite{wrtz_distribution_1998}), as well
as high intesities of infections were reported, throughout the newly
colonized area \cite{lefebvre_anguillicolosis:_2004}.  Based on a
broad base of work on its epidemiology \textit{A. crassus} can be
regarded as a model for parasite introduction and spread
\cite{taraschewski_hosts_2007}.

As in the natural host in Asia prevalences and intesities are lower
\cite{mnderle_occurrence_2006}, high epidemiological parameters were
attributed to the inadequate immune-response of the European Eel
\cite{knopf_swimbladder_2006}. Interestingly the differences in the
two host also affect the size and life-history of the worm: In
European eels the nematodes are bigger and develop and reproduce
faster \cite{knopf_differences_2004}.  While the Japanese eel is
capable of killing larvae of the parasite after vaccination
\cite{knopf_vaccination_2008} or under high infection pressure
\cite{heitlinger_massive_2009}, only pathological effects such a
thikening of the swimmbladder wall \cite{wrtz_histopathological_2000}
have been found in the European eel.



\subsection{Evolutionary significance}

\subsubsection{Divergence of \textit{A. crassus} populations}

\figuremacro{ula_stages}{Differences in developmental speed}{data
  courtesy of Urszula Weclawski
  %% \href{http://someurl}{urlanchor<}.
  %% \figuremacroW{ula_stages.png}{Title}{Caption}{0.8}
  % variation of the above macro with a width setting
}

Today, both theoretical arguments as well as field and laboratory data
suggest that evolution, including speciation, can occur very rapidly
given the right selective pressure. Such situations provide us with
the opportunity of examining how evolution and speciation work at the
molecular genetic level (Via 2002).


\subsubsection{Interest in \textit{A. crassus} based on its
    phylogeny}

  In a recent study on the phylogeny of the genus
  \textit{Anguillicola} we identified \textit{A. crassus} as the most
  basal species in the genus.

  The phylogeny is inconsistent with a


  The genus \textit{Anguillicola} holds a phylogenetic position basal
  to the Spirurina (clade III \textit{sensu} Blaxter
  \cite{blaxter_molecular_1998}), one of 5 major clades of nematodes
  \cite{nadler_molecular_2007, wijov_evolutionary_2006}. The Spirurina
  exclusively exhibit a parasitic lifestyle and comprise improtant
  human pathogens as well as prominent parasites of livestock
  (e.g. the Filaroidea and Ascarididae). This phylogenetic position
  makes the Anguillicoloidae an interesting system in the endeavour to
  understand the emergence of parasitism in Spirurina and as an
  ``outgroup'' for functional studies of parasitism in this
  clade. Some functionally interesting genes in this respect are
  thought to be under diversifying selection in an arms-race between
  host and parasite\cite{zang_serine_2001}.



\subsection{Functional insights from other nematodes used to formulate
  hypotheses for \textit{A.crassus}}

The analysis of ESTs, especially in nematode parasites, has been
employed to identify pathogenic factors as potential vaccine
candidates in numerous studies. (Blaxter 1995; Blaxter et al. 1996;
Daub et al. 2000; Blaxter 2000; Harcus et al. 2004; Mitreva et
al. 2004a; Mitreva et al. 2004b; Mitreva et al. 2005).


The complete genome sequence of the nematode Caenorhabditis elegans
(The C. elegans sequencing consortium 1998) and Caenorhabditis
briggsae (Stein et al. 2003), as well as the draft genomic assembly of
Brugia malayi (Ghedin et al. 2007) provide useful sources for mining
databases for homologous sequences. Brugia


\section{Advances in sequencing technology enabeling this study}


Recent advances in sequencing technology (often termed Next Generation
Sequencing; NGS), provide the opprotunity for rapid and cost-effective
generation of genome-scale data.

\figuremacro{sequencing_costs}{Falling seqeuncing
  costs}{Sequencing costs falling due to advances in
  Solexa-sequencing, due to improved read-length and data-volume on
  this plattform, Data provided by
  \href{http://www.genome.gov/sequencingcosts/}{National Human Genome
    Research Institute, NHGRI}.}

\subsection{Pyro-sequencing}

The longer read length of 454-sequencing \cite{pmid16056220} compared
to other NGS technologies, allows \textit{de novo} assembly of
Expressed Sequence Tags (ESTs) in organisms lacking previouse genomic
or transcriptomic data (for a comprehensive list of studies using this
approach before Oct 2010 see \cite{pmid20950480}).

Such transcriptomic datasetes are still less expensive than genomic
data-sets in terms sequencing costs and analytical needs.


\subsection{Illumina-Solexa sequencing}

As shorter read-length but higher throughput of the Illumina-Solexa
platform provides superior means for gene expression analyis
\cite{pmid21627854}:

Expression-tags (SuperSAGE \cite{pmid20967605}) provide the benefit of
classical SAGE-analysis  .

RNA-seq \cite{pmid19015660}

\begin{table}[htdp]
\centering
\begin{tabular}{ccc} % ccc means 3 columns, all centered; alternatives are l, r

{\bf Gene} & {\bf GeneID} & {\bf Length} \\ 
% & denotes the end of a cell/column, \\ changes to next table row
\hline % draws a line under the column headers

human latexin & 1234 & 14.9 kbps \\
mouse latexin & 2345 & 10.1 kbps \\
rat latexin   & 3456 & 9.6 kbps \\
% Watch out. Every line must have 3 columns = 2x &. 
% Otherwise you will get an error.

\end{tabular}
\caption[title of table]{\textbf{title of table} - Overview of latexin genes.}
% You only need to write the title twice if you don't want it to appear in bold in the list of tables.
\label{latexin_genes} % label for cross-links with \ref{latexin_genes}
\end{table}

% There you go. You already know the most important things.


% ----------------------------------------------------------------------

%%% Local Variables: ***
%%% mode:latex ***
%%% TeX-master: "../thesis.tex"  ***
%%% End: ***
     